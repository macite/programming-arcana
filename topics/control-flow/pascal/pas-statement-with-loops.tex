\clearpage
\subsection{Pascal Statement (with loops)} % (fold)
\label{sub:pas_statement_with_loops_}

In addition to the \nameref{sub:procedure call} and \nameref{sub:assignment_statement}, Pascal statements may be  \nameref{sub:branching}, \nameref{sub:looping}, or \nameref{sub:jump} statements.

\passyntax{psynt:looping-statement}{a statement (with branches and loops)}{control-flow/statement-with-loops}

\mynote{
\begin{itemize}
  \item See the following diagrams for details on this syntax:
  \begin{itemize}
    \item \nameref{sub:pas_if_statement}: for the syntax of an \nameref{sub:if_statement}.
    \item \nameref{sub:pas_case_statement}: for the syntax of an \nameref{sub:case_statement}.
    \item \nameref{sub:pas_while_loop}: for the syntax of an \nameref{sub:pre_test_loop}.
    \item \nameref{sub:pas_repeat_loop}: for the syntax of an \nameref{sub:post_test_loop}.
    \item \nameref{sub:pas_jump_statements}: for the \nameref{sub:jump} statements.
  \end{itemize}
  \item These statements can be coded within functions, procedures, and programs.
\end{itemize}
}

% subsection c_statement_with_loops_ (end)