Read over the concepts in this chapter and answer the following questions:
\begin{enumerate}
  \item What values can a Boolean expression have?
  \item What are the values of the following expressions?
  
  \begin{table}[h]
    \centering
    \begin{tabular}{|c|l|l|}
      \hline
      \textbf{Question} & \textbf{Expression} & \textbf{Given} \\
      \hline
      (a) & \texttt{1 > 5} & \\
      \hline
      (b) & \texttt{1 < 5} & \\
      \hline
      (c) & \texttt{a > b} & \texttt{a} is 1 and \texttt{b} is 2 \\
      \hline
      (d) & \texttt{(a > b) or (b > a)} & \texttt{a} is 1 and \texttt{b} is 2 \\
      \hline
      (e) & \texttt{(a > b) and (b > a)} & \texttt{a} is 2 and \texttt{b} is 2 \\
      \hline
      (f) & \texttt{a or b or c} & \texttt{a} is False, \texttt{b} is False, and \texttt{c} is True\\
      \hline
      (g) & \texttt{(a or b) and (c or d)} & \texttt{a} is False, \texttt{b} is True, \texttt{c} is True, and \texttt{d} is False \\
      \hline
      (h) & \texttt{(a or b) and (c or d)} & \texttt{a} is False, \texttt{b} is True, \texttt{c} is False, and \texttt{d} is False \\
      \hline
      (i) & \texttt{(a and b) or (c and d)} & \texttt{a} is False, \texttt{b} is True, \texttt{c} is True, and \texttt{d} is True \\
      \hline
      (j) & \texttt{a xor b} & \texttt{a} is True and \texttt{b} is True\\
      \hline
      (k) & \texttt{(a or b) and (not (a and b))} & \texttt{a} is True and \texttt{b} is True\\
      \hline
      (l) & \texttt{not True} & \\
      \hline
      (m) & \texttt{a and (not b)} & \texttt{a} is True and \texttt{b} is False\\
      \hline
    \end{tabular}    
  \end{table}
  
  \csection{
    The following table shows the C syntax for these boolean expressions.
    
    \begin{tabular}{|c|l|}
      \hline
      \textbf{Question} & \textbf{Expression} \\
      \hline
      (a) & \texttt{1 > 5} \\
      \hline
      (b) & \texttt{1 < 5} \\
      \hline
      (c) & \texttt{a > b} \\
      \hline
      (d) & \texttt{(a > b) || (b > a)} \\
      \hline
      (e) & \texttt{(a > b) \&\& (b > a)} \\
      \hline
      (f) & \texttt{a || b || c} \\
      \hline
      (g) & \texttt{(a || b) \&\& (c || d)} \\
      \hline
      (h) & \texttt{(a || b) \&\& (c || d)} \\
      \hline
      (i) & \texttt{(a \&\& b) || (c \&\& d)} \\
      \hline
      (j) & \texttt{a \^{} b} \\
      \hline
      (k) & \texttt{(a || b) \&\& (!(a \&\& b))} \\
      \hline
      (l) & \texttt{!false} \\
      \hline
      (m) & \texttt{a \&\& (!b)}\\
      \hline
    \end{tabular}
  }
  
\end{enumerate}


Apply what you have learnt to the following tasks:
\begin{enumerate}
  \item 
\end{enumerate}

If you want to further your knowledge in this area you can try to answer the following questions. The answers to these questions will require you to think harder, and possibly look at other sources of information.
