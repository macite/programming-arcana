\clearpage
\subsubsection{If Statement} % (fold)
\label{sub:if_statement}

The if statement is the most frequently used branching statement. It allows you to selectively run code based on the value of a Boolean expression (the condition). The if statement has an optional \emph{else} branch that is executed when the condition is false.

\begin{figure}[h]
   \centering
   \includegraphics[width=\textwidth]{./topics/control-flow/diagrams/IfStatement} 
   \caption{If statement lets you selectively run a branch of code}
   \label{fig:branching-if-statement}
\end{figure}

\mynote{
\begin{itemize}
  \item An if statement is an \textbf{action}. It allows you to command the computer to select a path based on a Boolean expression.
  \item The if statement has two branches, one that is taken when the condition is True, the other when it is False.
  \item The False branch may \emph{optionally} have instructions that are carried out when the condition is False. 
  \item If there are no instructions you want performed when the condition is False you do not need to include an else branch, and the if statement will just skip the True branch when the condition is False.
  \item The if statement has one entry point, two paths, and then one exit point.
\end{itemize}
}


% subsection if_statement (end)