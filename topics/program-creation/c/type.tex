\clearpage
\subsection{C Types} % (fold)
\label{sub:program-creation-c_types}

\nameref{sub:type}s are used to define how data is interpreted and the operations that can be performed on the data. Table \ref{tbl:program-creation-c-types} shows the three basic types of data, the associated C type, size in memory, and other related information. Table \ref{tbl:program-creation-c operators by type} shows the operators that are permitted for each Type.

\begin{table}[h] 
\begin{minipage}{\textwidth}
\centering
\begin{tabular}{|l|c|c|c|}
\hline
\multicolumn{4}{|c|}{\textbf{Whole Number Types}} \\
\hline
\emph{Name} & \emph{Size} & \multicolumn{2}{|c|}{\emph{Range (lowest .. highest)}} \\
\hline
\texttt{short} & 2 bytes/16 bits & \multicolumn{2}{|c|}{-32,767 .. 32,767} \\
\texttt{int} & 4 bytes/32 bits & \multicolumn{2}{|c|}{-2147483648 .. 2147483647} \\
\texttt{long long}    & 8 bytes/64 bits & \multicolumn{2}{|c|}{-9,223,372,036,854,775,807 ..} \\
  & & \multicolumn{2}{|c|}{9,223,372,036,854,775,807} \\
\hline
\multicolumn{4}{c}{} \\
\hline
\multicolumn{4}{|c|}{\textbf{Real Number Types}} \\
\hline
\emph{Name} & \emph{Size} & \emph{Range (lowest .. highest)} & \emph{Significant Digits} \\
\hline
\texttt{float} & 4 bytes/32 bits & 1.0e-38 .. 1.0e38 & 6 \\
\texttt{double} & 8 bytes/64 bits & 2.0e-308 .. 2.0e308 & 10 \\
\hline
\multicolumn{4}{c}{} \\
\hline
\multicolumn{4}{|c|}{\textbf{Text Types}} \\
\hline
\emph{Name} & \emph{Size} & \multicolumn{2}{|c|}{\emph{Known As}} \\
\hline
\texttt{char}  & 1 byte/8 bits & \multicolumn{2}{|c|}{} \\
\hline
\texttt{char*} & various\footnote{1 byte per character + 1 byte overhead} &  \multicolumn{2}{|c|}{c-string} \\
\hline
\end{tabular}
\caption{C Data Types}\label{tbl:program-creation-c-types}
\end{minipage}
\end{table}

\begin{table}[h]
  \centering
  \begin{tabular}{|c|c|l|}
    \hline
    \textbf{Type} & \textbf{Operations Permitted} & \textbf{Notes}\\
    \hline
    Whole Numbers     &   \texttt{( ) + - / * \%} & Division rounds down if all\\
                                    &                        & values are whole numbers.\\
    Real Numbers   &   \texttt{( ) + - / *} &    \\
       & & \\
    Text           &   \texttt{( ) }          & You cannot perform mathematical\\
    & & operations on text.\\
    \hline
  \end{tabular}
  \caption{C Permitted Operators by Type}
  \label{tbl:program-creation-c operators by type}
\end{table}

\csyntax{csynt:program-creation-typed-literal}{Typed Literals}{program-creation/typed-literal}

\mynote{
\begin{itemize}
  \item The \texttt{int} type is the typical whole number type.
  \item The \texttt{double} type is the typical real number type.
  \item C has limited support for text data. In most languages, text is represented using a \texttt{String} type. The C text type is named \texttt{c-string} to indicate this limited support. C includes a \nameref{sub:library} to add operations to manipulate \texttt{c-string} values.
  \item For example values see Table \vref{tbl:program-creation-c example expresions}.
\end{itemize}
}
% subsection c_types (end)