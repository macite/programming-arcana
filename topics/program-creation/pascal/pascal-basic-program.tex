\clearpage
\section{Pascal Program} % (fold)
\label{sec:pascal-basic-program}

Pascal includes syntax to allow you to declare a program in your code. The code for a Pascal program is written

Your program contains the instructions that are followed when the user runs the executable file. In many cases you will want to use code from other \nameref{sec:unit}s. This is achieved using the uses clause. By default all program's have access to the \texttt{System} unit, which contains \nameref{sec:writeln} and many other useful procedures.

\subsection*{Pascal Syntax} % (fold)
\label{sub:programsgeneralform}

\clearpage
\subsection{Program} % (fold)
\label{sub:program}

A program contains the instructions the computer will follow when that program is executed. In your source code you can declare a program in which you code the steps you want followed when your program is executed. When you compile this code the compiler will create an executable file (a \emph{program}) that the user can run. Running the program will then get the computer to perform the steps you wrote in the code.

\begin{figure}[h]
   \centering
   \includegraphics[width=\textwidth]{./topics/program-creation/diagrams/BasicProgramConcept} 
   \caption{A program contains instructions that command the computer to perform actions}
   \label{fig:program-creation-program}
\end{figure}


\mynote{
\begin{itemize}
  \item A program is an \textbf{artefact}, something you can create in your code.
  \item Figure \ref{fig:program-creation-program} shows the concepts related to programs.
  \item A program is a programming artefact used to define the steps to perform when the program is run.
  \item You use the compiler to convert the program's source code into an executable file.
  \item By declaring a program in your code you are telling the compiler to create a file the user can run.
  \item The program has an \textbf{entry point} that indicates where the program's instructions start.
  \item The name of the program determines the name of the executable file.
  \item Your program can use code from a \nameref{sub:library} or number of libraries.
  \item In programming terminology, an instruction is called a \nameref{sub:statement}.
\end{itemize}
}

% section program (end)
% subsection general_form (end)

\subsection*{Example} % (fold)
\label{sub:programExamples}

The code in Listing \ref{lst:pascal-program-creation-HelloWorld} and Listing \ref{lst:pascal-program-creation-ClrScr} show two example programs. Listing \ref{lst:pascal-program-creation-HelloWorld} contains the instructions that for a program which greets the world when it is executed, printing the text \emph{Hello World} to the Terminal. The program in Listing \ref{lst:pascal-program-creation-ClrScr} shows how to use code from the CRT \nameref{sec:unit}, the code for the \texttt{ClrScr} procedure is contained in the \texttt{CRT} Unit.

\lstinputlisting[caption={The classic ``Hello World'' program.},label={lst:pascal-program-creation-HelloWorld}]{./topics/program-creation/pascal/HelloWorld.pas}
\lstinputlisting[caption={Example of unit use, using the \texttt{CRT} unit},label={lst:pascal-program-creation-ClrScr}]{code/pascal/program-creation/ClearScreen.pas}

% subsection program_examples (end)

\mynote{
\begin{itemize}
    \item The program in Listing~\ref{lst:HelloWorld} is named `HelloWorld', see \nameref{sec:identifier}.
    \item A program's instructions start at the \texttt{begin} \nameref{sec:keyword}.
    \item The `Hello World' program in Listing~\ref{lst:HelloWorld} has one statement that uses \nameref{sec:writeln}.
    \item The statements are grouped within a \emph{block} that ends with the \texttt{end} \nameref{sec:keyword}.
    \item Program's end at the full stop (at \texttt{end.}).
    \item \nameref{sec:writeln} comes from the System unit, to use code in other units you need to include the uses clause.
\end{itemize}
}
% subsection program_study (end)

% section program (end)