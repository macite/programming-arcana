\clearpage
\subsection{Pascal Procedure Call} % (fold)
\label{sub:program-creation-pas_procedure_call}

A procedure call allows you to run the code in a procedure, getting its instructions to run before control returns back to this point in the program.

\passyntax{passynt:program-creation-procedure-call}{Procedure Call Syntax}{program-creation/procedure-call}

\passection{\pascode{lst:program-creation-pas-count-back}{Pascal Count Back}{code/pascal/program-creation/CountBack.pas}}

\mynote{
\begin{itemize}
  \item A procedure call is an \textbf{action} which commands the computer to run the code in a procedure.
  \item The procedure call starts with the procedure's \nameref{sub:identifier}, this indicates the name of the procedure to be called.
  \item Following the identifier is a list of values within parenthesis, these are the values (coded as \nameref{sub:expression}s) that are passed to the procedure for it to use.
  \item The code in \lref{lst:program-creation-pas-count-back} contains a \nameref{sub:program_in_pas}.
  \item This program contains four procedure calls.
  \item Each procedure call runs the \texttt{WriteLn} procedure to output text to the terminal. See the section on \nameref{sub:pas_console_output}.
  
\end{itemize}
}


% subsection c_procedure_call (end)