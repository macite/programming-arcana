\clearpage
\subsection{Pascal Statement} % (fold)
\label{sub:program-creation-pas_statement}

In a \nameref{sub:statement} you are commanding the computer to perform an \emph{action}. There are only a small number of statements you can choose from. At this stage the only statement we have discussed is the \nameref{sub:procedure call}, formally known as the \emph{procedure statement} in Pascal. This is shown in Figure \ref{csynt:program-creation-statement}, where we can see that at this stage all Statements are calls to \nameref{sub:procedure}s.

\passyntax{passynt:program-creation-statement}{Statement Syntax}{program-creation/statement}

\passection{\pascode{lst:program-creation-pas-knights}{Pascal Knights}{code/pascal/program-creation/Knights.pas}}

\mynote{
\begin{itemize}
  \item The code in Listing \ref{lst:program-creation-pas-knights} contains a \nameref{sub:program_in_pas}.
  \item This Program contains five procedure calls. See \nameref{sub:program-creation-pas_procedure_call}.
  \item Each procedure call runs the \texttt{printf} procedure to output text to the Terminal. See the section on \nameref{sub:pas_console_output}.
\end{itemize}
}


% subsection c_statement (end)