\clearpage
\subsection{Procedure Call} % (fold)
\label{sub:procedure call}

A procedure call is a kind of \nameref{sub:statement} that instructs the computer to run the code in a \nameref{sub:procedure}. If the procedure requires some data to work with, then this data is passed to the procedure as part of the procedure call. The procedure's name is used to identify the procedure to call.

\begin{figure}[h]
   \centering
   \includegraphics[width=\textwidth]{./topics/program-creation/diagrams/ProcedureCall} 
   \caption{A procedure calls runs a procedure, passing in values for the procedure to use}
   \label{fig:program-creation-procedure call}
\end{figure}


\mynote{
\begin{itemize}
  \item A procedure call is an \textbf{action}, you can call procedures in your code.
  \item Figure \ref{fig:program-creation-procedure call} shows the concepts related to the procedure call.
  \item A procedure call is an instruction to execute a procedure.
  \item You can code a procedure anywhere you can code a statement.
  \item The \nameref{sub:identifier} indicates the \nameref{sub:procedure} to call.
  \item Data values passed to the procedure are coded using \nameref{sub:expression}s.
  \item When the procedure's task is complete the program continues with the next \nameref{sub:statement}.
\end{itemize}
}

% section program (end)