\setcounter{chapter}{-1}
\chapter{Getting Started} % (fold)
\label{cha:getting-started}

\begin{quote}
  \Fontlukas\Large
  \renewcommand{\LettrineTextFont}{\relax}
  \lettrine[image=true,lines=3,lraise=0.1]
  {M}{agic} requires both knowledge and tools. Our first lesson will uncover the tools of the Magi. Tools that you will need to use to practice magic. Here, take this ancient wand this orb and caldron. Each of these tools is essential to the working of even the most basic spells. Now lets see if you can wield that wand. Take it in your hand like this, and \ldots
\end{quote}

\bigskip

Software development requires both knowledge and tools. This first chapter introduces key details of your computer, how programs are created and run, and the tools you will use to build your own software.

When you have understood the material in this chapter you will be able to create programs from source code, run your programs, and identify cause of errors when creating or running your program.

\minitoc

% ====================================
% = Learning Focus - Getting Started =
% ====================================
\clearpage
\section{Learning Focus}
Programming is about providing instructions to get a computer to do the things that you want. In order to learn how to do this we need to start by understanding a little about the computer and how it works. This first chapter focuses on developing your understanding of how the computer works, how programs are created, and how they run. When you have successfully understood the material in this chapter you will be able to:

\begin{itemize}
  \item Describe key components of your computer, and how they relate to program execution.
  \item Describe the environment within which programs execute, and how this can impact on programs working.
  \item Setup a computer to build and run programs.
  \item Run programs from the command line.
  \item Setup and use a code editor to create small programs.
  \item Compile programs from the command line.
\end{itemize}

Once you have achieved these learning outcomes the next chapter will help you start to explore the key concepts for the components within a program.

% =====================================
% = Concepts - Compilers and Programs =
% =====================================

\section{Concepts Related to Getting Started}
\label{sec:concepts_related_to_getting_started}

There are two big questions we need to address here:

\begin{enumerate}
  \item What are programs?
  \item How can you create a program?
\end{enumerate}

\subsection{What are programs?}

\subsection{How can you create a program?}


\section{Understanding Getting Started}
\label{sec:understanding_getting_started}



\section{Secrets of the Magi -- Advanced Shell Usage}

