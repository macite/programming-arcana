\subsection{Concept Questions} % (fold)
\label{sub:concept_questions_types}

Read over the concepts in this chapter and answer the following questions:
\begin{enumerate}
  \item What is a type?
  \item What is the relationship between a type and a value? 
  \item When you create your own type what have you created? A value, or something else?
  \item Why would you want to create your own type?
  \item What are the three main kinds of type you can create?
  \item What kind of data type(s) could you create to model the following in a program?
  \begin{enumerate}
    \item An address book, containing names, phone numbers, and email addresses.
    \item The kind of a `\emph{power up}' in a game, e.g. health pack, upgrade, bonus, etc.
    \item A field that is either an integer, a double, or some text.
    \item A button that has a location on the screen, a width and height, and some text that is drawn on the button.
  \end{enumerate}
  \item What is a record? What can it be used to model?
  \item What is an enumeration? What can it be used to model?
  \item What is a union? What can it be used to model?
  \item Why is it a good idea to use an enumeration in conjuncture with a union?
  \item Explain the different ways you can store/read a value when you are using a record.
  \item Explain the different ways you can store/read a value when you are using a enumeration.
  \item Explain the different ways you can store/read a value when you are using a union.
  \item Open one of your SwinGame projects and have a look in the \texttt{lib} folder. This folder contains a number of source code files used to access SwinGame functionality. Have a look in the \textbf{Types} file (types.c or sgTypes.pas), and examine the follow types. For each type write a short description of what it contains.
  \begin{enumerate}
    \item Rectangle
    \item Circle
    \item LineSegment
    \item Triangle
    \item Point 2D
    \item Vector
  \end{enumerate}
\end{enumerate}
% subsection concept_questions (end)

\clearpage
\subsection{Code Writing Questions: Applying what you have learnt} % (fold)
\label{sub:code_writing_questions_applying_what_you_have_learnt_types}

Apply what you have learnt to the following tasks:

\begin{enumerate}
  \item Create a \texttt{Contact} record that stores a person's name (50 characters), their phone number (20 characters), and email address (50) characters. 
  \item Use the \texttt{Contact} record to create a small address book program that lets you enter in the details of four contacts, and then outputs these to the Terminal.
  \item Implement the Lights program from \sref{sub:lights}.
  \item Implement the Shape Drawing program, add the code to create Ellipse and Triangle shapes.
  \item Implement the Small DB program, including the support for the double data in the row.
\end{enumerate}
% subsection code_writing_questions_applying_what_you_have_learnt (end)

\bigskip
\subsection{Extension Questions} % (fold)
\label{sub:extension_questions_types}

If you want to further your knowledge in this area you can try to answer the following questions. The answers to these questions will require you to think harder, and possibly look at other sources of information.

\begin{enumerate}
  \item Extend the Small DB program so that each `\emph{row}' has three `\emph{columns}'.
  \item Explore the sizes of the different data types you have created using the \texttt{Size Of} function: \texttt{sizeof} in C, or \texttt{SizeOf} in Pascal.
\end{enumerate}
% subsection extension_questions (end)