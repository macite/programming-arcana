\clearpage
\subsection{Return Statement} % (fold)
\label{sub:return_statement}

\csyntax{csynt:function-decl-return-statement}{a Return Statement}{function-decl/return-statement}

\csection{\ccode{clst:test-return}{Example illustrating return in action.}{code/c/storing-using-data/test-return.c}}

\mynote{
\begin{itemize}
  \item The Return Statement is used to end a Function, or Procedure, and to return a value.
  \item Procedures can end completing all of their instructions.
  \item Functions that return a value must have a \texttt{return} to indicate the value to be returned by the caller.
  \item The \emph{Expression} in the Return Statement is optional so that you can use the statement to end a Procedure.
  \item When the Return Statement is executed the current Function or Procedure ends, and the value of the \emph{Expression} is returned to the Function Call.
  \item Listing \ref{clst:test-return} illustrates the point that \texttt{return} ends the current function, with only the first \texttt{printf} call being run in the \texttt{test\_return()} Function.
\end{itemize}
}

% subsection return_statement (end)