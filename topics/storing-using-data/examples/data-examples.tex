\subsection{Times Table} % (fold)
\label{sub:times_table}

This program prints out the times table for a number entered by the user, displaying from 1 x n to 10 x n. The description of the program is in Table \ref{tbl:data-times-table}, the pseudocode in Listing \ref{lst:data-times-pseudo}, the C code in Listing \ref{lst:data-times-c}, and the Pascal code in Listing \ref{lst:data-times-pas}.

\begin{table}[h]
\centering
\begin{tabular}{l|p{10cm}}
  \hline
  \multicolumn{2}{c}{\textbf{Program Description}} \\
  \hline
  \textbf{Name} & \emph{Times Table} \\
  \\
  \textbf{Description} & Displays the Times Table from 1 x n to 10 x n. \\
  \hline
\end{tabular}
\caption{Description of the Times Table program}
\label{tbl:data-times-table}
\end{table}

\pseudocode{lst:data-times-pseudo}{Pseudocode for Times Table program.}{./topics/storing-using-data/examples/times-table.txt}

\mynote{
This is an updated version of the Seven Times Table Program. See Section \ref{sub:seven_times_table} \nameref{sub:seven_times_table}.
}

\clearpage

\csection{\ccode{lst:data-times-c}{C Times Table}{topics/storing-using-data/examples/times_table.c}}

\passection{\pascode{lst:data-times-pas}{Pascal Times Table}{topics/storing-using-data/examples/TimesTable.pas}}

% subsection times_table (end)

\clearpage
\subsection{Circle Area} % (fold)
\label{sub:circle_area_data}

This program prints out the area of a circle. The description of the program is in Table \ref{tbl:data-circle-area}, the pseudocode in Listing \ref{lst:data-circle-areas-pseudo}, the C code in Listing \ref{lst:data-circle-areas-c}, and the Pascal code in Listing \ref{lst:data-circle-areas-pas}.

\begin{table}[h]
\centering
\begin{tabular}{l|p{10cm}}
  \hline
  \multicolumn{2}{c}{\textbf{Program Description}} \\
  \hline
  \textbf{Name} & \emph{Circle Areas} \\
  \\
  \textbf{Description} & Displays the Circle Areas for circles with radius from 1.0 to 5.0 with increments of 0.5. \\
  \hline
\end{tabular}
\caption{Description of the Circle Areas program}
\label{tbl:data-circle-area}
\end{table}

\pseudocode{lst:data-circle-areas-pseudo}{Pseudocode for Circle Areas program.}{./topics/storing-using-data/examples/circle_areas.txt}

\mynote{
This is an updated version of the Circle Areas Program. See Section \ref{sub:circle_area} \nameref{sub:circle_area}.
}


\clearpage

\csection{\ccode{lst:data-circle-areas-c}{C Circle Areas}{topics/storing-using-data/examples/circle_areas.c}}

\passection{\pascode{lst:data-circle-areas-pas}{Pascal Circle Areas}{topics/storing-using-data/examples/CircleAreas.pas}}

% subsection circle_area (end)
