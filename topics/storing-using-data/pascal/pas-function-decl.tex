\clearpage
\subsection{Pascal Function Declaration} % (fold)
\label{sub:pas_function_declaration}

\passyntax{passynt:function-decl-function-decl}{a function}{function-decl/function-decl}

\passection{\pascode{plst:square}{Example function declaration of a \texttt{square} function.}{code/pascal/storing-using-data/TestSquare.pas}}

\mynote{
\begin{itemize}
  \item In Pascal \nameref{sub:function} and \nameref{sub:procedure} declarations are very similar.
  \item In Pascal, a function's declaration includes the \nameref{sub:type} of data the function will return.
  \item This if followed by the name of the Function, and its Parameters. In the same way as is done in the \nameref{sub:c_procedure_declaration_with_parameters_}.
  \item The body of the function is the same as a \nameref{sub:pas_procedure_declaration_with_parameters_}.
  \item See \nameref{sub:pas_function_call} for the syntax needed to call functions.
  \item Each Pascal function has a special \texttt{result} variable, the value of this variable is returned to the caller when the function ends.
\end{itemize}
}

% subsection c_procedure_declaration (end)