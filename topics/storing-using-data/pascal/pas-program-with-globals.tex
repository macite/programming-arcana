\clearpage
\subsection{Pascal Program (with Global Variables and Constants)} % (fold)
\label{sub:pas_program_with_global_variables}

You can declare global variables and constants within a Pascal Program file.

\passyntax{passynt:storing-using-data-program}{a program (with global variables and constants)}{storing-using-data/program-with-globals}

\mynote{
\begin{itemize}
  \item This syntax allows you to declare \nameref{sub:global_variable}s and Constants.
  \item See Listing \ref{lst:variable-test-pas} for an example of declaring Global Variable and Constants.
  \item In Listing \ref{lst:variable-test-pas} \ldots
  \begin{itemize}
    \item \texttt{globalFloat} and \texttt{globalInt} are global variables. These can be accessed in both the \texttt{test} procedure and \texttt{main} procedure.
    \item \texttt{PI} is a global constant, with the value 3.1415. This can be read in both the \texttt{test} procedure and \texttt{main} procedure.
  \end{itemize}
  \item Global variable should be avoided.
  \item There are a number of conventions, called coding standards, that describe how your code should appear for a given language. In this text we will use a common Pascal convention of having all \emph{Constants} in \textbf{UPPER CASE}, with underscores ( \_ ) used to separate words. So the \emph{Maximum Height} constant becomes \texttt{MAXIMUM\_HEIGHT}.
\end{itemize}
}

\clearpage

\passection{\pascode{lst:variable-test-pas}{Example variable and constant declarations}{code/pascal/storing-using-data/VariableTest.pas}}

% \clearpage
% 
% \csection{\ccode{clst:def-consts}{C program with a defined constant, and constant variable}{code/c/storing-using-data/defined-consts.c}}

% subsection c_program_with_global_variables (end)