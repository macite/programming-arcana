
\chapter{Dynamic Memory Allocation} % (fold)
\label{cha:dynamic_memory_allocation}

\begin{quote}
  \Fontlukas\Large
  \renewcommand{\LettrineTextFont}{\relax}
  \lettrine[image=true,lines=3,lraise=0.1]
  {T}{here} are many places you can draw upon to power your spells. So far you have been limited by the constraints of this realm. The tools I give you now will let you stretch beyond this realm, and will open your mind to even greater powers. You will need your orb, your wand, and \ldots
\end{quote}

\bigskip

So far data has been limited by the constraints of the Stack. With the stack, the compiler must know how much space to allocate to each variable ahead of time. This means you are limited to working with a fixed number of values, whether those values are stored in a number of variables or stored in an array. This constraint is not a problem for small programs, but most programs will require the flexibility to work with a variable number of data elements.

This chapter introduces the tools needed to dynamically allocate additional memory for your program to use. With these tools you will be able to dynamically allocate additional space for your program to use as you need it. As memory is finite you will also see how you can release this memory back to the computer when you no longer require it.

When you have understood the material in this chapter you will be able to dynamically allocate memory for your program, increasing and decreasing the number of values that you are storing.

\minitoc

% ========================================
% = Concepts - Dynamic Memory Allocation =
% ========================================

\clearpage
\section{Dynamic Memory Allocation Concepts} % (fold)
\label{sec:dynamic_memory_allocation_concepts}

\subsection{Heap} % (fold)
\label{sub:heap}

When your program is executed it allocated memory to work with. This memory is divided into different areas based on the kind of values that will be stored there. Previously all of the program's data was housed on the Stack, dynamically allocated memory is allocated into a separate area known as the Heap. Any memory that you allocate to your program will come from this location.

\begin{figure}[h]
   \centering
   \includegraphics[width=0.7\textwidth]{./topics/dynamic-memory/diagrams/Heap} 
   \caption{The Heap is used to store all dynamically allocated values}
   \label{fig:heap}
\end{figure}

\mynote{
\begin{itemize}
  \item \fref{fig:heap} includes the following areas:
  \begin{enumerate}
    \item Your program's machine code is loaded into the \textbf{Code Area}.
    \item The \textbf{Stack} is used to manage the execution of the program's functions and procedures.
    \item \textbf{Global Variables} are allocated their own space.
    \item The new area is the \textbf{Heap}. This is used to store all dynamically allocated values.
  \end{enumerate}
  \item Values can be stored in the \emph{global variables}, in local variables on the \emph{Stack}, and on the \emph{Heap} using dynamic memory allocation functions and procedures.
  \item The space taken up by the \textbf{global variables} is fixed based on the size of the variables you have declared.
  \item Each function/procedure takes a fixed amount of space on the stack. The space allocated is enough to store each of the local variables, plus some additional space for various overheads.
  \item The compiler take care of managing memory in the stack and for the global variables.
  \item \textbf{You} are responsible for any memory allocation done on the heap.
\end{itemize}
}

\clearpage
\subsubsection{Allocating memory on the heap} % (fold)
\label{ssub:allocating_memory_on_the_heap}

Dynamic memory allocation is performed with a couple of operations that will be provided by the programming language you are using. These operations allow you to do the following:

\begin{itemize}
  \item \textbf{Allocate Space}: You ask the Operating System to allocate you some space into which you want to store a certain value. The Operating System will then allocate you space on the Heap that is large enough to store the value you require.
  \item \textbf{Free Allocation}: When you have finished using a piece of memory you have been allocated on the Heap, you can tell the operating system that you have finished with this memory, and that it is free to allocate this to some other value.
\end{itemize}

These are the two basic actions that exist for performing dynamic memory management. Basically, you can ask for memory, and you can give it back. Once you have been allocated space, that space will be reserved for your use until you free that allocation. So it is important to remember to free the memory you have been allocated when you no longer require it.

\begin{figure}[h]
   \centering
   \includegraphics[width=0.75\textwidth]{./topics/dynamic-memory/diagrams/HeapAlloc} 
   \caption{You can ask for space, and return the space you were allocated}
   \label{fig:heap-alloc}
\end{figure}

\mynote{
\begin{itemize}
  \item \fref{fig:heap-alloc} shows the idea behind the two operations.
  \item You can ask to be allocated space, this will give you access to a space on the heap. You can then use this to store a value.
  \item You can tell the Operating System when you are finished with the space, so that it can allocate it to something else.
\end{itemize}
}


% subsubsection allocating_memory_on_the_heap (end)
\clearpage
\subsubsection{Accessing dynamically allocated memory} % (fold)
\label{ssub:accessing_dynamically_allocated_memory}

By its very nature, dynamic memory allocation must work a little differently to the way we have been working with values so far. So far, when you wanted to work with a value you declared a variable, or an array. This would have been a \nameref{sub:local_variable}, with its value allocated on the stack along with the other variables you were working with in the current function or procedure. The variable and its value were closely related, with the value being located within the variable. With dynamic memory allocation the values you are allocated are on the heap. This means that their values are not bound within a variable, but exist entirely outside of any variables that appear in your code.

One of the challenges of working with dynamically allocated memory is that you can no longer `\emph{see}' these values in your code. When you were working with variables, they were in the code, you could see them and think about the value they stored. With dynamically allocated memory you do not have this advantage, these values will be allocated as a result of the operations that are performed while the code is running. This is why it is called \textbf{dynamically} allocated memory. It is \emph{not} memory allocated to variables, it is \textbf{memory allocated upon request}.

This raises one very important question, as illustrated in \fref{fig:heap-access}:
\begin{quote}
  \emph{If the values exist outside of variables, how do you access them?}
\end{quote}
For this we require a new kind of data, a new \nameref{sub:type}. This type is used to store a value that tells you where the data you want can be located. It is like an address, telling you where the data can be found. This is the \nameref{sub:pointer} type.

\begin{figure}[h]
   \centering
   \includegraphics[width=0.7\textwidth]{./topics/dynamic-memory/diagrams/HeapAccess} 
   \caption{How can you access these dynamically allocated values?}
   \label{fig:heap-access}
\end{figure}




% subsubsection accessing_dynamically_allocated_memory (end)

% subsection heap (end)
\clearpage
\subsection{Pointer} % (fold)
\label{sub:pointer}

A Pointer is a new kind of data type, just like Integer, Double, and Boolean. A Pointer Value is an address, a location in memory where a value can be found. The name `\emph{Pointer}' is very descriptive, a \emph{Pointer} points to a value. It tells you, `The data I refer to is over there...'.

\begin{figure}[h]
   \centering
   \includegraphics[width=\textwidth]{./topics/dynamic-memory/diagrams/Pointer} 
   \caption{A Pointer Value is the address of a value, in effect it \emph{points} to a value}
   \label{fig:pointer}
\end{figure}

\mynote{
\begin{itemize}
  \item A Pointer is an \textbf{existing artefact}, a data type that is built into the Programming Language.
  \item A Pointer's Value points to a value.
  \item The CPU architecture tells you the size of its pointers. A 32bit machine has 32bit pointers. A 64bit machine has 64bit pointers.
  \item It is a good idea to picture a pointer as a value that \emph{points} to a location in memory.
  \item Pointers are used to point to locations in the Heap, when you are allocated space the Operating System will give you a Pointer Value that you can use to get to that location.
\end{itemize}
}

\clearpage
\subsubsection{What can a pointer point to?} % (fold)
\label{ssub:what_can_a_pointer_point_to_}

Pointers store a value that is an address of the value that it points to. This means that you can point to \emph{any} value in memory, regardless of where it is. You can have Pointer values that point  to \nameref{sub:local_variable}s, \nameref{sub:global_variable}s, \nameref{sub:parameter}s, \nameref{sub:array} elements, fields of \nameref{ssub:record}s or \nameref{ssub:union}s. One of its key ability, however, is the ability to point to values on the \nameref{sub:heap}.

\begin{figure}[h]
   \centering
   \includegraphics[width=\textwidth]{./topics/dynamic-memory/diagrams/PointerPointing} 
   \caption{A Pointer can point to any value, at any location in memory}
   \label{fig:pointer-pointing}
\end{figure}

\mynote{
\begin{itemize}
  \item Languages usually require you to declare the kind of data that a Pointer Value will refer to. So rather than just having a generic \emph{pointer}, you will have things like a \emph{pointer to an Integer}, or a \texttt{pointer to a User Data value}. This makes it easier to work out what you can do with the value the pointer points to.
\end{itemize}
}

% subsubsection what_can_a_pointer_point_to_ (end)

\clearpage
\subsubsection{Where can pointer values be stored?} % (fold)
\label{ssub:where_can_pointer_values_be_stored_}

A Pointer value is the same as any other value. It can be stored in \nameref{sub:local_variable}s, \nameref{sub:global_variable}s, it can be passed to a function in a \nameref{sub:parameter}, it can be returned from a \nameref{sub:function}, and it can also exist on the \nameref{sub:heap}.

\fref{fig:pointer-values} shows an illustration of some values in memory. The \texttt{start} variable is located somewhere on the stack as a local variable.  This variable is storing a pointer value that points to a \texttt{Node}\footnote{The \texttt{Node} would be a record type declared in the code. This type would contain an Integer value field named \texttt{data}, and a pointer field named \texttt{next}.} that is on the Heap. Each of the nodes on the heap are also storing pointer values that refer to other values that are also on the heap.

\begin{figure}[h]
   \centering
   \includegraphics[width=\textwidth]{./topics/dynamic-memory/diagrams/PointerValues} 
   \caption{Pointers can be stored anywhere a value can be stored}
   \label{fig:pointer-locations}
\end{figure}

\mynote{
\begin{itemize}
  \item A pointer value is no different from any other value, and can be stored on the stack, the heap, or in global variables.
  \item Languages provide a special value for pointers that do not point to a value. In C this is the \texttt{NULL} value, in Pascal it is the \texttt{nil} value, in both cases it is a value that points to nothing.
\end{itemize}
}


% subsubsection where_can_pointer_values_be_stored_ (end)

\clearpage
\subsubsection{How are pointers used?} % (fold)
\label{ssub:how_are_pointers_used_}

You need to be able to perform certain actions to make pointers useful. These include:

\begin{itemize}
  \item You must be able to get a pointer to a value. For example, you should be able to get a pointer to a value stored in a Variable.
  \item Once you have a Pointer value, you must be able to follow that pointer to its value so that you can \ldots
  \begin{itemize}
    \item Read the value it points to.
    \item Store a value at the location pointed to.
  \end{itemize}
\end{itemize}

\begin{figure}[h]
   \centering
   \includegraphics[width=0.97\textwidth]{./topics/dynamic-memory/diagrams/PointerActions} 
   \caption{You can get pointers to values, and you can follow pointers to values}
   \label{fig:pointer-actions}
\end{figure}

\mynote{
\begin{itemize}
  \item You can get the address of values in \nameref{sub:local_variable}s, \nameref{sub:global_variable}s, \nameref{sub:parameter}s, fields of \nameref{ssub:record}s and \nameref{ssub:union}s. Basically, you can get the address of any value you can read.
  \item Once you have the address (the Pointer value), you can store, or you can use it.
  \item You need to follow the pointer, called \textbf{dereferencing} the pointer, to read its value or to assign a new value to the location it refers to.
  \item Remember there are two values with pointers:
  \begin{enumerate}
    \item There is the value of the pointer itself. This is the address that is pointed to. The circle at the start of the line in the illustrations.
    \item There is the value that pointed to. The one at the end of the arrow in the illustrations.
  \end{enumerate}
  \item You can interact with both of these values, depending on whether you \emph{follow} the pointer or use the pointer's value directly.
\end{itemize}
}

% subsubsection how_are_pointers_used_ (end)

\clearpage
\subsubsection{What is the pointer value? What can you do with it?} % (fold)
\label{ssub:what_is_the_pointer_value_what_can_you_do_with_it_}

Memory is laid out as a sequence of bytes, into which values can be stored. The bytes can be thought of as being in a long line, with each being given numbered based on its position in that line. So the first byte would be byte 0, the next is byte 1, the next byte 2, and so on. This number is then the \textbf{address} of that byte. So byte 975708474 is next to byte 975708475, which is next to byte 975708476, etc. This number is also unique, so there is only one byte 975708474. It is this number that is the Pointer value, the number of the byte that it is pointing to.

\fref{fig:pointer-actions} shows an example of memory used by an array of three values. Each value is a Double, so each one occupies 8 bytes. If the first is at address 975708474, then the second starts at address 975708482 (975708474 + 8 bytes). This Figure also shows a pointer, \texttt{p}, that points to this value. That means that \texttt{p} has the value 975708474, being the address of this value, stored within it.

One feature that languages have is called \textbf{pointer arithmetic}. When you add, or subtract, a value from a pointer the compiler will work in terms of the kinds of values the pointer points to. So in \fref{fig:pointer-actions} \texttt{p} is a pointer to a Double, this means that when you add one to p you get the value 975708482, which is 1 \textbf{Double} past \texttt{p}. Therefore, \texttt{p + 2} would be 2 \emph{doubles} past \texttt{p}, at 975708490, and so on.

\begin{figure}[h]
   \centering
   \includegraphics[width=0.97\textwidth]{./topics/dynamic-memory/diagrams/PointerArithmetic} 
   \caption{The pointer value is the \emph{address} of the value it points to}
   \label{fig:pointer-actions}
\end{figure}

\mynote{
\begin{itemize}
  \item Pointer arithmetic is something you need to know exists, but not something that you would work with frequently.
\end{itemize}
}

% subsubsection what_is_the_pointer_value_what_can_you_do_with_it_ (end)

% subsection pointer (end)
\clearpage
\subsection{Allocating Memory} % (fold)
\label{sub:allocating_memory}

% subsection allocating_memory (end)
\clearpage
\subsection{Freeing Memory Allocations} % (fold)
\label{sub:freeing_memory_allocations}

% subsection freeing_memory_allocations (end)
\clearpage
\subsection{Issues with Pointers} % (fold)
\label{sub:issues_with_pointers}

\begin{quote}
\emph{With great power comes great responsibility.}
\end{quote}

Pointers give you great flexibility in your programs, allowing you to allocate your program more memory as you need it, and return that allocation when you are finished with it. Conceptually this seems very simple, but pointers are a source of many issues in programs.

\subsubsection{Access Violations} % (fold)
\label{ssub:access_violations}

The first kind of error you are likely to encounter is caused by trying to accessing memory that does not exist. This will cause your program to crash. \fref{fig:segfault} shows a common example where this occurs. Trying to follow a pointer to \emph{Nothing} will crash the program with an access violation. This applies whether you are reading or writing to the value at the end of the pointer. The common name for this kind of error is a \textbf{segmentation fault}, \emph{segfault} for short.

The only way to avoid these access violations is to \textbf{take care} with your pointers, see \fref{fig:compiler-complaint}. When you start working with pointers you need to go a little slower, and think a little more carefully about what it is you are doing. Having a good understanding of how these dynamic memory allocation tools work is the first step toward achieving this.

\begin{figure}[h]
   \centering
   \includegraphics[width=0.65\textwidth]{./topics/dynamic-memory/diagrams/AccessViolation} 
   \caption{Trying to follow a pointer that goes nowhere is a runtime error}
   \label{fig:segfault}
\end{figure}

\begin{figure}[h]
   \centering
   \includegraphics[width=\textwidth]{./topics/dynamic-memory/images/compiler_complaint} 
   \caption{To avoid access violation, take care with your pointers. From \url{http://xkcd.com/371/}}
   \label{fig:compiler-complaint}
\end{figure}

\mynote{
Here are some tips to help you avoid these access violations:
\begin{itemize}
  \item If there is any doubt, check your pointers before using them.
  \item Always initialise your pointers to \emph{Nothing}, as uninitialised pointers may point to something, but it wont be something useful.
  \item You can not see dynamically allocated memory in your code, so use a pencil and paper to sketch this out as you think through the code.
\end{itemize}
}

% subsubsection access_violations (end)
\clearpage
\subsubsection{Memory Leaks} % (fold)
\label{ssub:memory_leaks}

The next error is one that will not cause your program to crash, but will consume all of the computers memory if it is allowed to run for an extended time. Remember that with dynamic memory allocation you are responsible for releasing the memory back to the system. If you do not do this there will come a time when there is no memory left to allocate\ldots Memory leaks are hard to detect, as they do not cause your program to crash or generate any errors in its calculations. All that happens is that over time it consumes more and more memory.

Once again, the only way to avoid these issues is to \textbf{take care} with your pointers. You need to make sure that you know where the values are allocated, and where they are released. There should be reasons why you would the memory was allocated, and reasons why it is being released.

\begin{figure}[h]
   \centering
   \includegraphics[width=0.8\textwidth]{./topics/dynamic-memory/diagrams/MemoryLeak} 
   \caption{If you \emph{forget} a piece of allocated memory, it can never be freed!}
   \label{fig:memory_leak}
\end{figure}

\mynote{
Here are some tips to help you avoid memory leaks:
\begin{itemize}
  \item Have a clear idea of where memory is allocated, and where it is freed
  \item Think about the pointer values in local variables at the end of each function and procedure. Do any of these values refer to something that no other pointer does? When the function or procedure ends, the variable's value will be lost. If it is the only thing referring to some allocated memory then that memory can no longer be freed, and you have a memory leak.
\end{itemize}
}

% subsubsection memory_leaks (end)
\clearpage
\subsubsection{Accessing Released Memory} % (fold)
\label{ssub:accessing_released_memory}

The next error occurs when you are overly zealous about releasing memory. You must not release memory before you are finished with it. The problem occurs when you continue to access a value, after its memory has been released. This is one of the most difficult problems to locate, as it will not cause any problems initially. 

Take \fref{fig:read_unallocated} as an example. This demonstrates a case where two pointers refer to one value. It is possible to free that value via one pointer, and then forget that the second refers to the same location. When you read the value from \texttt{p2} later, it is \emph{likely} to still be \texttt{10}, so the program will continue to run as normal. The issue will only appear later when something else is allocated to use that piece of memory. All of a sudden the value you thought was allocated to \texttt{p2} is now changing apparently on its own. Worst of all, the actual cause of the bug could be hundred of lines of code away from where the problem appears. This is what makes this kind of error very difficult to find.

The solution, once again, is to \textbf{take care} with pointers.

\begin{figure}[h]
   \centering
   \includegraphics[width=0.75\textwidth]{./topics/dynamic-memory/diagrams/ReadUnallocated} 
   \caption{You can still read values from memory even when they are unallocated...}
   \label{fig:read_unallocated}
\end{figure}

\mynote{
Here are some tips to help you avoid accessing released memory:
\begin{itemize}
  \item When you free memory, spend some time thinking about the things that could be referring to the value you just released.
\end{itemize}
}


% subsubsection accessing_released_memory (end)



% subsection issues_with_pointers (end)
\clearpage
\subsection{Linked List} % (fold)
\label{sub:linked_list}

Pointers and dynamic memory allocation make it possible to store values in new and interesting ways. One way of structuring this data is to dynamically allocate each value, and link these together using pointers. An illustration of this is shown in \fref{fig:linked-list}.

Linked lists have the advantage of being very fast to perform insert and delete actions, when compared with arrays. The disadvantage is an increase in storage size to keep all the pointers, and the fact you must loop through the nodes to access any value in the list. 

\begin{figure}[htbp]
   \centering
   \includegraphics[width=0.90\textwidth]{./topics/dynamic-memory/diagrams/LinkedList} 
   \caption{Illustration of a linked list in memory}
   \label{fig:linked-list}
\end{figure}

\pseudocode{lst:iterate-linked-list}{Pseudocode for looping through a linked list}{topics/dynamic-memory/concepts/list-iterate.txt}

\mynote{
\begin{itemize}
  \item A Linked List is a \textbf{term} given to a certain way data can be structured in memory.
  \item A Linked List has \textbf{Nodes}, the equivalent of the elements of an array.
  \item Each \textbf{Node}, has some data and a pointer to the \textbf{Next} element in the list.
  \item The \textbf{Last} element in the list has \textbf{Nothing} as its next node.
  \item To access a Node in the list you must loop through from the first node until you reach the node you are after.
  \item You can insert and delete elements by changing the links in the list.
  \item If the grey node in \fref{fig:linked-list} is being \emph{inserted} then the previous node must be adjusted to point to it, and it to point to the next element of the list.
  \item If the grey node in \fref{fig:linked-list} is being \emph{deleted} then the previous node changes its link to skip that node and point to the next node in the list.
  \item The pseudocode in \lref{lst:iterate-linked-list} shows the standard way of applying an action to each node of a Linked List.
\end{itemize}
}

% subsection linked_list (end)
\clearpage
\subsection{Summary of Dynamic Memory Allocation Concepts} % (fold)
\label{sub:dymanic_memory_summary}

This chapter has introduced a number of concepts related to working with pointers and performing dynamic memory allocation.

\begin{figure}[htbp]
   \centering
   \includegraphics[width=0.9\textwidth]{./topics/dynamic-memory/diagrams/Summary} 
   \caption{Memory management focuses on allocating memory, releasing this allocation, pointers, and the heap}
   \label{fig:dynamic-memory-summary}
\end{figure}


\mynote{
\begin{itemize}
  \item \textbf{Heap} - an area of memory you can be allocated to store values.
  \item \textbf{Allocate Memory} - gives you ownership of a piece of the heap's memory.
  \item \textbf{Release Memory} - once you own the memory it is yours until it is released. If you forget to release it, it cannot be used by others.
  \item \textbf{Pointers} - are values that point to locations in memory. They store the address of the area of memory they refer to, and are needed to give you access to the heap.
\end{itemize}
}

% subsection summary (end)

% section dynamic_memory_allocation_concepts (end)


% ===================================
% = Using Dynamic Memory Allocation =
% ===================================
\clearpage
\section{Using Dynamic Memory Allocation} % (fold)
\label{sec:using_dynamic_memory_allocation}

Dynamic memory allocation makes it possible for you to allocate additional space for your program to use from the \nameref{sub:heap}.

\subsection{Designing Linked List} % (fold)
\label{sub:designing_linked_list}

% subsection designing_linked_list (end)

\subsection{Designing Small DB 2} % (fold)
\label{sub:designing_small_db_2}

% subsection designing_small_db_2 (end)

% section using_dynamic_memory_allocation (end)

% ==================================
% = Dynamic Memory Allocation in C =
% ==================================
\cleardoublepage
\def\pageLang{c}
\section{Dynamic Memory Allocation in C} % (fold)
\label{sec:dynamic_memory_allocation_in_c}

\subsection{Small DB 2, the dynamic array version in C} % (fold)
\label{sub:small_db_2_the_dynamic_array_version}

\sref{sec:using_dynamic_memory_allocation}, \nameref{sec:using_dynamic_memory_allocation}, introduced a version of the Small DB program with a dynamic array structure, as opposed to the fixed array structure used to manage the rows in \cref{cha:more_data_types}. The C code for the altered functions and procedures is shown in \lref{clst:dynamic-array-db}, the original version can be found in \lref{lst:c-small-db}.

\straightcode{\ccode{clst:dynamic-array-db}{C code for the dynamic array version of Small DB, see \lref{lst:c-small-db} for the original version of this program}{code/c/dynamic-memory/array-db-for-chap.c}}

\mynote{
\begin{itemize}
  \item This version of the Small DB program includes the ability to add, delete, and print rows from the data store.
  \item The data store includes a dynamic array that is managed using \nameref{ssub:realloc}.
  \item See \sref{sec:using_dynamic_memory_allocation} for a discussion of how this works.
\end{itemize}
}
\subsection{Small DB 2, the linked version} % (fold)
\label{sub:small_db_2_the_linked_version}

% subsection small_db_2_the_linked_version (end)
\sref{sec:using_dynamic_memory_allocation}, \nameref{sec:using_dynamic_memory_allocation}, introduced a version of the Small DB program with a linked structure, as opposed to the array structure used to manage the rows in \cref{cha:more_data_types}. The C code for the altered functions and procedures is shown in \lref{clst:linked-db}, the array version can be found in \lref{lst:c-small-db}.

\straightcode{\ccode{clst:linked-db}{C code for the linked version of Small DB, see \lref{lst:c-small-db} for the array version of this program}{code/c/dynamic-memory/linked-db-for-chap.c}}

\mynote{
\begin{itemize}
  \item \texttt{print\_all}, \texttt{delete\_a\_row}, and \texttt{add\_a\_row} are the only procedures that have changed significantly.
  \item Each of these is explained in more detail in \sref{sec:using_dynamic_memory_allocation}.
  \item Each row has a pointer to the next row in the database, this will point to nothing in the last row.
\end{itemize}
}
\clearpage
\subsection{C Variable Declaration (with pointers)} % (fold)
\label{sub:c_variable_declaration_with_pointers_}

In C you can declare pointer variables. This includes \nameref{sub:local_variable}s, \nameref{sub:global_variable}s, and \nameref{sub:parameter}s.

\csyntax{csynt:var-decl-with-ptr}{variable declarations with pointers}{dynamic-memory/pointer-decl}

\mynote{
\begin{itemize}
  \item This code allows you to declare your own \nameref{sub:pointer} variables in C.
  \item For variable declarations, the main aspect to pay attention to is the \texttt{*}. This indicates that the variable is a pointer.
  \item See \lref{clst:simple-var-ptr} for example variable declarations.
\end{itemize}
}

\begin{figure}[p]
  \csection{\ccode{clst:simple-var-ptr}{C code with pointer variables}{code/c/dynamic-memory/simple_ptr_vars.c}}    
\end{figure}


\subsubsection{Using pointers to emulate pass by reference in C} % (fold)
\label{ssub:using_pointers_to_emulate_pass_by_reference_in_c}

C does not have built in support for \nameref{sub:pass_by_reference}. Instead, in C you must pass a \nameref{sub:pointer} to the variable you want passed to the function or procedure. \lref{clst:simple-var-ptr} shows an examples of procedures that accept pointers variables.

\mynote{
\begin{itemize}
  \item A \texttt{void} pointer can be used to point to \emph{any} value. 
  \item Also see: \sref{sub:c_procedure_call_with_pass_by_reference} \nameref{sub:c_procedure_call_with_pass_by_reference}, and \sref{ssub:c_reference_parameters} \nameref{ssub:c_reference_parameters}.
\end{itemize}
}

% subsubsection using_pointers_to_emulate_pass_by_reference_in_c (end)

% subsection c_variable_declaration_with_pointers_ (end)
\clearpage

\subsection{C Pointer Operators} % (fold)
\label{sub:c_pointer_operators}

C provides a number of pointer operators that allow you to get and use pointers.

\begin{table}[h]
  \centering
  \begin{tabular}{|l|l|l|p{8cm}|}
    \hline
    \textbf{Name} & \textbf{Operator}  & \textbf{Example}  & \textbf{Description} \\
    \hline
    Address Of & \texttt{\&} & \texttt{\&x} & Gets a pointer to the variable/field etc. \\
    \hline
    Dereference & \texttt{*} & \texttt{*ptr} & Follow the pointer, and read the value it points to.\\
    \hline
    & \texttt{->} & \texttt{ptr->field\_name} & Follow a pointer to a struct or union, and read a field value. \\
    \hline
    & \texttt{[]} & \texttt{ptr[2]} & Allows you to access a pointer as if it were an array. \\
    \hline
  \end{tabular}
  \caption{C Pointer Operators}
  \label{tbl:c-ptr-operators}
\end{table}

You can get a pointer to a value using the ampersand operator (\&). This operator lets you get the address of the variable, field, etc of the 

\csection{\ccode{clst:pointer_operators}{C code showing pointer operator usage}{code/c/dynamic-memory/test-ptr-operators.c}}  

\mynote{
\begin{itemize}
  \item The address of operator gets a pointer to the value in the expression that follows it.
  \item Dereference means `\emph{follow the pointer, and read what it points to}'.
  \item Use the asterisks (\texttt{*}) to dereference the pointer and get the value it points to.
  \item You can use \texttt{->} to dereference a pointer to a structure value, and then to read one of the fields from within the structure. 
  \item \lref{clst:pointer_operators} shows how you can get addresses of different variables, and how you can access the value pointed to using \texttt{*} and \texttt{->}.
\end{itemize}
}

% subsection c_pointer_operators (end)
\clearpage
\subsection{C Type Declarations (with pointers)} % (fold)
\label{sub:c_type_decl_with_ptrs}

In C you can declare custom types that make use of pointers. This includes alias types (see \nameref{sub:c_type_declaration}), structs (see \nameref{sub:c_structure_declaration}), and unions (see \nameref{sub:c_union_declaration}).

\csyntax{csynt:type-decl-with-ptr}{array and alias type declarations with pointers}{dynamic-memory/type-decl-with-ptrs}

\mynote{
\begin{itemize}
  \item This is the C syntax to for custom alias type that include \nameref{sub:pointer}s.
  \item The main difference is the inclusion of the \texttt{*} in the \emph{direct type declaration}. This indicates that the custom type can alias pointer types. This would allow you to declare a type such as \texttt{person\_ptr} that is a pointer to a person.
  \item  The inner type declaration allows you to have array of pointers. In these cases you use the brackets to indicate if you want to declare a pointer to an array, or an array of pointers. 
\end{itemize}
}

\begin{figure}[p]
\csection{\ccode{clst:test-type-alias-ptr}{C code demonstrating type aliasing with pointers}{code/c/dynamic-memory/alias-types-with-ptrs.c}}  
\end{figure}


% subsection c_array_declaration (end)
\clearpage
\subsection{C Memory Allocation Functions} % (fold)
\label{sub:c_memory_allocation}

C includes a number of memory allocation functions: \nameref{ssub:malloc}.

\subsubsection{Malloc} % (fold)
\label{ssub:malloc}

\texttt{malloc} is the standard memory allocation function. You tell it how much space you want, and it allocates you that many bytes on the heap. This is a function, that returns a pointer to the space allocated.

\begin{table}[h]
  \centering
  \begin{tabular}{|c|p{9.5cm}|}
    \hline
    \multicolumn{2}{|c|}{\textbf{Function Prototype}} \\
    \hline
    \multicolumn{2}{|c|}{} \\
    \multicolumn{2}{|c|}{\texttt{void *malloc(size\_t size )}} \\
    \multicolumn{2}{|c|}{} \\
    \hline
    \multicolumn{2}{|c|}{\textbf{Returns}} \\
    \hline
    \texttt{void *} & Destination is returned, can be ignored. \\
    \hline
    \textbf{Parameter} & \textbf{Description} \\
    \hline
    \texttt{ size } & The number of bytes to allocate on the heap. \\
    \hline
  \end{tabular}
  \caption{Details of the \texttt{malloc} function}
  \label{tbl:malloc}
\end{table}

\csection{\ccode{clst:malloc}{Example calls to \texttt{malloc}}{code/c/dynamic-memory/malloc-example.c}}  

\mynote{
\begin{itemize}
  \item \texttt{malloc} is used for\emph{memory allocation}.
  \item You need to include \textbf{stdlib.h} to use \texttt{malloc}.
  \item \texttt{malloc} allows you to allocate space on the heap. It returns a pointer to this space.
  \item \texttt{malloc} returns a \texttt{void} pointer, you need to type cast this to the kind of pointer you want, for example \texttt{(int *)} casts it to an integer pointer.
\end{itemize}
}


% subsubsection malloc (end)

\begin{table}[h]
  \centering
  \begin{tabular}{|c|p{9.5cm}|}
    \hline
    \multicolumn{2}{|c|}{\textbf{Function Prototype}} \\
    \hline
    \multicolumn{2}{|c|}{} \\
    \multicolumn{2}{|c|}{\texttt{void *calloc( size\_t num, size\_t size )}} \\
    \multicolumn{2}{|c|}{} \\
    \hline
    \multicolumn{2}{|c|}{\textbf{Returns}} \\
    \hline
    \texttt{void *} & Destination is returned, can be ignored. \\
    \hline
    \textbf{Parameter} & \textbf{Description} \\
    \hline
    \texttt{ num } & The number of elements to allocate to the array.\\
    & \\
    \texttt{ size } & The size of each element to be allocated on the heap. \\
    \hline
  \end{tabular}
  \caption{Details of the \texttt{calloc} function}
  \label{tbl:calloc}
\end{table}



% subsection c_memory_allocation (end)

% section dynamic_memory_allocation_in_c (end)


% =======================================
% = Dynamic Memory Allocation in Pascal =
% =======================================
\cleardoublepage
\def\pageLang{pas}
\section{Dynamic Memory Allocation in Pascal} % (fold)
\label{sec:dynamic_memory_allocation_in_pas}

\subsection{Small DB 2, the dynamic array version in Pascal} % (fold)
\label{sub:pas-small_db_2_the_dynamic_array_version}

\sref{sec:using_dynamic_memory_allocation}, \nameref{sec:using_dynamic_memory_allocation}, introduced a version of the Small DB program with a dynamic array structure, as opposed to the fixed array structure used to manage the rows in \cref{cha:more_data_types}. The Pascal code for the altered functions and procedures is shown in \lref{plst:dynamic-array-db}, the original version can be found in \lref{lst:pas-small-db}.

\straightcode{\pascode{plst:dynamic-array-db}{Pascal code for the dynamic array version of Small DB, see \lref{lst:pas-small-db} for the original version of this program}{code/pascal/dynamic-memory/ArrayDBforChap.pas}}

\mynote{
\begin{itemize}
  \item This version of the Small DB program includes the ability to add, delete, and print rows from the data store.
  \item The data store includes a dynamic array.
  \item See \sref{sec:using_dynamic_memory_allocation} for a discussion of how this works.
\end{itemize}
}
\clearpage
\subsection{Small DB 2, the linked version in Pascal} % (fold)
\label{sub:pas_small_db_2_the_linked_version}

% subsection small_db_2_the_linked_version (end)
\sref{sec:using_dynamic_memory_allocation}, \nameref{sec:using_dynamic_memory_allocation}, introduced a version of the Small DB program with a linked structure, as opposed to the array structure used to manage the rows in \cref{cha:more_data_types}. The Pascal code for the altered functions and procedures is shown in \lref{plst:linked-db}, the original version can be found in \lref{lst:pas-small-db}.

\straightcode{\pascode{plst:linked-db}{Pascal code for the linked version of Small DB, see \lref{plst:dynamic-array-db} for the array version of this program}{code/pascal/dynamic-memory/LinkedDBforChap.pas}}

\mynote{
\begin{itemize}
  \item \texttt{PrintAll}, \texttt{DeleteRow}, and \texttt{AddRow} are the only procedures that have changed significantly.
  \item Each of these is explained in more detail in \sref{sec:using_dynamic_memory_allocation}.
  \item Each row has a pointer to the next row in the database, this will point to nothing in the last row.
  \item The \texttt{DataStore} has a pointer to the first and last rows in the database.
  \item Adding and removing rows is done by changing the links between row values on the heap.
\end{itemize}
}
\clearpage
\subsection{Pascal Variable Declaration (with pointers)} % (fold)
\label{sub:pas_variable_declaration_with_pointers_}

In Pascal you can declare pointer variables and types. Pointer variables can be used to declare \nameref{sub:local_variable}s and \nameref{sub:global_variable}s, but cannot be used as \nameref{sub:parameter}s. To pass a pointer to a parameter you need to declare your own pointer type and use that.

\passyntax{psynt:var-decl-with-ptr}{variable declarations with pointers}{dynamic-memory/pointer-decl}

\mynote{
\begin{itemize}
  \item This code allows you to declare your own \nameref{sub:pointer} variables in C.
  \item For variable declarations, the main aspect to pay attention to is the \texttt{\^}: this indicates that the variable is a pointer.
  \item See \lref{plst:simple-var-ptr} for example variable declarations.
  \item The variables in \texttt{Main} could be declared as \texttt{\^ Integer} or as \texttt{IntPtr}. It is good practice to declare and use your own pointer types, so \texttt{IntPtr} is preferred.
\end{itemize}
}

\passection{\pascode{plst:simple-var-ptr}{Pascal code with pointer variables}{code/pascal/dynamic-memory/SimplePtrVars.pas}}    

% subsection c_variable_declaration_with_pointers_ (end)
\clearpage

\subsection{Pascal Pointer Operators} % (fold)
\label{sub:pas_pointer_operators}

Pascal provides a number of pointer operators that allow you to get and use pointers.

\begin{table}[h]
  \centering
  \begin{tabular}{|l|l|l|p{8cm}|}
    \hline
    \textbf{Name} & \textbf{Operator}  & \textbf{Example}  & \textbf{Description} \\
    \hline
    Address Of & \texttt{@} & \texttt{@x} & Gets a pointer to the variable/field etc. \\
    \hline
    Dereference & \texttt{\^} & \texttt{ptr\^} & Follow the pointer, and read the value it points to.\\
    \hline
  \end{tabular}
  \caption{C Pointer Operators}
  \label{tbl:pas-ptr-operators}
\end{table}

You can get a pointer to a value using the \emph{at} operator (@). This operator lets you get the address of a variable, field, etc.

\passection{\pascode{plst:pointer_operators}{Pascal code showing pointer operator usage}{code/pascal/dynamic-memory/TestPointerOperators.pas}}  

\mynote{
\begin{itemize}
  \item The address of operator gets a pointer to the value in the expression that follows it.
  \item Dereference means `\emph{follow the pointer, and read what it points to}'.
  \item Use the \emph{caret} (\texttt{\^}) to dereference the pointer and get the value it points to.
  \item \lref{plst:pointer_operators} shows how you can get addresses of different variables, and how you can access the value pointed to using \texttt{\^}.
\end{itemize}
}

% subsection c_pointer_operators (end)
\clearpage
\subsection{Pascal Type Declarations (with pointers)} % (fold)
\label{sub:pas_type_decl_with_ptrs}

In Pascal you can declare custom types that make use of pointers.

\passyntax{psynt:type-decl-with-ptr}{array and alias type declarations with pointers}{dynamic-memory/type-decl-with-ptrs}

\passection{\pascode{plst:test-type-alias-ptr}{Pascal code demonstrating type aliasing with pointers}{code/pascal/dynamic-memory/AliasTypesWithPtrs.pas}}  

\clearpage
\subsubsection{Pascal Structure Declarations (with pointer fields)} % (fold)
\label{ssub:pas_structure_declarations_with_pointer_fields_}

The fields of a structure may be pointers. 

\passyntax{psynt:struct-decl-with-ptr}{struct fields with pointers}{dynamic-memory/struct-decl-with-ptr}

\passection{\pascode{plst:simple-node}{Pascal code with a struct that contains a pointer}{code/pascal/dynamic-memory/SimpleNode.pas}}  

% subsubsection c_structure_declarations_with_pointer_fields_ (end)
\clearpage
\subsection{Pascal Memory Allocation Functions} % (fold)
\label{sub:pas_memory_allocation}

Pascal includes a number of memory allocation functions: \nameref{ssub:new}, \nameref{ssub:dispose}, and \nameref{ssub:set_length}.

\subsubsection{New} % (fold)
\label{ssub:new}

In Pascal the \texttt{New} procedure allocates space for a pointer. The amount of memory allocated is based on the size of the type referred to by the pointer, for example an Integer pointer is allocated enough space to store one integer value.

\begin{table}[h]
  \centering
  \begin{tabular}{|c|p{9.5cm}|}
    \hline
    \multicolumn{2}{|c|}{\textbf{Function Prototype}} \\
    \hline
    \multicolumn{2}{|c|}{} \\
    \multicolumn{2}{|c|}{\texttt{procedure New(var ptr: Pointer );}} \\
    \multicolumn{2}{|c|}{} \\
    \hline
    \textbf{Parameter} & \textbf{Description} \\
    \hline
    \texttt{ ptr } & The pointer to allocate the space for. After the call this will point to the allocated memory. \\
    \hline
  \end{tabular}
  \caption{Details of the \texttt{New} procedure}
  \label{tbl:new}
\end{table}

\passection{\pascode{plst:new}{Example calls to \texttt{New}}{code/pascal/dynamic-memory/NewExample.pas}}

\mynote{
\begin{itemize}
  \item \texttt{New} is used for \emph{memory allocation}.
  \item \texttt{New} allows you to allocate space on the heap.
\end{itemize}
}

% subsubsection malloc (end)

\clearpage
\subsubsection{Dispose} % (fold)
\label{ssub:dispose}

When you allocate memory you are responsible for freeing that memory when you no longer require it. The \texttt{Dispose} procedure allows you to do this. 

\begin{table}[h]
  \centering
  \begin{tabular}{|c|p{9.5cm}|}
    \hline
    \multicolumn{2}{|c|}{\textbf{Procedure Prototype}} \\
    \hline
    \multicolumn{2}{|c|}{} \\
    \multicolumn{2}{|c|}{\texttt{procedure Dispose( ptr : Pointer );}} \\
    \multicolumn{2}{|c|}{} \\
    \hline
    \textbf{Parameter} & \textbf{Description} \\
    \hline
    \texttt{ ptr } & The pointer to the space to free on the heap.\\
    \hline
  \end{tabular}
  \caption{Details of the \texttt{Dispose} procedure}
  \label{tbl:dispose}
\end{table}

\mynote{
\begin{itemize}
  \item \texttt{Dispose} allows you to free the memory allocated to a pointer.
  \item See \lref{plst:new} for example code.
  \item \texttt{ptr} is a pointer to a memory block on the heap, i.e. space previously allocated with \nameref{ssub:new}.
  \item You can also pass \texttt{ptr} a \texttt{nil} value, in which case nothing occurs.
  \item It is good practice to assign a \texttt{nil} value to the pointer after freeing it.
\end{itemize}
}

% subsubsection free (end)

\clearpage
\subsubsection{Set Length} % (fold)
\label{ssub:set_length}

Pascal includes support for dynamic arrays. These are arrays where the contents is stored on the heap, and can be dynamically resized during execution using the \texttt{SetLength} procedure.

\begin{table}[h]
  \centering
  \begin{tabular}{|c|p{9.5cm}|}
    \hline
    \multicolumn{2}{|c|}{\textbf{Procedure Prototype}} \\
    \hline
    \multicolumn{2}{|c|}{} \\
    \multicolumn{2}{|c|}{\texttt{procedure SetLength( arr : DynamicArray; len: Integer );}} \\
    \multicolumn{2}{|c|}{} \\
    \hline
    \textbf{Parameter} & \textbf{Description} \\
    \hline
    \texttt{ arr } & The pointer to the space to free on the heap.\\
    & \\
    \texttt{ len } & The new length for the array \texttt{arr}, preserving any existing data up to the new length.\\
    \hline
  \end{tabular}
  \caption{Details of the \texttt{SetLength} procedure}
  \label{tbl:setlength}
\end{table}

\passection{\pascode{plst:new}{Example calls to \texttt{SetLength}}{code/pascal/dynamic-memory/DynamicArrayExample.pas}}  

\mynote{
\begin{itemize}
  \item \texttt{SetLength} allows you to set the length of a dynamic array.
  \item \texttt{High}, \texttt{Low}, and \texttt{Length} determine the valid indexes and length of the array.
  \item Data for a dynamic array is allocated on the heap.
  \item Dynamic arrays are declared without specifying the indexes (just use \texttt{array of type}).
\end{itemize}
}


% subsubsection set_length (end)

% subsection c_memory_allocation (end)

% section dynamic_memory_allocation_in_c (end)




% ===========================================
% = Understanding Dynamic Memory Allocation =
% ===========================================
\clearpage
\def\pageLang{none}

% \section{Understanding Dynamic Memory Allocation} % (fold)
% \label{sec:understanding_dynamic_memory_allocation}
% 
% % section understanding_dynamic_memory_allocation (end)
% 
% % =========================================
% % = Examples of Dynamic Memory Allocation =
% % =========================================
% \clearpage
% \section{Examples of Dynamic Memory Allocation} % (fold)
% \label{sec:examples_of_dynamic_memory_allocation}
% 
% % section examples_of_dynamic_memory_allocation (end)

% ===========================================
% = Exercises for Dynamic Memory Allocation =
% ===========================================
\clearpage
\section{Exercises for Dynamic Memory Allocation} % (fold)
\label{sec:exercises_for_dynamic_memory_allocation}

\subsection{Concept Questions} % (fold)
\label{sub:dynamic_memory_concept_questions}

Read over the concepts in this chapter and answer the following questions:
\begin{enumerate}
  \item What is the difference between the heap and the stack?
  \item Why would you want to allocate space on the heap?
  \item How can you allocate space on the heap?
  \item Why do you need to free the space you are allocated? Why do you not need to do this with values stored on the stack?
  \item What is a pointer?
  \item What can a pointer point to?
  \item Why do you need pointers to make use of the heap?
  \item Where can pointers be stored?
  \item How can you get a pointer to an existing value?
  \item What can you do with the pointer?
  \item The pointer has a value, and points to a value. What is the value of the pointer? How is this different to the value it points to?
  \item What are the different ways you can allocate memory? Describe each, and explain what they can be used for.
  \item What additional issues are you likely to encounter when working with pointers? Explain each, and how you plan to handle these issues.
\end{enumerate}

% subsection concept_questions (end)
\clearpage
\subsection{Code Reading Questions} % (fold)
\label{sub:dynamic_memory_code_reading_questions}

Use what you have learnt to read and understand the following code samples, and answer the associated questions.
\begin{enumerate}
  \item Read the code for the dynamic array version of Small DB 2 (for your language of choice) and do the following:
  \begin{enumerate}
    \item Draw a picture of an empty data store that shows what it looks like in memory.
    \item Draw a new picture showing how the data store will appear after one row is added.
    \item Draw a new picture showing how the data store will appear after three rows have been added.
    \item Explain how the Add Row code is able to add rows to the data store.
    \item Explain how the Delete Row code is able to delete a row from the data store. Include a drawing that illustrates the process.
    \item Explain the steps you would need to perform to add an \emph{Insert Row} option for the user.
  \end{enumerate}
  \item Read the code for the linked version of Small DB 2 (for your language of choice) and do the following:
  \begin{enumerate}
    \item Draw a picture of an empty data store that shows what it looks like in memory.
    \item Draw a new picture showing how the data store will appear after one row is added.
    \item Draw a new picture showing how the data store will appear after three rows have been added.
    \item Explain how the Add Row code is able to add rows to the data store.
    \item Explain how the Delete Row code is able to delete a row from the data store. Include a drawing that illustrates the process.
    \item Explain the steps you would need to perform to add an \emph{Insert Row} option for the user.
  \end{enumerate}
\end{enumerate}

% subsection code_reading_questions (end)
\clearpage
\subsection{Code Writing Questions: Applying what you have learnt} % (fold)
\label{sub:dynamic_memory_code_writing_questions_applying_what_you_have_learnt}

Apply what you have learnt to the following tasks.

\begin{enumerate}
  \item Alter your address book program from \cref{cha:more_data_types}, so that you can enter any number of contacts, and output their details to the Terminal. Use a small menu to allow the user to choose between adding a new contact, printing contacts, and quitting the program.
  
  \item Revisit your statistics program from \cref{cha:managing_multiple_values}.
  \begin{enumerate}
    \item Alter its implementation so that the user can enter a variable number of values. The program can start by asking how many values the user will enter, and sizing the array appropriately.
    \item Add a loop and menu to the program so that the user can add more values, display statistics, or quit.
  \end{enumerate} 
  \item Revisit your small-db program from \cref{cha:more_data_types}.
  \begin{enumerate}
    \item Alter its implementation to introduce a \texttt{Data Store} type that contains a dynamic number of \texttt{row} values.
    \item Introduce a menu with options to allow the user to add a row, delete a row, print all rows, and quit the program.
  \end{enumerate}
\end{enumerate}

% subsection code_questions_applying_what_you_have_learnt (end)

\subsection{Extension Questions} % (fold)
\label{sub:dynamic_memory_extension_questions}

If you want to further your knowledge in this area you can try to answer the following questions. The answers to these questions will require you to think harder, and possibly look at other sources of information.

\begin{enumerate}
  \item Try implementing an alternate approach to deleting a row from the array version of the small db program.
  \item Alter the \texttt{Row} type in your small db program to have a variable number of column values (rename).
  \item Compare the dynamic array and linked versions of the Small DB 2 program. Discuss the relative advantages and disadvantages of each approach.
  \item Test the speed difference between the dynamic array and linked versions of the Small DB 2 program for the following operations:
  \begin{enumerate}
    \item Adding rows (test with adding 10, 100, 1000, and 10000 rows)
    \item Inserting rows (test with inserting 10, 100, 1000, and 10000 rows)
    \item Deleting rows (deleting 10, 100, 1000, and 10000 rows)
  \end{enumerate}
\end{enumerate}

% subsection extension_questions (end)


% section exercises_for_dynamic_memory_allocation (end)

% % ============================================
% % = Dynamic Memory Allocation in the Project =
% % ============================================
% \clearpage
% \section{Dynamic Memory Allocation in the Project} % (fold)
% \label{sec:dynamic_memory_allocation_in_the_project}
% 
% 
% 
% % section dynamic_memory_allocation_in_the_project (end)


% chapter dynamic_memory_allocation (end)