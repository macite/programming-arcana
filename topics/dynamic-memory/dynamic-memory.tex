\chapter{Dynamic Memory Allocation} % (fold)
\label{cha:dynamic_memory_allocation}

So far data has been limited by the constraints of the Stack. With the stack, the compiler must know how much space to allocate to each variable ahead of time. This means you are limited to working with a fixed number of values, whether those values are stored in a number of variables or stored in an array. This constraint is not a problem for small programs, but most programs will require the flexibility to work with a variable number of data elements.

This chapter introduces the tools needed to dynamically allocate additional memory for your program to use. With these tools you will be able to dynamically allocate additional space for your program to use as you need it. As memory is finite you will also see how you can release this memory back to the computer when you no longer require it.

By the end of this chapter you will be able to dynamically allocate memory for your program, increasing and decreasing the number of values that you are storing.

\minitoc

% ========================================
% = Concepts - Dynamic Memory Allocation =
% ========================================

\clearpage
\section{Dynamic Memory Allocation Concepts} % (fold)
\label{sec:dynamic_memory_allocation_concepts}

\subsection{Heap} % (fold)
\label{sub:heap}

When your program is executed it allocated memory to work with. This memory is divided into different areas based on the kind of values that will be stored there. Previously all of the program's data was housed on the Stack, dynamically allocated memory is allocated into a separate area known as the Heap. Any memory that you allocate to your program will come from this location.

\begin{figure}[h]
   \centering
   \includegraphics[width=0.7\textwidth]{./topics/dynamic-memory/diagrams/Heap} 
   \caption{The Heap is used to store all dynamically allocated values}
   \label{fig:heap}
\end{figure}

\mynote{
\begin{itemize}
  \item \fref{fig:heap} includes the following areas:
  \begin{enumerate}
    \item Your program's machine code is loaded into the \textbf{Code Area}.
    \item The \textbf{Stack} is used to manage the execution of the program's functions and procedures.
    \item \textbf{Global Variables} are allocated their own space.
    \item The new area is the \textbf{Heap}. This is used to store all dynamically allocated values.
  \end{enumerate}
  \item Values can be stored in the \emph{global variables}, in local variables on the \emph{Stack}, and on the \emph{Heap} using dynamic memory allocation functions and procedures.
  \item The space taken up by the \textbf{global variables} is fixed based on the size of the variables you have declared.
  \item Each function/procedure takes a fixed amount of space on the stack. The space allocated is enough to store each of the local variables, plus some additional space for various overheads.
  \item The compiler take care of managing memory in the stack and for the global variables.
  \item \textbf{You} are responsible for any memory allocation done on the heap.
\end{itemize}
}

\clearpage
\subsubsection{Allocating memory on the heap} % (fold)
\label{ssub:allocating_memory_on_the_heap}

Dynamic memory allocation is performed with a couple of operations that will be provided by the programming language you are using. These operations allow you to do the following:

\begin{itemize}
  \item \textbf{Allocate Space}: You ask the Operating System to allocate you some space into which you want to store a certain value. The Operating System will then allocate you space on the Heap that is large enough to store the value you require.
  \item \textbf{Free Allocation}: When you have finished using a piece of memory you have been allocated on the Heap, you can tell the operating system that you have finished with this memory, and that it is free to allocate this to some other value.
\end{itemize}

These are the two basic actions that exist for performing dynamic memory management. Basically, you can ask for memory, and you can give it back. Once you have been allocated space, that space will be reserved for your use until you free that allocation. So it is important to remember to free the memory you have been allocated when you no longer require it.

\begin{figure}[h]
   \centering
   \includegraphics[width=0.75\textwidth]{./topics/dynamic-memory/diagrams/HeapAlloc} 
   \caption{You can ask for space, and return the space you were allocated}
   \label{fig:heap-alloc}
\end{figure}

\mynote{
\begin{itemize}
  \item \fref{fig:heap-alloc} shows the idea behind the two operations.
  \item You can ask to be allocated space, this will give you access to a space on the heap. You can then use this to store a value.
  \item You can tell the Operating System when you are finished with the space, so that it can allocate it to something else.
\end{itemize}
}


% subsubsection allocating_memory_on_the_heap (end)
\clearpage
\subsubsection{Accessing dynamically allocated memory} % (fold)
\label{ssub:accessing_dynamically_allocated_memory}

By its very nature, dynamic memory allocation must work a little differently to the way we have been working with values so far. So far, when you wanted to work with a value you declared a variable, or an array. This would have been a \nameref{sub:local_variable}, with its value allocated on the stack along with the other variables you were working with in the current function or procedure. The variable and its value were closely related, with the value being located within the variable. With dynamic memory allocation the values you are allocated are on the heap. This means that their values are not bound within a variable, but exist entirely outside of any variables that appear in your code.

One of the challenges of working with dynamically allocated memory is that you can no longer `\emph{see}' these values in your code. When you were working with variables, they were in the code, you could see them and think about the value they stored. With dynamically allocated memory you do not have this advantage, these values will be allocated as a result of the operations that are performed while the code is running. This is why it is called \textbf{dynamically} allocated memory. It is \emph{not} memory allocated to variables, it is \textbf{memory allocated upon request}.

This raises one very important question, as illustrated in \fref{fig:heap-access}:
\begin{quote}
  \emph{If the values exist outside of variables, how do you access them?}
\end{quote}
For this we require a new kind of data, a new \nameref{sub:type}. This type is used to store a value that tells you where the data you want can be located. It is like an address, telling you where the data can be found. This is the \nameref{sub:pointer} type.

\begin{figure}[h]
   \centering
   \includegraphics[width=0.7\textwidth]{./topics/dynamic-memory/diagrams/HeapAccess} 
   \caption{How can you access these dynamically allocated values?}
   \label{fig:heap-access}
\end{figure}




% subsubsection accessing_dynamically_allocated_memory (end)

% subsection heap (end)
\clearpage
\subsection{Pointer} % (fold)
\label{sub:pointer}

A Pointer is a new kind of data type, just like Integer, Double, and Boolean. A Pointer Value is an address, a location in memory where a value can be found. The name `\emph{Pointer}' is very descriptive, a \emph{Pointer} points to a value. It tells you, `The data I refer to is over there...'.

\begin{figure}[h]
   \centering
   \includegraphics[width=\textwidth]{./topics/dynamic-memory/diagrams/Pointer} 
   \caption{A Pointer Value is the address of a value, in effect it \emph{points} to a value}
   \label{fig:pointer}
\end{figure}

\mynote{
\begin{itemize}
  \item A Pointer is an \textbf{existing artefact}, a data type that is built into the Programming Language.
  \item A Pointer's Value points to a value.
  \item The CPU architecture tells you the size of its pointers. A 32bit machine has 32bit pointers. A 64bit machine has 64bit pointers.
  \item It is a good idea to picture a pointer as a value that \emph{points} to a location in memory.
  \item Pointers are used to point to locations in the Heap, when you are allocated space the Operating System will give you a Pointer Value that you can use to get to that location.
\end{itemize}
}

\clearpage
\subsubsection{What can a pointer point to?} % (fold)
\label{ssub:what_can_a_pointer_point_to_}

Pointers store a value that is an address of the value that it points to. This means that you can point to \emph{any} value in memory, regardless of where it is. You can have Pointer values that point  to \nameref{sub:local_variable}s, \nameref{sub:global_variable}s, \nameref{sub:parameter}s, \nameref{sub:array} elements, fields of \nameref{ssub:record}s or \nameref{ssub:union}s. One of its key ability, however, is the ability to point to values on the \nameref{sub:heap}.

\begin{figure}[h]
   \centering
   \includegraphics[width=\textwidth]{./topics/dynamic-memory/diagrams/PointerPointing} 
   \caption{A Pointer can point to any value, at any location in memory}
   \label{fig:pointer-pointing}
\end{figure}

\mynote{
\begin{itemize}
  \item Languages usually require you to declare the kind of data that a Pointer Value will refer to. So rather than just having a generic \emph{pointer}, you will have things like a \emph{pointer to an Integer}, or a \texttt{pointer to a User Data value}. This makes it easier to work out what you can do with the value the pointer points to.
\end{itemize}
}

% subsubsection what_can_a_pointer_point_to_ (end)

\clearpage
\subsubsection{Where can pointer values be stored?} % (fold)
\label{ssub:where_can_pointer_values_be_stored_}

A Pointer value is the same as any other value. It can be stored in \nameref{sub:local_variable}s, \nameref{sub:global_variable}s, it can be passed to a function in a \nameref{sub:parameter}, it can be returned from a \nameref{sub:function}, and it can also exist on the \nameref{sub:heap}.

\fref{fig:pointer-values} shows an illustration of some values in memory. The \texttt{start} variable is located somewhere on the stack as a local variable.  This variable is storing a pointer value that points to a \texttt{Node}\footnote{The \texttt{Node} would be a record type declared in the code. This type would contain an Integer value field named \texttt{data}, and a pointer field named \texttt{next}.} that is on the Heap. Each of the nodes on the heap are also storing pointer values that refer to other values that are also on the heap.

\begin{figure}[h]
   \centering
   \includegraphics[width=\textwidth]{./topics/dynamic-memory/diagrams/PointerValues} 
   \caption{Pointers can be stored anywhere a value can be stored}
   \label{fig:pointer-locations}
\end{figure}

\mynote{
\begin{itemize}
  \item A pointer value is no different from any other value, and can be stored on the stack, the heap, or in global variables.
  \item Languages provide a special value for pointers that do not point to a value. In C this is the \texttt{NULL} value, in Pascal it is the \texttt{nil} value, in both cases it is a value that points to nothing.
\end{itemize}
}


% subsubsection where_can_pointer_values_be_stored_ (end)

\clearpage
\subsubsection{How are pointers used?} % (fold)
\label{ssub:how_are_pointers_used_}

You need to be able to perform certain actions to make pointers useful. These include:

\begin{itemize}
  \item You must be able to get a pointer to a value. For example, you should be able to get a pointer to a value stored in a Variable.
  \item Once you have a Pointer value, you must be able to follow that pointer to its value so that you can \ldots
  \begin{itemize}
    \item Read the value it points to.
    \item Store a value at the location pointed to.
  \end{itemize}
\end{itemize}

\begin{figure}[h]
   \centering
   \includegraphics[width=0.97\textwidth]{./topics/dynamic-memory/diagrams/PointerActions} 
   \caption{You can get pointers to values, and you can follow pointers to values}
   \label{fig:pointer-actions}
\end{figure}

\mynote{
\begin{itemize}
  \item You can get the address of values in \nameref{sub:local_variable}s, \nameref{sub:global_variable}s, \nameref{sub:parameter}s, fields of \nameref{ssub:record}s and \nameref{ssub:union}s. Basically, you can get the address of any value you can read.
  \item Once you have the address (the Pointer value), you can store, or you can use it.
  \item You need to follow the pointer, called \textbf{dereferencing} the pointer, to read its value or to assign a new value to the location it refers to.
  \item Remember there are two values with pointers:
  \begin{enumerate}
    \item There is the value of the pointer itself. This is the address that is pointed to. The circle at the start of the line in the illustrations.
    \item There is the value that pointed to. The one at the end of the arrow in the illustrations.
  \end{enumerate}
  \item You can interact with both of these values, depending on whether you \emph{follow} the pointer or use the pointer's value directly.
\end{itemize}
}

% subsubsection how_are_pointers_used_ (end)

\clearpage
\subsubsection{What is the pointer value? What can you do with it?} % (fold)
\label{ssub:what_is_the_pointer_value_what_can_you_do_with_it_}

Memory is laid out as a sequence of bytes, into which values can be stored. The bytes can be thought of as being in a long line, with each being given numbered based on its position in that line. So the first byte would be byte 0, the next is byte 1, the next byte 2, and so on. This number is then the \textbf{address} of that byte. So byte 975708474 is next to byte 975708475, which is next to byte 975708476, etc. This number is also unique, so there is only one byte 975708474. It is this number that is the Pointer value, the number of the byte that it is pointing to.

\fref{fig:pointer-actions} shows an example of memory used by an array of three values. Each value is a Double, so each one occupies 8 bytes. If the first is at address 975708474, then the second starts at address 975708482 (975708474 + 8 bytes). This Figure also shows a pointer, \texttt{p}, that points to this value. That means that \texttt{p} has the value 975708474, being the address of this value, stored within it.

One feature that languages have is called \textbf{pointer arithmetic}. When you add, or subtract, a value from a pointer the compiler will work in terms of the kinds of values the pointer points to. So in \fref{fig:pointer-actions} \texttt{p} is a pointer to a Double, this means that when you add one to p you get the value 975708482, which is 1 \textbf{Double} past \texttt{p}. Therefore, \texttt{p + 2} would be 2 \emph{doubles} past \texttt{p}, at 975708490, and so on.

\begin{figure}[h]
   \centering
   \includegraphics[width=0.97\textwidth]{./topics/dynamic-memory/diagrams/PointerArithmetic} 
   \caption{The pointer value is the \emph{address} of the value it points to}
   \label{fig:pointer-actions}
\end{figure}

\mynote{
\begin{itemize}
  \item Pointer arithmetic is something you need to know exists, but not something that you would work with frequently.
\end{itemize}
}

% subsubsection what_is_the_pointer_value_what_can_you_do_with_it_ (end)

% subsection pointer (end)
\clearpage
\subsection{Allocating Memory} % (fold)
\label{sub:allocating_memory}

% subsection allocating_memory (end)
\clearpage
\subsection{Freeing Memory Allocations} % (fold)
\label{sub:freeing_memory_allocations}

% subsection freeing_memory_allocations (end)
\clearpage
\subsection{Issues with Pointers} % (fold)
\label{sub:issues_with_pointers}

\begin{quote}
\emph{With great power comes great responsibility.}
\end{quote}

Pointers give you great flexibility in your programs, allowing you to allocate your program more memory as you need it, and return that allocation when you are finished with it. Conceptually this seems very simple, but pointers are a source of many issues in programs.

\subsubsection{Access Violations} % (fold)
\label{ssub:access_violations}

The first kind of error you are likely to encounter is caused by trying to accessing memory that does not exist. This will cause your program to crash. \fref{fig:segfault} shows a common example where this occurs. Trying to follow a pointer to \emph{Nothing} will crash the program with an access violation. This applies whether you are reading or writing to the value at the end of the pointer. The common name for this kind of error is a \textbf{segmentation fault}, \emph{segfault} for short.

The only way to avoid these access violations is to \textbf{take care} with your pointers, see \fref{fig:compiler-complaint}. When you start working with pointers you need to go a little slower, and think a little more carefully about what it is you are doing. Having a good understanding of how these dynamic memory allocation tools work is the first step toward achieving this.

\begin{figure}[h]
   \centering
   \includegraphics[width=0.65\textwidth]{./topics/dynamic-memory/diagrams/AccessViolation} 
   \caption{Trying to follow a pointer that goes nowhere is a runtime error}
   \label{fig:segfault}
\end{figure}

\begin{figure}[h]
   \centering
   \includegraphics[width=\textwidth]{./topics/dynamic-memory/images/compiler_complaint} 
   \caption{To avoid access violation, take care with your pointers. From \url{http://xkcd.com/371/}}
   \label{fig:compiler-complaint}
\end{figure}

\mynote{
Here are some tips to help you avoid these access violations:
\begin{itemize}
  \item If there is any doubt, check your pointers before using them.
  \item Always initialise your pointers to \emph{Nothing}, as uninitialised pointers may point to something, but it wont be something useful.
  \item You can not see dynamically allocated memory in your code, so use a pencil and paper to sketch this out as you think through the code.
\end{itemize}
}

% subsubsection access_violations (end)
\clearpage
\subsubsection{Memory Leaks} % (fold)
\label{ssub:memory_leaks}

The next error is one that will not cause your program to crash, but will consume all of the computers memory if it is allowed to run for an extended time. Remember that with dynamic memory allocation you are responsible for releasing the memory back to the system. If you do not do this there will come a time when there is no memory left to allocate\ldots Memory leaks are hard to detect, as they do not cause your program to crash or generate any errors in its calculations. All that happens is that over time it consumes more and more memory.

Once again, the only way to avoid these issues is to \textbf{take care} with your pointers. You need to make sure that you know where the values are allocated, and where they are released. There should be reasons why you would the memory was allocated, and reasons why it is being released.

\begin{figure}[h]
   \centering
   \includegraphics[width=0.8\textwidth]{./topics/dynamic-memory/diagrams/MemoryLeak} 
   \caption{If you \emph{forget} a piece of allocated memory, it can never be freed!}
   \label{fig:memory_leak}
\end{figure}

\mynote{
Here are some tips to help you avoid memory leaks:
\begin{itemize}
  \item Have a clear idea of where memory is allocated, and where it is freed
  \item Think about the pointer values in local variables at the end of each function and procedure. Do any of these values refer to something that no other pointer does? When the function or procedure ends, the variable's value will be lost. If it is the only thing referring to some allocated memory then that memory can no longer be freed, and you have a memory leak.
\end{itemize}
}

% subsubsection memory_leaks (end)
\clearpage
\subsubsection{Accessing Released Memory} % (fold)
\label{ssub:accessing_released_memory}

The next error occurs when you are overly zealous about releasing memory. You must not release memory before you are finished with it. The problem occurs when you continue to access a value, after its memory has been released. This is one of the most difficult problems to locate, as it will not cause any problems initially. 

Take \fref{fig:read_unallocated} as an example. This demonstrates a case where two pointers refer to one value. It is possible to free that value via one pointer, and then forget that the second refers to the same location. When you read the value from \texttt{p2} later, it is \emph{likely} to still be \texttt{10}, so the program will continue to run as normal. The issue will only appear later when something else is allocated to use that piece of memory. All of a sudden the value you thought was allocated to \texttt{p2} is now changing apparently on its own. Worst of all, the actual cause of the bug could be hundred of lines of code away from where the problem appears. This is what makes this kind of error very difficult to find.

The solution, once again, is to \textbf{take care} with pointers.

\begin{figure}[h]
   \centering
   \includegraphics[width=0.75\textwidth]{./topics/dynamic-memory/diagrams/ReadUnallocated} 
   \caption{You can still read values from memory even when they are unallocated...}
   \label{fig:read_unallocated}
\end{figure}

\mynote{
Here are some tips to help you avoid accessing released memory:
\begin{itemize}
  \item When you free memory, spend some time thinking about the things that could be referring to the value you just released.
\end{itemize}
}


% subsubsection accessing_released_memory (end)



% subsection issues_with_pointers (end)
\clearpage
\subsection{Linked List} % (fold)
\label{sub:linked_list}

Pointers and dynamic memory allocation make it possible to store values in new and interesting ways. One way of structuring this data is to dynamically allocate each value, and link these together using pointers. An illustration of this is shown in \fref{fig:linked-list}.

Linked lists have the advantage of being very fast to perform insert and delete actions, when compared with arrays. The disadvantage is an increase in storage size to keep all the pointers, and the fact you must loop through the nodes to access any value in the list. 

\begin{figure}[htbp]
   \centering
   \includegraphics[width=0.90\textwidth]{./topics/dynamic-memory/diagrams/LinkedList} 
   \caption{Illustration of a linked list in memory}
   \label{fig:linked-list}
\end{figure}

\pseudocode{lst:iterate-linked-list}{Pseudocode for looping through a linked list}{topics/dynamic-memory/concepts/list-iterate.txt}

\mynote{
\begin{itemize}
  \item A Linked List is a \textbf{term} given to a certain way data can be structured in memory.
  \item A Linked List has \textbf{Nodes}, the equivalent of the elements of an array.
  \item Each \textbf{Node}, has some data and a pointer to the \textbf{Next} element in the list.
  \item The \textbf{Last} element in the list has \textbf{Nothing} as its next node.
  \item To access a Node in the list you must loop through from the first node until you reach the node you are after.
  \item You can insert and delete elements by changing the links in the list.
  \item If the grey node in \fref{fig:linked-list} is being \emph{inserted} then the previous node must be adjusted to point to it, and it to point to the next element of the list.
  \item If the grey node in \fref{fig:linked-list} is being \emph{deleted} then the previous node changes its link to skip that node and point to the next node in the list.
  \item The pseudocode in \lref{lst:iterate-linked-list} shows the standard way of applying an action to each node of a Linked List.
\end{itemize}
}

% subsection linked_list (end)
\clearpage
\subsection{Summary of Dynamic Memory Allocation Concepts} % (fold)
\label{sub:dymanic_memory_summary}

This chapter has introduced a number of concepts related to working with pointers and performing dynamic memory allocation.

\begin{figure}[htbp]
   \centering
   \includegraphics[width=0.9\textwidth]{./topics/dynamic-memory/diagrams/Summary} 
   \caption{Memory management focuses on allocating memory, releasing this allocation, pointers, and the heap}
   \label{fig:dynamic-memory-summary}
\end{figure}


\mynote{
\begin{itemize}
  \item \textbf{Heap} - an area of memory you can be allocated to store values.
  \item \textbf{Allocate Memory} - gives you ownership of a piece of the heap's memory.
  \item \textbf{Release Memory} - once you own the memory it is yours until it is released. If you forget to release it, it cannot be used by others.
  \item \textbf{Pointers} - are values that point to locations in memory. They store the address of the area of memory they refer to, and are needed to give you access to the heap.
\end{itemize}
}

% subsection summary (end)

% section dynamic_memory_allocation_concepts (end)

% ===================================
% = Using Dynamic Memory Allocation =
% ===================================
\clearpage
\section{Using Dynamic Memory Allocation} % (fold)
\label{sec:using_dynamic_memory_allocation}

Dynamic memory allocation makes it possible for you to allocate additional space for your program to use from the \nameref{sub:heap}.

\subsection{Designing Linked List} % (fold)
\label{sub:designing_linked_list}

% subsection designing_linked_list (end)

\subsection{Designing Small DB 2} % (fold)
\label{sub:designing_small_db_2}

% subsection designing_small_db_2 (end)

% section using_dynamic_memory_allocation (end)

% ==================================
% = Dynamic Memory Allocation in C =
% ==================================
\clearpage
\section{Dynamic Memory Allocation in C} % (fold)
\label{sec:dynamic_memory_allocation_in_c}

% section dynamic_memory_allocation_in_c (end)

% ===========================================
% = Understanding Dynamic Memory Allocation =
% ===========================================
\clearpage
\section{Understanding Dynamic Memory Allocation} % (fold)
\label{sec:understanding_dynamic_memory_allocation}

% section understanding_dynamic_memory_allocation (end)

% =========================================
% = Examples of Dynamic Memory Allocation =
% =========================================
\clearpage
\section{Examples of Dynamic Memory Allocation} % (fold)
\label{sec:examples_of_dynamic_memory_allocation}

% section examples_of_dynamic_memory_allocation (end)

% ===========================================
% = Exercises for Dynamic Memory Allocation =
% ===========================================
\clearpage
\section{Exercises for Dynamic Memory Allocation} % (fold)
\label{sec:exercises_for_dynamic_memory_allocation}

% section exercises_for_dynamic_memory_allocation (end)

% ============================================
% = Dynamic Memory Allocation in the Project =
% ============================================
\clearpage
\section{Dynamic Memory Allocation in the Project} % (fold)
\label{sec:dynamic_memory_allocation_in_the_project}



% section dynamic_memory_allocation_in_the_project (end)


% chapter dynamic_memory_allocation (end)