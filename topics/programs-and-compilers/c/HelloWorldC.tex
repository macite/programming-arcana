\subsection{Hello World in C++} % (fold)
\label{sub:hello_world_in_c}

Now that you have the compiler installed you can create your first program: the famous \textbf{Hello World} discussed in \sref{sub:hello world}. The C++ code for this is shown in \lref{lst:hello-world-c-c}. This code tells the computer to `write' the text \emph{Hello World!} to the Terminal, and follow it with a new line. Do the following to create this program for yourself, see the notes below for hints:

\begin{enumerate}
  \item Open your text editor
  \item Create a new text file
  \item Type\footnote{Do not just copy and paste it out of the text, type it in yourself as this will help you learn the concepts being covered.} in the text below, making sure you get every character correct.
  \item Save your program's code in a file called \textbf{hello-world.cpp}, and note the directory where it is saved
  \item Open the Terminal
  \item Change into the directory where you save the file
  \item Compile the program using \texttt{skm clang++ hello-world.cpp -o HelloWorld}
  \item Run the program using \texttt{./HelloWorld}
\end{enumerate}

Well done, you have now created and run your first C++ program!

\csection
{
\ccode{lst:hello-world-c-c}{Hello World code in C++.}{code/c/program-creation/hello-world.c}
}

\mynote{
\begin{itemize}
  \item Check out the SplashKit installation guide for how to setup the text editor on your machine.
  \item See \nameref{ssub:bash} in \sref{sub:terminal} for an example of how to use the Terminal.
  \item See \sref{sec:using_these_concepts_compiling_a_program} \nameref{sec:using_these_concepts_compiling_a_program} for the overall process and the output you should expect from the program.
\end{itemize}
}

% subsection hello_world_in_c (end)