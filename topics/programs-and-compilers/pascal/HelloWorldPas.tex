\clearpage
\subsection{Hello World in Pascal} % (fold)
\label{sub:hello_world_in_pascal}

Now that you have the compiler installed you can create your first program: the famous \textbf{Hello World} discussed in \sref{sub:hello world}. The Pascal code for this is shown in \lref{lst:hello-world-pas-pas}. This code tells the computer to `write' the text \emph{Hello World!} to the Terminal. Do the following to create this program for yourself, see the notes below for hints:

\begin{enumerate}
  \item Open your text editor
  \item Create a new text file
  \item Type\footnote{Do not just copy and paste it out of the text, type it in yourself as this will help you learn the concepts being covered.} in the text below, making sure you get every character correct.
  \item Save your program's code in a file called \textbf{HelloWorld.pas}, and note the directory where it is saved
  \item Open the Terminal
  \item Change into the directory where you save the file
  \item Compile the program using \texttt{fpc -S2 HelloWorld.pas}
  \item Run the program using \texttt{./HelloWorld}
\end{enumerate}

Well done, you have now created and run your first Pascal program!

\passection
{
\pascode{lst:hello-world-pas-pas}{Hello World code in Pascal.}{code/pascal/program-creation/HelloWorld.pas}
}

\mynote{
\begin{itemize}
  \item See \sref{sec:installing_a_text_editor} \nameref{sec:installing_a_text_editor} for details on installing the Text Editor.
  \item See \nameref{ssub:bash} in \sref{sub:terminal} for an example of how to use the Terminal.
  \item See \sref{sec:using_these_concepts_compiling_a_program} \nameref{sec:using_these_concepts_compiling_a_program} for the overall process and the output you should expect from the program.
\end{itemize}
}

% subsection hello_world_in_c (end)