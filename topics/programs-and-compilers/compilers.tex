\chapter{Building Programs} % (fold)
\label{cha:building programs}

\begin{quote}
  %\frakfamily\selectfont\Large
  \Fontlukas\Large
  \renewcommand{\LettrineTextFont}{\relax}
  \renewcommand{\LettrineFontHook}{\color{red}}
  \lettrine[image=true,lines=3,lhang=.2, loversize=.25, findent=0.1em]
  {W}{elcome} neophyte! So you seek to master the art of Magic? To work magic you must learn the structure of incantations, and how to use a wand to convert these into magic energy. 
  
  Let us begin with a simple spell, a spell that will greet the world around you! Utter these words, and wave your wand\ldots
\end{quote}

\bigskip

Welcome to the Programming Arcana\footnote{Arcana is defined as secrets or mysteries, Wiktionary defines it as ``specialized knowledge that is mysterious to the uninitiated.''. This fits well with the idea of programming, and we think its a cool word to describe the \emph{magic} of programming!}, a book about learning to program. This book contains a number of chapters that take you from knowing nothing, or little, about programming to a position where the mysteries are revealed. By the end of the material you will be able to create your own programs and you will be ready to start learning other programming languages and approaches to software development.

This book is divided into a number of chapters, each of which introduces you to a programming task and the arcane knowledge that must be attained to understand how the task is accomplished. As with any arcane knowledge there are special terms that are used by those who know its secrets. In each chapter you will be introduced to the terms you need to understand in order to perform the current task. This will provide you with the tools you need to describe programs to other software developers, and will help you understand how the structures within your programs work to achieve their goals.

Like magic, you must learn the structure of source code, and how to use tools to convert these into programs. Let us begin with a simple program, a program that will greet the world.

\minitoc

% =====================================
% = Concepts - Compilers and Programs =
% =====================================
\clearpage
\section{Concepts Related to Building Programs} % (fold)
\label{sec:concepts_related_to_building_programs}

In this chapter you will learn about the basic tools you need to use to create your own programs. You will see an example program, and then use these tools to convert that code into an executable program. You will then be able to run the program you created, and see it perform the tasks you coded.

This chapter provides a background on what programs are, and the general processes of how they are created. This will introduce you to the tools you will be using, and fill in some details of what they are doing and how to use them. The topics covered include the following:
\begin{itemize}
  \item \nameref{sub:what_is_a_program_}: This section introduces you to the idea of what a program is, and what it contains. You will need to be familiar with programs and what they are, as you will need to be able to create your own.
  \item \nameref{sub:machine_code}: Talks about the kinds of instructions the computer understands, and why it is not very productive to work at this level. You need to understand that this exists behind the scenes, but do not need to be overly familiar with it.
  \item \nameref{sub:assembly}: The next level of language is called Assembly. It is very close to machine code, but much easier to understand and use. However, this is still too low a level to be very productive. Just like machine code, you only need to know this exists
  \item \nameref{sub:source_code_and_the_compiler}: This is the level we are going to be working with in this book. This code is much easier to work with than assembly, and allows you to create your own programs reasonably quickly once you have learnt the basics. These are the tools you are going to be working with throughout this text.
  \item \nameref{sub:terminal}: This is a command line environment that lets you issue text commands to the user. You will use this to create and run your programs.
  \item \nameref{sub:hello world}: A \emph{classic} program used to check that you have everything working correctly. This section shows you the code that you can use this to check that you have all the tools setup correctly, and to check their usage.
  \item \nameref{sub:compiling_code}: This final section will show you how to use these tools to compile and run your own Hello World program. \fref{fig:run-1-helloworld} shows the Hello World program running from the Terminal.
\end{itemize}

Before getting into these details the next section provides you will an outline of the book, and how you can go about reading it.

\begin{figure}[b]
   \centering
   \includegraphics[width=0.7\textwidth]{./topics/programs-and-compilers/images/HelloWorld} 
   \caption{Hello World run from the Terminal}
   \label{fig:run-1-helloworld}
\end{figure}


\clearpage
\subsection{Book Overview} % (fold)
\label{sub:book_overview}

This book focuses on programming concepts, and gives you the option of programming these using either the C or Pascal programming language. Both languages are very capable, with each having their own advantages and disadvantages. Pascal was designed as a teaching language and is easy to program with while still being very capable. C is very flexible and is the basis for a number of other languages.

The layout of each chapter is the same, and has the following format:
\begin{enumerate}
  \item \textbf{Concepts}: The first part of the chapter introduces the concepts that will be covered. This is done in a language neutral manner, with the focus being on how to think about the tools being presented. This will introduce each concept with an illustration, and accompany this with explanatory text.
  \item \textbf{Using the Concepts}: The next section shows how these concepts can be used to achieve a task. This task will try to cover all the concepts presented in a practical manner. This is done in a language neutral way, and talks about how to use the concepts to achieve a goal.
  \item \textbf{Languages}: The next two sections present the syntax you need to use these concepts in \textbf{C} and \textbf{Pascal}. You should use this as a reference, and can read this alongside reading about how to use the concepts.
  \item \textbf{Understanding}: Following the language specific details, the next section explains in detail how the concepts work within the computer. Use this to get an understanding of how the concepts work in more detail. This section will show you illustrations of what is happening within the computer when your code is running.
  \item \textbf{Examples}: Each chapter will have at least one example showing you how these concepts can be used. This will include the code, and some explanatory text to discuss what is being presented.
  \item \textbf{Exercises}: The exercises allow you to put into practice what you have read about. You cannot learn to program without practice. These exercises are a good start, but you should try to come up with your own project so that you can test out these new concepts on something you are interested in working on.
\end{enumerate}



\clearpage
\subsubsection{Formatting} % (fold)
\label{ssub:formatting}

This book has a number of visual formatting guides. These are designed to help you navigate through the material easily.

\pseudosection{
Text formatted in this way relates to an algorithm description. This will describe the steps that need to be performed in a way that is language neutral and can be applied to C, Pascal, and possibly other languages.
}

\csection{
Text formatted in this way relates to the C programming language. If you are going to use C you need to pay attention to the text in these boxes, otherwise you can skip over them.
}

\passection{
Text formatted in this way relates to the Pascal programming language. If you are going to use Pascal you need to pay attention to the text in these boxes, otherwise you can skip over them.

}

\mynote{
Text formatted in this way covers notes related to the current concept or illustration. This book makes extensive use of notes to capture important points, so do not skip over these.
}

The language sections of each chapter also add markers to each page to clearly mark where they start, and where they end. If this is your first programming experience you should stick with one of these languages, so you can skip the pages that are marked as being for the other language.


% subsubsection formatting (end)


% subsection book_overview (end)

\clearpage
\subsection{Programming Jargon and Concept Taxonomy} % (fold)
\label{sub:concept_taxonomy}

Programming has a lot of its own jargon. As you learn to develop software it is also important that you start to learn this \emph{special language} that software developers use to discuss their programs. You will find that this terminology is used in many places. It is used in programming texts, in discussions between developers, in discussion boards, blogs, anywhere that developers are discussing software development. Having a clear understanding of this terminology will help you make the most of these resources.

The concepts in this book are closely linked to this programming terminology. To help you understand each concept, we have classified them using one of the following categories:

\begin{itemize}
  \item \textbf{Artefact}: An artefact is something that you can create in your code.
  \item \textbf{Action}: Actions are things that you can \emph{command} the computer to do.
  \item \textbf{Term}: These are general terms, used to describe some aspect.
\end{itemize}

When you are reading about the different concepts in this book you can use these classifications to help you think about how you may use the knowledge you are gaining.

\subsubsection{Artefacts} % (fold)
\label{ssub:artefacts}

Artefacts are things that you create in your code. Programming is a very \emph{abstract} activity, you spend most of your time working with concepts and ideas. You write text, code, that will create things within the computer when your code is run. 

When you are learning about a new kind of artefact come up with ways of visualising it. It is a \textbf{thing} that you are creating with your code. Try to picture the artefact within your code. These artefacts are the basic building blocks that you have to work with. You need to be very familiar with them, how they work, and what you can do with them.

% subsubsection artefacts (end)

\subsubsection{Actions} % (fold)
\label{ssub:actions}

Actions get the computer to perform a task. Your actions will be coded within the \textbf{artefacts} that you create, and will define how artefacts behave when they are used. The actions themselves are commands that you issue to the computer. They are executed one at a time, and each kind of action gets the computer to carry out certain tasks.

When you are learning a new kind of action you need to see what this action does. To start with you should play with it, test it out, and see if you can understand what it is getting the computer to do. As you progress you need to start thinking about how you can sequence these actions so that the computer performs the tasks you want it to. There are only a very few kinds of actions, so it is by combining them that you can get the computer to do what you want. 

% subsubsection actions (end)

\subsubsection{Terms} % (fold)
\label{ssub:terms}

The remaining terms are words that developers use to explain concepts. These are not things that you create, or actions that you request. These are just words that you need to \emph{know}.

When you are learning a new term you need to try to commit it to memory. Memorise the terms, try to use them in sentences, explain them to others. All of these tasks will help you understand, and remember these terms.

% subsubsection terms (end)

% subsection concept_taxonomy (end)
\clearpage
\subsection{Programs} % (fold)
\label{sub:what_is_a_program_}

If you are going to learn to develop software you will need to become intimately aware of what a program is. After all, as a developer you will be creating your own programs.

A program is a file that contains instructions that get the computer to perform a task. Programs are lists of commands\footnote{C and Pascal are both \emph{imperative} programming languages. In the imperative paradigm a program is seen as a list of commands instructing the computer to perform actions.} telling the computer what to do, and the order in which to do it. Each instruction is very simple, but they can be executed very quickly, allowing computers to perform quite remarkable feats.

\begin{figure}[h]
   \centering
   \includegraphics[width=0.9\textwidth]{./topics/programs-and-compilers/diagrams/Program} 
   \caption{A program contains instructions that command the computer to perform a task}
   \label{fig:what-is-a-program}
\end{figure}

\mynote{
\begin{itemize}
  \item You can \textbf{run} programs, which gets the computer to follow the instructions found within the program's file.
  \item To run, the program's instructions must first be loaded into memory.
  \item Once in memory, the computer starts running the instructions one after the other.
  \item When the last instruction is completed the program ends.
  \item There are several different ways to run a program:
  \begin{itemize}
    \item You can \emph{double-click} the program in a file browser.
    \item On tablets and app-phones you can \emph{tap} the program's icon.
    \item Advanced uses can enter \emph{text commands} in the Terminal to start programs.
  \end{itemize}
\end{itemize}
}

\clearpage
\subsubsection{What happens when a program runs?} % (fold)
\label{ssub:what_happens_when_a_program_runs_}

When you run a program, regardless of how it is started, the Operating System loads it from disk into memory and then starts it running. It is important that the file you try to run is a program. These are \emph{special} files that contain instructions the computer can understand.

\begin{figure}[h]
   \centering
   \includegraphics[width=0.9\textwidth]{./topics/programs-and-compilers/diagrams/ProgramExe} 
   \caption{Programs are loaded from disk into memory, then run}
   \label{fig:what-is-a-program-exe}
\end{figure}

\mynote{
\begin{itemize}
  \item Your Operating System is a piece of software that is responsible for managing your computer's hardware.
  \item One of the Operating System's responsible is to start programs.
  \item The \nameref{sub:terminal} can be used to start programs.
  \item Programs can also output to the Terminal.
  \item A Program must contain instructions that the computer can understand.
\end{itemize}
}

% subsubsection what_happens_when_a_program_runs_ (end)

% subsection what_is_a_program_ (end)
\clearpage
\subsection{Machine Code} % (fold)
\label{sub:machine_code}

\emph{What instructions do Computers understand?}

Computers do not really \emph{understand} anything, computers are \textbf{unintelligent}. They are a machine that respond in a set way to a given number of instructions. The instructions that a computer uses is called its {\em instruction set} and contain instructions to perform basic mathematic operations, loading and storing data in memory, comparing numeric values, and moving to the new instruction elsewhere in the program. These very simple actions are performed very quickly, and can be use to create everything you have ever seen a computer do.

The computers instructions can be seen as binary numbers, numbers made from 0's and 1's. These values are like switches that are either off (0) or on (1). Setting these \emph{switches} to different sequences will cause the computer to perform different actions. For example, the \emph{switch} combination \texttt{0000 0011}, may cause the computer to add two numbers together. Any time you want the computer to perform this task you set the switches to that combination. These binary instructions are called \textbf{machine code}.

\begin{figure}[h]
   \centering
   \includegraphics[width=0.8\textwidth]{./topics/programs-and-compilers/diagrams/MachineCode} 
   \caption{The computer responds to machine code instructions}
   \label{fig:machine-code}
\end{figure}

\mynote{
\begin{itemize}
  \item The \textbf{CPU}, Central Processing Unit, is the workhorse of the computer. It executes the program's instructions.
  \item The instructions a CPU uses is called its \textbf{Instruction Set}, and different CPUs have different instruction sets.
  \item Some common instruction sets: ARM (used in most mobiles and tablets), x86-64 (used in desktops and laptops). 
\end{itemize}
}

\clearpage

\subsubsection{Programming in Machine Code} % (fold)
\label{ssub:programming_in_machine_code}

Listing \ref{lst:machine code} shows a chunk of the machine code for a small program. These 1s and 0s are the codes used to instruct the computer when this program is executed. Programs can be written directly in machine code, but this is a time consuming task. This is further complicated by the fact that machine code is unique to each kind of CPU. This means that programming at this level is entirely dependent on the kind of processor that you are targeting.

\begin{lstlisting}[caption={128 bits from the 106,752 bits of Machine Code from a small program.},label={lst:machine code}]
...
0110 0111 0111 0010 0000 0000 0110 0011 0100 1110 0101 1111 0100 0001 0101 1000
0110 0111 0111 0010 0000 0000 0111 0110 0101 1111 0101 1111 0110 1001 0101 1111
...
\end{lstlisting}

No one wants to have to work at this level of details, and fortunately you do not need to. Software developers have created tools to help them create programs without having to think about these low level details. These tools make it possible to work at a \textbf{higher level of abstraction}. They take the code you write, and do the hard work of converting that to the machine code of the computer you want to run it on.

\mynote{
\begin{itemize}
  \item You can look at the machine code of any program on your computer. You just need the right tools.
  \item If you open the program's executable file in a text editor it will look very strange, and not at all like a large list of binary values. This is because the text editor displays one character for every byte (or two bytes depending on the file) from the file.
  \item A \textbf{Hex Editor} is a program that is useful for examining binary data. It shows you one character for every four bits in the file.
\end{itemize}
}

\begin{table}[h]
  \ttfamily
  \centering
\begin{tabular}{|c|c||c|c||c|c||c|c|}
  \hline
  Binary & Hex & Binary & Hex & Binary & Hex & Binary & Hex  \\
  \hline
  0000 & \textbf{0} & 0001  & \textbf{1}  & 0010  &  \textbf{2} & 0011 & \textbf{3} \\
  0100 & \textbf{4} & 0101 & \textbf{5} & 0110 & \textbf{6} & 0111 & \textbf{7} \\
  1000 & \textbf{8} & 1001 & \textbf{9} & 1010 & \textbf{A} & 1011 & \textbf{B} \\
  1100 & \textbf{C} & 1101 & \textbf{D} & 1110 & \textbf{E} & 1111 & \textbf{F} \\
  \hline
\end{tabular}
  \caption{Binary to Hexadecimal}
\end{table}

\begin{figure}[h]
   \centering
   \includegraphics[width=0.55\textwidth]{./topics/programs-and-compilers/images/HexEditor} 
   \caption{A HexEditor allows you to view the machine code of any program}
   \label{fig:hex-editor}
\end{figure}


% subsubsection programming_in_machine_code (end)

% subsection machine_code (end)
\clearpage
\subsection{Assembly} % (fold)
\label{sub:assembly}

The next level of abstraction up from machine code is called \textbf{Assembly}, or \textbf{Assembler Code}. Here the numeric machine code instructions are given symbolic names that are, to some degree, more understandable for humans. The code \texttt{0000 0011} may be given the symbolic name \texttt{add}, for example.

Programs written in this language cannot be executed directly by the computer, it isn't machine code. Assembler code is converted to machine code by a program called an \textbf{Assembler}. This program reads the instructions from the assembler code and outputs machine code. So, for example, anywhere it encounters \texttt{add} in the code it can output \texttt{0000 0011}.

\begin{figure}[h]
   \centering
   \includegraphics[width=0.8\textwidth]{./topics/programs-and-compilers/diagrams/Assembly} 
   \caption{The computer responds to machine code instructions}
   \label{fig:assembly}
\end{figure}

\mynote{
\begin{itemize}
  \item As with \nameref{sub:machine_code}, Assembly is liked to individual CPUs.
  \item Assembly is very close to Machine Code, its machine code with symbolic names for the instructions.
\end{itemize}
}

\clearpage
\subsubsection{Programming in Assembly} % (fold)
\label{ssub:programming_in_assembly}

The code in Listing \vref{asmcode} shows an example of some assembler code. This is the assembler code that was used to generate the machine code from Listing \ref{lst:machine code}. The machine code was 13,344 bytes in size, where the same program in assembler code is only 658 bytes. The assembler reads these 658 bytes, combines it with instructions from program libraries, and outputs machine code. 
\lstset{language=[x86masm]{assembler}}

\begin{lstlisting}[caption={Assembler Sample},label={asmcode}]
  .cstring
LC0:
  .ascii "Hello\0"
  .text
.globl _main
_main:
  pushl	%ebp
  movl	%esp, %ebp
  pushl	%ebx
  subl	$20, %esp
  call	___i686.get_pc_thunk.bx 
"L000001$pb":
  leal	LC0-"L000001$pb"(%ebx), %eax
  movl	%eax, (%esp)
  call	L_printf$stub
  addl	$20, %esp
  popl	%ebx
  popl	%ebp
  ret
\end{lstlisting}

From a programmer's perspective, assembler code is much easier to work with than machine code, though there are still issues with the use of assembler code. Firstly Assembly is bound to the instruction set of the CPU that you are targeting, meaning that if you want to support other kinds of CPU you will need to rewrite the program. The other main issue with assembler code is that while it is more understandable, you are still working with the primitive instructions of the CPU. Working at this level takes considerable effort to write even simple programs.

Assembly languages were first developed in the 1950s, and were known as a \textbf{Second Generation}\footnote{First Generation being Machine Code.} programming languages. This step forward did make programming easier, but the tools have advanced since then and now we can work at an even higher level of abstraction.

% subsubsection programming_in_assembly (end)

% subsection assembly (end)
\clearpage
\subsection{Source Code and the Compiler} % (fold)
\label{sub:source_code_and_the_compiler}

The next step in programming language evolution moved from machine level instructions to something more human readable. These languages, known as \textbf{Third Generation Languages}, use move advanced programs than assemblers to convert their instructions into machine code. Programs written in these languages have their code converted to machine code by a \textbf{compiler}.

A \textbf{Compiler} is a program that converts \textbf{Source Code} into machine code that is saved into an executable file called a \emph{Program}. The program can then be executed independent of the compiler and the source code.

Internally, a compiler will perform a number of steps, as shown in \fref{fig:compiler}.

\begin{enumerate}
  \item \textbf{Preprocessing}: The code is read from your source code files. This may involve some processing of the text itself, which includes things like ignoring any comments in the code.
  \item \textbf{Compiling}: The code is then converted into assembly instructions, and an assembly program is output.
  \item \textbf{Assembling}: The assembly version of the program is converted into machine code, and stored in \textbf{object files}.
  \item \textbf{Linking}: In the final step the compiler uses a \textbf{Linker} to join together the machine code from your program, with other machine code you have used from the programming libraries. This then outputs an executable program.
\end{enumerate}

\begin{figure}[h]
   \centering
   \includegraphics[width=0.7\textwidth]{./topics/programs-and-compilers/diagrams/Compiler} 
   \caption{Compilers turn Source Code into Machine Code}
   \label{fig:compiler}
\end{figure}

\clearpage
\subsubsection{Programming with a Third Generation Language} % (fold)
\label{ssub:programming_with_a_third_generation_language}

\lref{lst:hello-world-c-1} and \lref{lst:hello-world-pas-1} show two examples of source code. This code describes a small program that can be used to output a message to the \nameref{sub:terminal}.

\begin{multicols}{2}
  \ccode{lst:hello-world-c-1}{Example C++ code}{code/c/program-creation/hello-world.c}
  \columnbreak
  \pascode{lst:hello-world-pas-1}{Example Pascal code}{./topics/program-creation/pascal/HelloWorld.pas}
\end{multicols}

The code shown in \lref{lst:hello-world-c-1} shows the code for the C++ program that was used to generate the assembler code, and machine code shown in the previous code listings. This code must be converted by the C++ compiler into machine code before it can be run. It is interesting to note the size of the C++ file: it is only 50 bytes! The compiler converts this 50 bytes into the 13,344 bytes of machine code. 

\lref{lst:hello-world-pas-1} shows the same program written in the Pascal programming language. Like its equivalent C++ code, this must be compiled to create a program you can run.

Programs written in a third generation programming language are much easier to understand than their assembler or machine code counterparts. It is also possible that this code can be compiled to run on different types of CPU, making it more portable. Most modern programming languages are third generation programming languages.


The code that a programmer writes in these languages is called \textbf{Source Code}. Typically source code is saved into a text file with a file extension that helps identify the language it is written in. For example, programs written in the C++ language are saved into files with a {\tt .c} file extension whereas Pascal programs are saved into files with a {\tt .pas} extension.

\mynote{
\begin{itemize}
  \item There are may different Third Generation Languages, including both C++ and Pascal.
  \item Each language has its own compiler that understands that language's code.
  \item The C++ compiler we will use is \textbf{clang++} (or \textbf{g++} - the \textbf{GNU C++ Compiler}).
  \item The Pascal compiler we will use is called \textbf{fpc} - this stands for \textbf{Free Pascal Compiler}.
\end{itemize}
}



% subsubsection programming_with_a_third_generation_language (end)

% subsection source_code_and_the_compiler (end)
% \clearpage
\subsection{Challenges and Rewards} % (fold)
\label{sub:challenges_and_rewards}

% subsection challenges_and_rewards (end)

Programming in a third generation language, like C++ and Pascal, requires you to master several different things, as shown in the following list. The following chapters will work on building your knowledge and skills in each of these aspects.

\begin{enumerate}
  \item What is it that you want the program to do? Writing a program is like writing instructions for someone to carry out. If you do not know how to perform the task yourself you will not be able to tell someone else how to perform the task.
  \item You need to understand what the computer is capable of doing. Computers are unintelligent, so writing a program is more challenging than giving instructions to a person as the computer cannot interpret what you mean and will follow your instructions to the letter, regardless of the effect. The capabilities of the computer limit the flexibility you have for expressing your solution.
  \item The language you choose to develop with also limits how you express your solution. You need to understand the artefacts that you can create, how these artefacts are written in source code, and how these are executed by the computer.
  \item Finally you need to understand how to locate and correct issues with your programs. This includes responding to syntax errors reported to you by the compiler, as well being able to locate errors where the program does not operate the way you intended. 
\end{enumerate}

\emph{With all of these challenges, what appeal does software development have?}

There is nothing better than seeing a program you created running on a computer. You have brought the machine to life, getting it to perform a task the way you want it performed. Once you get a program working it can become easy to get hooked and working on new features and functions becomes a real joy. The greater the challenge the program offers, the greater your sense of achievement when you see the working product in operation.

% subsection compiling_code (end)
\clearpage
\subsection{Terminal} % (fold)
\label{sub:terminal}

Once you have written some source code, you need to be able to compile it. This means, you need to run the \textbf{compiler}, and give it your source code files to compile. The best way to do this when you really want to learn about programming, is to run the compiler directly yourself. To do this you needed to use a \textbf{Terminal} program.

The Terminal is a program that gives you command line access to the computer. With command line access you can enter text commands to start programs. These programs can output details back to the Terminal for you to read, and interactive programs can also read input from you via this same Terminal.

\begin{figure}[h]
   \centering
   \includegraphics[width=0.8\textwidth]{./topics/programs-and-compilers/diagrams/Terminal} 
   \caption{The Terminal program gives you command line access to your computer}
   \label{fig:terminal}
\end{figure}

\mynote{
\begin{itemize}
  \item On Ubuntu \textbf{Linux} you can find the Terminal in the \emph{Accessories} folder within \emph{Applications}. See Figure \ref{fig:program-creation-ubuntu-terminal}.
  \item On \textbf{MacOS} you can find the Terminal in the \emph{Utilities} folder within \emph{Applications}. See Figure \ref{fig:program-creation-macos-terminal}.
  \item On \textbf{Windows} you will need to download and install \emph{MinGW}, making sure to select the \emph{MinSYS} option during the install process. The \emph{MinGW Shell} is then the equivalent of Terminal on the other operating systems. You will find this in \emph{Program Files}, \emph{MinGW}. See Figure \ref{fig:program-creation-mingw-shell}.

\end{itemize}
}

\clearpage
\subsubsection{The Shell} % (fold)
\label{ssub:the_shell}

The \textbf{Terminal} program itself just provides a text environment, allowing text input and output. Within this environment a \textbf{Shell} program is run to interpret your commands. This is an interactive program that will display a prompt to you, at which you enter your commands.

There are a number of different Shell programs, each of which has its own set of instructions. The Shell we are going to use in this book is called \textbf{Bash}. This shell program is available on Linux, Mac OS, and Windows. As a Unix shell it is native for Linux and Mac OS, and with Windows you can install \textbf{MinGW} to use these commands.

\begin{figure}[h]
   \centering
   \includegraphics[width=0.6\textwidth]{./topics/programs-and-compilers/images/Bash} 
   \caption{The Terminal running Bash}
   \label{fig:bash}
\end{figure}

A shell program is very simple. It provides a text prompt at which you can enter commands. The Shell then reads the text you entered, and performs an action based on the text you entered. You can use the Shell to perform operations like copying and deleting files, and starting programs.

\mynote{
\begin{itemize}
  \item The name `Shell' came from idea that this was the outermost \emph{shell} of the computer, the interface between the user and the computer's internals.
  \item Bash stands for `\emph{Bourne-again shell}', as Bash is a replacement for the \emph{Bourne} shell.
  \item It will take some time to get used to using Bash, but the more you learn about it the more useful it will become.
\end{itemize}
}

% subsubsection the_shell (end)
\clearpage
\subsubsection{Using Bash} % (fold)
\label{ssub:bash}

To get started using Terminal you will need to know some Bash commands, see \tref{tbl:bash-commands}.

\begin{table}[h]
  \centering
  \begin{tabular}{|l|l|p{5cm}|}
    \hline
    \textbf{Action} & \textbf{Command} & \textbf{Description} \\
    \hline
    Change Directory & \texttt{cd} & Moves the shell to a different working directory. \\
    \hline
    Print Working Directory & \texttt{pwd} & Outputs the current working directory.\\
    \hline
    List Files & \texttt{ls} & Outputs a list of files.\\
    \hline
    Copy File(s) & \texttt{cp} & Copies files from one location to another.\\
    \hline
    Move File(s) & \texttt{mv} & Moves files from one location to another.\\
    \hline
    Delete File(s) & \texttt{rm} & Removes files from the computer. There is no recycle bin with this, so take care!\\
    \hline
    Create a Directory & \texttt{mkdir} & Makes a new directory. \\
    \hline
  \end{tabular}
  \caption{Some bash commands to get you started}
  \label{tbl:bash-commands}
\end{table}

To get started with Bash, you need to understand a little bit about the \textbf{file system}. Each operating system needs a way of storing its files, and there are going to be lots of files stored on a computer. This means that it would be cumbersome to try and keep these all in one place. Instead, the Operating System places files in \textbf{directories} (also known as \emph{Folders}). A directory can contain files, and other directories. 

When you are working in Bash, you will have a \textbf{working directory}. This is the directory where Bash will start searching for the files you are interacting with. To start working with the compiler you will need to be able to use the \textbf{change directory} command to move to the directory that contains your source code files.

With the \textbf{Change Directory} command you tell Bash which directory you want to move into, with the different parts of this path being separated by forward slashes (/). Example commands to move to your Documents directory are shown in \tref{tbl:dirs}, with screenshots for Linux in \fref{fig:linux-files}, Mac OS in \fref{fig:mac-files}, and Windows in \fref{fig:win-files}.

\begin{table}[h]
  \centering
  \begin{tabular}{|l|l|}
  \hline
  \textbf{Operating System} & \textbf{CD Command}  \\
  \hline
  \emph{Linux} & \texttt{cd /home/\emph{uname}/Documents} \\
  or & \texttt{cd $\sim$/Documents} \\
  \hline
  \emph{Mac OS} & \texttt{cd /Users/\emph{uname}/Documents} \\
  or & \texttt{cd $\sim$/Documents} \\
  \hline
  \emph{Windows 7} & \texttt{cd /c/Users/\emph{uname}/Documents} \\
  or & \texttt{cd /c/Users/\emph{uname}/My{\textbackslash} Documents} \\
  \hline
  \emph{Windows XP} & \texttt{cd /c/Documents{\textbackslash} and{\textbackslash} Settings/\emph{uname}/My{\textbackslash} Documents} \\
  \hline
\end{tabular}
  \caption{CD command to move into your documents directory on various Operating Systems. In these examples \texttt{\emph{uname}} should be replaced by your user name. The examples in \fref{fig:linux-files}, \fref{fig:mac-files}, and \fref{fig:win-files} are for the user \texttt{acain}.}
  \label{tbl:dirs}
\end{table}

\mynote{
\begin{itemize}
  \item You can find many resources on using Bash on the web, a simple overview of the basic commands can be found at \url{http://swinbrain.ict.swin.edu.au/wiki/Getting_started_with_Bash}.
  \item After you run the \texttt{cd} command, you can check which directory you are in using the \textbf{pwd} command. To do this just type \texttt{pwd} and press enter.
  \item Once you get used to the \texttt{cd} command you can start exploring the other commands.
\end{itemize}
}

\begin{figure}[p]
   \centering
   \includegraphics[width=0.9\textwidth]{./topics/programs-and-compilers/images/LinuxFiles} 
   \caption{Changing directories in Linux (Ubuntu)}
   \label{fig:linux-files}
\end{figure}

\begin{figure}[p]
   \centering
   \includegraphics[width=0.9\textwidth]{./topics/programs-and-compilers/images/MacFiles} 
   \caption{Changing directories in MacOS}
   \label{fig:mac-files}
\end{figure}


\begin{figure}[p]
   \centering
   \includegraphics[width=\textwidth]{./topics/programs-and-compilers/images/WindowsFiles} 
   \caption{Changing directories in Windows}
   \label{fig:win-files}
\end{figure}



% subsubsection bash (end)

% subsection terminal (end)
\clearpage
\subsection{Hello World} % (fold)
\label{sub:hello world}

There is a special program that should be the first program a software developer creates. This program is called `Hello World', it runs and displays the text `Hello World' onto the console. The program requires very little code, and can be used to make sure that everything is set up correctly on your machine.

This chapter covers the knowledge needed to create a simple program that outputs information to the console. This first step is important and will require you to have installed the compiler, see Appendix \ref{chap:InstallingTools} for instructions. At the end of this material you will be able to create and run programs using the languages presented.

This book focuses on programming concepts, and gives you the option of programming these using either the C or Pascal programming language. Both languages are very capable, with each having their own advantages and disadvantages. Pascal was designed as a teaching language and is easy to program with while still being very capable. C is very flexible and is the basis for a number of other languages.

\begin{multicols}{2}
  \ccode{lst:hello-world-c}{Hello World code in C.}{code/c/program-creation/hello-world.c}
  \columnbreak
  \pascode{lst:hello-world-pas}{Hello World code in Pascal.}{./topics/program-creation/pascal/HelloWorld.pas}
\end{multicols}

Listings \ref{lst:hello-world-c} and \ref{lst:hello-world-pas} show the code for the `Hello World' program written with the C and Pascal programming languages. Both programs result in the same output when run: they write the text `Hello World!' to the console. They both use the same basic programming structures, and they both go about performing the task in the same way. At this stage, however, they are both just fancy text. What we need to do is use a special tool to convert these into \emph{programs}.

% section hello world (end)

\input{topics/programs-and-compilers/concepts/Summary}


% section concepts_related_to_building_programs (end)

% ========================
% = Using these concepts =
% ========================

\clearpage
\section{Using these concepts: Compiling a Program} % (fold)
\label{sec:using_these_concepts_compiling_a_program}

Now that the concepts have been presented, let us have a look at how these can be used to create a program. We will take the code from Hello World, and use a compiler to turn this code into a program that we can then execute. This process is the same for large and small programs.

\subsection{Installing the Tools}
\label{subs:install}
Before we get started you will need to install a few tools on your computer. The good news is that all of the tools we need are free (and many are open source - meaning you could one day look at the code used to create these and contribute back to helping continue to make these tools awesome).

The tools we will include all of the tools we have covered in this chapter:

\begin{itemize}
   \item A \textbf{terminal} and \textbf{shell} used to run scripts, and 
   \item A \textbf{compiler} used to convert code into an executable program you can run,
   \item Visual Studio Code: A \textbf{text editor} designed to work with code,
   \item \textbf{SplashKit} - a set of tools and libraries that we can use to make more interesting programs as you start to learn to program. 
\end{itemize}

Installation instructions for Linux, macOS, and Windows can be found at \url{https://www.splashkit.io/articles/installation/}. Follow the instructions for your operating system, and at the language step you can choose to install either the C++ or Pascal compiler. Make sure to check that each step succeeds, and if you get stuck you can ask questions on the SplashKit forum (\url{https://splashkit.discourse.group}).

If you cannot get these working, then there is a Virtual Machine that is also available that comes with all of the software preinstalled. You can download this VM, and open and run it using something like VirtualBox (\url{https://www.virtualbox.org}) (which is free), VMware (\url{https://www.vmware.com}), or Parallels (\url{https://www.parallels.com/}).

\clearpage
\subsection{Creating a Project with the SplashKit Manager: skm} % (fold)
\label{sub:skm}

SplashKit is an open source set of tools and libraries that can help you make more interesting applications as you learn to program. Within SplashKit the SplashKit Manager (\textbf{skm}) is a command line tool designed to take your compiler call, and add the extra options needed to link it with the SplashKit library. So when you use \textbf{skm}, you add the text `skm' to the start of the instruction in the Terminal. Let's have a look at how this works.

The best way to get started with a SplashKit program is to create a new folder (directory) in your file system that will be used to store the code and other files related to that program. The SplashKit manager can be used to create an empty program file along with files that store setings for Visual Studio Code.

\csection{
The following commands create a \emph{MyProject} folder, setup a new C++ project, and then compile it into a program.

\bashcode{lst:c-create-project}{Creating a C++ project with skm.}{code/c/program-creation/create-cpp-project.sh}
}

\passection{
The following commands create a \emph{MyProject} folder, setup a new Pascal project, and then compile it into a program.

\bashcode{lst:pas-create-project}{Creating a Pascal project with skm.}{code/pascal/program-creation/create-fpc-project.sh}
}

Now that you have the project started, we can edit the code to make the program do something interesting.

\clearpage
\subsection{Making the Hello World Program} % (fold)
\label{sub:compiling_code}

\lref{lst:hello-world-c} and \lref{lst:hello-world-pas} show the source code for the Hello World program. To make this into a program you need to:

\begin{enumerate}
  \item Write the code into the program code file.
  \item Save the file.
  \item Compile it.
\end{enumerate}

Step 1 and 2 can be accomplished with any text editor, but the best ones to use highlight your code. Each programming language has rules that determine how its code must be structured, known as the language's \textbf{syntax}. Programming editors, such as Visual Studio Code, make use of this structure to help visually format the code to make it easier to read. This is called \textbf{syntax highlighting}. This highlighting can help you identify any little mistakes you make. 

\fref{fig:cpp-example} shows the Hello World code in Visual Studio Code with C++, which has a consistent look across all platforms. The figure also includes the command line instructions needed to setup a new SplashKit C++ project, and compile it.

\begin{figure}[htbp]
   \centering
   \includegraphics[width=0.8\textwidth]{./topics/programs-and-compilers/images/CppExample} 
   \caption{Editing and Compiling C++ code in Visual Studio Code}
   \label{fig:cpp-example}
\end{figure}

\fref{fig:fpc-example} shows the Hello World code in Visual Studio Code, which has a consistent look across all platforms. The figure also includes the command line instructions needed to setup a new SplashKit Pascal project, and compile it.

\begin{figure}[htbp]
   \centering
   \includegraphics[width=0.8\textwidth]{./topics/programs-and-compilers/images/FpcExample} 
   \caption{Editing and Compiling Pascal code in Visual Studio Code}
   \label{fig:fpc-example}
\end{figure}

Use Visual Studio Code to open the \textbf{folder} that contains your project. This will give you access to that project's files and configure Visual Studio Code using the settings that skm has provided.

\clearpage
\subsubsection{Compiling the Code} % (fold)
\label{ssub:running_the_compiler}

Once you have saved the code it is time to compile your program. For this you are going to need switch back to the \nameref{sub:terminal} (or open a new one if you have closed it - reemmber to change into the directory where you saved the source code file). 

When you run the compiler you need to give it two kinds of information: options, and the names of the files to compile. The compiler will read the code in the files you give it, and convert this to machine code as shown in \sref{sub:source_code_and_the_compiler} \nameref{sub:source_code_and_the_compiler}. The exact command you use depends on the language and compiler you are running.

\csection{
The C++ compiler we will be using is called \textbf{clang++} (or \textbf{g++} the \textbf{GNU C++ Compiler}). The command you need to run in the Terminal is shown in Listing \ref{lst:compile-hello-world-c}. The \emph{-o name} option tells the compiler the name of the program file to create. In our example this will compile the code in \emph{hello-world.cpp} and save the machine code into a program called \emph{HelloWorld}.

\bashcode{lst:compile-hello-world-c}{Compiling C code.}{code/c/program-creation/compile-hello-world.sh}
}

\passection{
The Pascal compiler we will be using is called \textbf{fpc}, which stands for the \textbf{Free Pascal Compiler}. The command you need to run in the Terminal is shown in Listing \ref{lst:compile-hello-world-pas}. The \emph{-S2} option is used to tell fpc to compile using the latest `Free Pascal' version of the language. In our example this will compile the code in \emph{HelloWorld.pas} and save the machine code into a program called \emph{HelloWorld}, which it gets from the name of the Pascal file.

\bashcode{lst:compile-hello-world-pas}{Compiling Pascal code.}{code/pascal/program-creation/compile-hello-world.sh}
}

Once the code compiles, you will have a program you can run! Running the program will load it into memory, and start it running through the steps you coded. In this case, the HelloWorld program will output the text `\texttt{Hello World!}' to the Terminal. To run the program you need to use its file name. The command you need to enter is shown in \lref{lst:run-hello-world}, and in \fref{fig:run-helloworld}. The \texttt{./} before the file tells Bash to look in the current directory for the program.

\bashcode{lst:run-hello-world}{Bash command to run HelloWorld}{topics/programs-and-compilers/run-hello-world.sh}

\begin{figure}[h]
   \centering
   \includegraphics[width=0.7\textwidth]{./topics/programs-and-compilers/images/HelloWorld} 
   \caption[Hello World Terminal]{Hello World run from the Terminal}
   \label{fig:run-helloworld}
\end{figure}

% subsubsection running_the_compiler (end)

\subsubsection{When things do not work} % (fold)
\label{ssub:when_things_do_not_work}

Compilers are very specific about the code you give it. If the source code you try to compile does not follow all of the syntax rules of the language then the compiler will fail, and end with an error message. These errors, called \textbf{syntax errors}, could be as small as missing a semicolon (;), or misspelling a name. To get your program to compile you will need to correct any syntax errors the compiler finds in your code.

\begin{figure}[h]
   \centering
   \includegraphics[width=0.8\textwidth]{./topics/programs-and-compilers/images/SyntaxErrors} 
   \caption{These Terminals show some syntax errors from programs that are missing a single semicolon (;)}
   \label{fig:syntax-errors}
\end{figure}

\fref{fig:syntax-errors} shows an example output caused by removing a single semicolon from the Hello World program's code. The numbers in the error messages give you an idea of where the compiler got to when it encountered the error. So the text \texttt{hello-world.c:6:} output from the C compiler indicates that the compiler got to line 6 before it encountered the error. The text \texttt{HelloWorld.pas(2,1)} output from the Pascal compiler indicates that it got up to line 2 character 1 before it encountered the error.

When the compiler encounters these issues it does not create the executable program. You need to learn to use these error messages to help you locate errors in your code, so that you can fix them, and then run the compiler again to generate your program.

\mynote{
\begin{itemize}
  \item Syntax errors are very common. Do not worry when this occurs to you.
  \item Always start with the first error. The compiler will try to continue compiling your code after it finds an error. This can mean that later errors do not really exist, once you fix the earlier ones.
  \item Unfortunately compiler error messages are not always very clear on what the cause of the error is. You need to learn how to read and understand these messages.
  \item To get good at programming requires lots of practice.
\end{itemize}
}

Here are a couple of things to check if you have compiler errors with the Hello World program:

\begin{itemize}
   \item Spaces in filename - avoid spaces in file names. The terminal uses spaces to separate different values in its instructions. In this context a space will then separate two values, rather than being captured as a space within the filename. If you do want a space use hyphens, or change case. For example \texttt{hello-world.cpp} or \texttt{HelloWorld.cpp} - there is no space here, but you can see the different words clearly.
   \item Missing semicolons - this is easy to do. Check for a missing \texttt{;} at the end of the previous line.
   \item Identifier not found - check the name you are using in the code. Most likely it is a small error. Compilers are often case sensitive, so WriteLine and writeline are different names.
   \item If you get errors with the hello world code, double check against the code provided.
\end{itemize}


% subsubsection when_things_do_not_work (end)



% section the_compiler (end)


% section using_these_concepts_compiling_a_program (end)

% =============
% = C Section =
% =============
\clearpage
\def\pageLang{c}
\section{Building Programs in C} % (fold)
\label{sec:building_programs_in_c}

The C programming language is very powerful and flexible. In this section you will see the tools that you need to install to start programming with C on your computer.

To explore this material you will need a small program

\subsection{Hello World in C} % (fold)
\label{sub:hello_world_in_c}

The program used in the last chapter was the classic Hello World program. The C code for this is shown in \lref{lst:hello-world-c-c}. This code will be explained in future chapters, for the moment you will need to copy\footnote{Do not just copy and paste it out of the text, type it in yourself as this will help you learn the concepts being covered.} it into a source code file, save it, and then compile it.

\csection
{
\ccode{lst:hello-world-c-c}{Hello World code in C.}{code/c/program-creation/hello-world.c}
}

\mynote{
\begin{itemize}
  \item This section will show you the tools you need to install to get started.
  \item The Hello World program is a good start, it outputs text to the Terminal when it is executed.
  \item You will need to install the following tools to type, compile, and run this:
  \begin{enumerate}
    \item The \textbf{gcc compiler} to compile your code.
    \item A \textbf{text editor} to enter your code into.
  \end{enumerate}
  \item This chapter will also show you how to create graphical programs using SwinGame.
\end{itemize}
}

% subsection hello_world_in_c (end)
\subsection{Installing the GNU C Compiler} % (fold)
\label{sub:installing_gcc}

You can get the GNU C Compiler (gcc) for Linux, Mac OS, and Windows. The following section describe where it is you can find these programs, and how to install them.

\subsubsection{Installing gcc on Linux} % (fold)
\label{ssub:linux}

It should be relatively easy to install \textbf{gcc} for Linux. If gcc is not already installed on your machine you should check with the Linux distribution on how to install the build tools. To install this on Ubuntu linux use the command shown in \lref{lst:apt-get-gcc}.

\bashcode{lst:apt-get-gcc}{The command line instruction to install gcc on Ubuntu}{topics/programs-and-compilers/c/apt-get-gcc.txt}

% subsubsection linux (end)

\subsubsection{Installing gcc on Mac OS} % (fold)
\label{ssub:installing_gcc_on_mac_os}

To install \textbf{gcc}\footnote{XCode also installs an improved C compiler called \textbf{clang}. To use clang just replace \texttt{gcc} with \texttt{clang} in the instructions in this book. For example, rather than \texttt{gcc -o HelloWorld hello-world.c} use \texttt{clang -o HelloWorld hello-world.c}} on Mac OS you need to install the \textbf{XCode} developer tools. You can get the latest version of XCode from the \textbf{Mac App Store} (yes it is free).

XCode links:
\begin{itemize}
  \item Apple's XCode website \url{http://developer.apple.com/xcode/}
  \item XCode in the Mac App Store  \url{http://itunes.apple.com/au/app/xcode/id448457090?mt=12}
  \item For older version of Mac OS, download XCode from \href{https://developer.apple.com/downloads/download.action?path=Developer_Tools/xcode_3.2.2_developer_tools_beta_20728/xcode322_2148_developerdvd.dmg}{Apple's developer site}.\footnote{These will require you to have an Apple Developer account, which can be done for free at \url{http://developer.apple.com/programs/register}.}
\end{itemize}

% subsubsection installing_gcc_on_mac_os (end)

\subsubsection{Installing gcc on Windows} % (fold)
\label{ssub:installing_gcc_on_windows}

To get \textbf{gcc} on Windows you need to install the \textbf{MinGW} program. This includes the MinGW Shell, the Terminal equivalent for Windows. Read the Getting Started page from the links below, and follow the steps for the \textbf{Graphical User Interface Installer}. When following the installer make sure that you choose to install the MinGW Shell as well as the C and C++ compilers.

\begin{itemize}
  \item The project's Website \url{www.mingw.org}
  \item Instructions for installing MinGW: \url{http://www.mingw.org/wiki/Getting_Started}
  \item Follow the link from the Getting Started article, or download MinGW from Source Forge at \url{http://sourceforge.net/projects/mingw/}
  \item It is a good idea to restart after installing MinGW to make sure all its settings are applied.
\end{itemize}

% subsubsection installing_gcc_on_windows (end)
\clearpage
\subsubsection{Testing gcc has installed} % (fold)
\label{ssub:testing_your_install}

To test that you have successfully installed gcc you can do the following:
\begin{enumerate}
  \item Open a Terminal window. (See the notes on the \nameref{sub:terminal} page for the location of the Terminal program)
  \item At the terminal type \textbf{\texttt{gcc}}. You should see something similar to \fref{fig:gcc-install}. If you see the message `\texttt{-bash: gcc: command not found}' this means the install has not worked correctly. Please review the install steps for your Operating System or get someone to check your install.
\end{enumerate}

\begin{figure}[h]
   \centering
   \includegraphics[width=0.8\textwidth]{./topics/programs-and-compilers/images/gccInstall} 
   \caption{Testing gcc has been installed}
   \label{fig:gcc-install}
\end{figure}

% subsection testing_your_install (end)

% subsection installing_gcc (end)
\clearpage
\subsection{Installing a Text Editor} % (fold)
\label{sub:installing_a_text_editor}

When you are programming you will spend most of your time in a Text Editor. The best kinds of editors for program code include syntax highlighting where the editor uses details about the language to highlight parts of the code as you write it. This can make it easier to find and fix small errors as you go.

\subsubsection{gEdit on Linux} % (fold)
\label{ssub:gedit_on_linux}

The \textbf{gEdit} program is a syntax highlighting text editor for Linux. In Ubuntu this is the standard \textbf{Text Editor} program found in \emph{Accessories}. This should already be installed. Another good text editors is \texttt{Kate}.

% subsubsection gedit_on_linux (end)

\subsubsection{TextWrangler Mac OS} % (fold)
\label{ssub:textwrangler_or_textmate_on_mac_os}

TextWrangler is a free syntax highlighting text editor for Mac OS. You can download and install this from the \textbf{Mac App Store}, or from their web site.

\begin{itemize}
  \item Mac App Store link: \url{http://itunes.apple.com/au/app/textwrangler/id404010395?mt=12}
  \item Website: \url{http://www.barebones.com/products/textwrangler/}
  \item TextMate is another good editor, though it is not free: \url{http://macromates.com/}
\end{itemize}

% subsubsection textwrangler_or_textmate_on_mac_os (end)

\subsubsection{Notepad++ for Windows} % (fold)
\label{ssub:notepad_for_windows}

Notepad++ is a free syntax highlighting text editor for Windows. You can download and install this from the Notepad++ website.

\begin{itemize}
  \item Notepad++: \url{http://notepad-plus-plus.org/}
\end{itemize}

% subsubsection notepad_for_windows (end)

% subsection installing_a_text_editor (end)
\clearpage
\subsection{Graphical Applications with SwinGame} % (fold)
\label{sub:graphical_applications_with_swingame}

SwinGame is a 2D game creation library. It contains a number of resources that you can use to create small games using the C programming language. To get started with SwinGame you need to download a \emph{template} from the website. The template includes everything you need to get started creating a game in SwinGame.

\subsubsection{Coding a SwinGame} % (fold)
\label{ssub:coding_a_swingame}

The code for your SwinGame can be found in the \textbf{\texttt{src}} folder. This includes a \texttt{\textbf{GameMain.c}} file, where you can place your source code.

% subsubsection coding_a_swingame (end)

\subsubsection{Compiling a SwinGame} % (fold)
\label{ssub:compiling_a_swingame}

There are a number of steps that you need to follow to compile a SwinGame. Fortunately, all of these are written for you in the script called \textbf{\texttt{build.sh}}. This script to included in the project template. 

% subsubsection compiling_a_swingame (end)

\subsubsection{Running a SwinGame} % (fold)
\label{ssub:running_a_swingame}

% subsubsection running_a_swingame (end)

% subsection graphical_applications_with_swingame (end)


% section building_programs_in_c (end)


% chapter building_programs (end)