\clearpage
\subsection{Hello World} % (fold)
\label{sub:hello world}

There is a special program that should be the first program a software developer creates. This program is called `Hello World', it runs and displays the text `Hello World' onto the console. The program requires very little code, and can be used to make sure that everything is set up correctly on your machine.

This chapter covers the knowledge needed to create a simple program that outputs information to the console. This first step is important and will require you to have installed the compiler, see Appendix \ref{chap:InstallingTools} for instructions. At the end of this material you will be able to create and run programs using the languages presented.

This book focuses on programming concepts, and gives you the option of programming these using either the C or Pascal programming language. Both languages are very capable, with each having their own advantages and disadvantages. Pascal was designed as a teaching language and is easy to program with while still being very capable. C is very flexible and is the basis for a number of other languages.

\begin{multicols}{2}
  \ccode{lst:hello-world-c}{Hello World code in C.}{code/c/program-creation/hello-world.c}
  \columnbreak
  \pascode{lst:hello-world-pas}{Hello World code in Pascal.}{./topics/program-creation/pascal/HelloWorld.pas}
\end{multicols}

Listings \ref{lst:hello-world-c} and \ref{lst:hello-world-pas} show the code for the `Hello World' program written with the C and Pascal programming languages. Both programs result in the same output when run: they write the text `Hello World!' to the console. They both use the same basic programming structures, and they both go about performing the task in the same way. At this stage, however, they are both just fancy text. What we need to do is use a special tool to convert these into \emph{programs}.

% section hello world (end)
