\clearpage
\def\pageLang{none}
\section{Installing a Text Editor} % (fold)
\label{sec:installing_a_text_editor}

When you are programming you will spend most of your time in a Text Editor. The best kinds of editors for program code include \emph{syntax highlighting} where the editor uses details about the language to highlight parts of the code as you write it. This can make it easier to find and fix small errors as you go.

There are many different syntax highlighting text editors, each will support different programming languages and operating systems. The following text editors support the C and Pascal programming languages. Install the one appropriate for your operating system.

\begin{itemize}
  \item \nameref{ssub:gedit_on_linux}
  \item \nameref{ssub:textwrangler_or_textmate_on_mac_os}
  \item \nameref{ssub:notepad_for_windows}
\end{itemize}

\subsubsection{gEdit on Linux} % (fold)
\label{ssub:gedit_on_linux}

The \textbf{gEdit} program is a syntax highlighting text editor for Linux. In Ubuntu this is the standard \textbf{Text Editor} program found in \emph{Accessories}. This will come as part of the operating system installation, so you do not need to install a separate editor. 

% subsubsection gedit_on_linux (end)

\subsubsection{TextWrangler on Mac OS} % (fold)
\label{ssub:textwrangler_or_textmate_on_mac_os}

TextWrangler is a free syntax highlighting text editor for Mac OS. You can download and install this from the \textbf{Mac App Store}, or from their web site.

\begin{itemize}
  \item Mac App Store link: \url{http://itunes.apple.com/au/app/textwrangler/id404010395?mt=12}
  \item Website: \url{http://www.barebones.com/products/textwrangler/}
  \item TextMate is another good editor, though it is not free: \url{http://macromates.com/}
\end{itemize}

% subsubsection textwrangler_or_textmate_on_mac_os (end)

\subsubsection{Notepad++ on Windows} % (fold)
\label{ssub:notepad_for_windows}

Notepad++ is a free syntax highlighting text editor for Windows. You can download and install this from the Notepad++ website.

\begin{itemize}
  \item Notepad++: \url{http://notepad-plus-plus.org/}
\end{itemize}

% subsubsection notepad_for_windows (end)

% subsection installing_a_text_editor (end)