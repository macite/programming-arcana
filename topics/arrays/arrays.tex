\chapter{Managing Multiple Values} % (fold)
\label{cha:managing_multiple_values}

\begin{quote}
  \Fontlukas\Large
  \renewcommand{\LettrineTextFont}{\relax}
  \lettrine[image=true,lines=3,lraise=0.1]
  {T}{rue} magic has the power to affect not just one, but many targets. The secrets you will learn today will make it possible for your magic to be applied to many targets. The key is to\ldots
\end{quote}

\bigskip


Previous chapters have introduced a number programming artefacts for you to create within your code. However, when it comes to working with data in your programs you have been limited in the way you deal with multiple values. This chapter introduces the concepts and practices that make it easier to work with multiple values in your code.

When you have understood the material in this chapter you will be able to work with multiple values more easily, allowing your programs to work with many data values. 

\minitoc

% ============
% = Concepts =
% ============
\input{topics/arrays/concepts/array-concepts}

% =============
% = C Section =
% =============
\clearpage
\def\pageLang{c}
\section{Managing Multiple Values in C} % (fold)
\label{sec:arrays_in_c}

\subsection{Implementing Statistics Calculator in C} % (fold)
\label{sub:implementing_statistics_calculator_in_c}

\sref{sec:arrays_using_these_concepts} of this Chapter introduced the Statistics Calculator. A partial implementation of this program is shown in Listing \ref{lst:c-stats-calc}, with the logic in the \texttt{max} and \texttt{variance} functions still to be implemented. This program reads a number of values from the user into an array, and then calculates and outputs the \textbf{sum}, \textbf{mean}, \textbf{variance}, and \textbf{maximum} value from this data.

\straightcode{\ccode{lst:c-stats-calc}{C code for the Statistics Calculator}{code/c/array/simple-stats.c}}

\mynote{
\begin{itemize}
  \item \texttt{strings.h} is included to give access to the various functions needed to manipulate string values. See the comments associated with \lref{clst:populate_array}.
  \item \texttt{math.h} is included to give access to the \texttt{pow} function that will be needed in the implementation of the \texttt{variance} function.
  \item Arrays in C are always passed by reference.
  \item C does not keep track of the size of an array, the \texttt{size} parameter in each function call carries this data along with the array.
  \item The \texttt{DATA\_SIZE} constant stores the number of values that will be stored in the array. This can easily be changed to allow the program to read a different number of values.
\end{itemize}
}

% subsection implementing_statistics_calculator_in_c (end)

\clearpage
\subsection{C Array Declaration} % (fold)
\label{sub:c_array_declaration}

C allows you to declare variables that are arrays. This is done using the \texttt{[ ]} to denote the number of elements in the array (\emph{n}). Indexes will then be \emph{0} to \emph{n-1}.

\csyntax{csynt:type-decl-variable-decl}{Array Variable and Type Declarations}{arrays/array-decl}

\csection{\ccode{clst:test-array}{C code demonstrating array declaration}{code/c/array/test-array.c}}

\mynote{
\begin{itemize}
  \item This is the C syntax to declare a \nameref{sub:array}.
  \item Arrays in C do not remember their length, you must keep track of this yourself.
  \item You can initialise an array when it is declared using a list of values in braces (\{\ldots\}). This can only be done to initialise arrays, and is not valid elsewhere.
  \item The size of the array must be able to be determined at compile time.
\end{itemize}
}

% subsection c_array_declaration (end)
\subsection{C Array Copying} % (fold)
\label{sub:c_array_copying}

In C you cannot use simple assignment to copy all of the elements of an array into another array. Instead you can use the \texttt{memcpy} (memory copy) function to perform this task for you. It copies a chunk of memory from one location to another.

\begin{table}[h]
  \centering
  \begin{tabular}{|c|p{9.5cm}|}
    \hline
    \multicolumn{2}{|c|}{\textbf{Function Prototype}} \\
    \hline
    \multicolumn{2}{|c|}{} \\
    \multicolumn{2}{|c|}{\texttt{void *memcpy(void *destination, const void *source, size\_t num )}} \\
    \multicolumn{2}{|c|}{} \\
    \hline
    \multicolumn{2}{|c|}{\textbf{Returns}} \\
    \hline
    \texttt{void *} & Destination is returned, can be ignored. \\
    \hline
    \textbf{Parameter} & \textbf{Description} \\
    \hline
    \texttt{ destination } & The location where the data is copied to.\\
    & \\
    \texttt{ source } & The data to copy.\\
    & \\
    \texttt{ num } & The number of bytes to copy. \\
    \hline
  \end{tabular}
  \caption{Details of the \texttt{memcpy} function}
  \label{tbl:memcpy}
\end{table}

The \texttt{memcpy} function needs to be told the number of bytes to copy. The \texttt{sizeof} operator can be used to get this information for a type or variable. The details of this operator are shown in \fref{csynt:sizeof}. 

\csyntax{csynt:sizeof}{The \texttt{sizeof} operator lets you get the size, in bytes, of a type or variable.}{arrays/sizeof}

\csection{\ccode{clst:array-copy}{C code demonstrating array copying}{code/c/array/array-copy.c}}

\mynote{
\begin{itemize}
  \item The \texttt{size\_t} type is a whole number used to store the size of types.
  \item To get the size of a variable or type using the \texttt{sizeof} operator.
  \item You can think of the \texttt{sizeof} operator as being a function that returns the size of a type or variable.
  \item The expression in \texttt{sizeof} must be a variable or type name.
\end{itemize}
}

% subsection c_array_copying (end)
\clearpage
\subsection{C For Loop} % (fold)
\label{sub:c_for_loop}

The \nameref{sub:for_loop} in C can do much more than just counting, but that is its primary purpose. You can use this to implement the logic to process each element of an array.

The for loop itself is controlled by three aspects. The first is an initialiser, it sets the first value for the control variable (usually \texttt{i} if you are  using it to index an array). The second part is the condition, the body will run \emph{while} this is true just like a \nameref{sub:c_while_loop}. The third part is a post loop increment, you use this to move the index to the next value. 

The standard for loop is: \csnipet{for(i = 0; i < size; i++)\{...\}}. This can be read as `for \texttt{i} starts at 0, while \texttt{i} is less than \texttt{size}, do the following then increment \texttt{i}'. If \texttt{size} is three then this counts 0, 1, 2. Repeating the body of the loop three times.

\csyntax{csynt:looping-for-loop}{a For Loop}{looping/for-loop}

\csection{\ccode{clst:test-for}{Code illustrating the for loop in C}{code/c/array/test-for.c}}

\mynote{
\begin{itemize}
  \item This is the C syntax for implementing a \nameref{sub:for_loop}.
  \item The \emph{initialiser} of the for loop is used to set the initial values before the loop body is first executed.
  \item The \emph{control} is a \emph{condition}, the loop executes \emph{while}\footnote{In the same way a while loop does, checking the condition to determine if the body should execute or be skipped when the condition is checked.} this is true.
  \item The \emph{increment} is a post loop action, used to move to the next index.
\end{itemize}
}

% subsection c_while_loop (end)
\clearpage
\subsection{C String} % (fold)
\label{sub:c_string}

C was designed to build for use with the Unix operating system. When the language was designed string manipulation was not a high priority, and therefore C does not have built in capabilities to perform tasks like concatenating strings, and copying strings (i.e. assigning a string a value after it has been declared).

Working with c-strings requires that you think about how the text is represented in the computer. \tref{tbl:c-string-fred} shows the characters used to store the text value `Fred'. 

As C does not keep the length of the array there needs to be a means of determining how long the string is. The method that C choose was to place a \textbf{sentinel} value at the end of the string. This marks the position in the array where the string ends. The sentinel is the \texttt{null} character, the one with value the \texttt{0}.

\begin{table}[h]
\begin{minipage}{\textwidth}
  \centering
\begin{tabular}{|l|c|c|c|c|c|}
\hline
Characters: & F & r & e & d & \texttt{\textbackslash 0} \\
\hline
Bytes Values\footnote{Byte values are shown as decimal.}: & \texttt{70} & \texttt{114} & \texttt{101} & \texttt{100} & \texttt{0} \\
\hline
\end{tabular}
\caption{The characters and byte values for the c-string containing the text `Fred'}
\label{tbl:c-string-fred}
\end{minipage}
\end{table}

Space characters are distinct from the \texttt{null} character. \tref{tbl:c-string-fred-smith} shows the characters involved in storing the text `Fred Smith'. The space character is the value 32, and the sentinel value only appears at the end of the c-string. To store `Fred Smith' you need an array that can store at least 11 characters. Ten for the characters in the name, and one for the sentinel.

\begin{table}[h]
\begin{minipage}{\textwidth}
  \centering
\begin{tabular}{|l|c|c|c|c|c|c|c|c|c|c|c|c|}
\hline
Characters: & F & r & e & d &  & S & m & i & t & h & \texttt{\textbackslash 0}\\
\hline
Bytes Values\footnote{Byte values are shown as decimal.}: & \texttt{70} & \texttt{114} & \texttt{101} & \texttt{100} & \texttt{32} & \texttt{83} &\texttt{109} & \texttt{105} & \texttt{116} & \texttt{104} & \texttt{0} \\
\hline
\end{tabular}
\caption{Characters for `Fred Smith', the space has the character value 32.}
\label{tbl:c-string-fred-smith}
\end{minipage}
\end{table}

It is possible for an array to have more characters that are needed. \tref{tbl:c-string-fred-null-smith} shows an array with 11 characters that is storing the c-string `Fred'. The \texttt{null} character at index 4 (the $5^{th}$ character) ends the c-string and the remainder of the data in the array will be ignored by the c-string functions.

\begin{table}[h]
\begin{minipage}{\textwidth}
  \centering
\begin{tabular}{|l|c|c|c|c|c|c|c|c|c|c|c|c|}
\hline
Characters: & F & r & e & d & \texttt{\textbackslash 0} & S & m & i & t & h & \texttt{\textbackslash 0}\\
\hline
Bytes Values\footnote{Byte values are shown as decimal.}: & \texttt{70} & \texttt{114} & \texttt{101} & \texttt{100} & \texttt{0} & \texttt{83} &\texttt{109} & \texttt{105} & \texttt{116} & \texttt{104} & \texttt{0} \\
\hline
\end{tabular}
\caption{This would only print `Fred', as the 0 character indicates the end of the c-string}
\label{tbl:c-string-fred-null-smith}
\end{minipage}
\end{table}

The code in \lref{clst:test-strings} shows some examples of the main operations you may want to perform on strings. This includes the following actions:
\begin{itemize}
  \item \textbf{Initialisation}: Creating and initialising a string.
  \item \textbf{Input}: Reading words, and lines, from the Terminal.
  \item \textbf{Comparison}: Checking if two strings are equal. Notice that you also need to check the \texttt{null} value.
\end{itemize}
Other common string operations are found in \lref{clst:populate_array}. These included:
\begin{itemize}
  \item \textbf{Copy}: Assigning one string to another, as you cannot use the assignment statement to achieve this in C.
  \item \textbf{Concatenate}: Adding one string to the end of another.
\end{itemize}

\csection{\ccode{clst:test-strings}{Code illustrating working with Strings in C}{code/c/array/test-string.c}}

\mynote{
\begin{itemize}
  \item In C a String is an array of characters. There is little built in support beyond this in the C language itself. The Standard C libraries include functions that can be used to work with String data, in \texttt{strings.h}.
  \item \textbf{Remember} to ask for enough space to store the text and the sentinel value when declaring a c-string. If you want to store 4 characters then you need to ask for an array with space for 5, the 4 characters + and 1 sentinel value.
  \item The c-string functions will look for the null character. If the null character is missing from the end of the c-string then these functions will not work as you want. The problem is that they may appear to be working, though in reality they are interacting with memory that is not associated with the c-string you are working on.
  \item \textbf{Take care when working with c-strings!} Many security issues in software relate to incorrect handling of c-strings.
\end{itemize}
}

\clearpage

\subsubsection{Print and Scanning in Strings} % (fold)
\label{ssub:print_and_scanning_in_strings}

The \texttt{stdio.h} header also provides version of \texttt{printf} and \texttt{scanf} that are used to write values to, and read values from strings. The \texttt{sprintf} function writes data into a \texttt{destination} string, whereas the \texttt{sscanf} function reads data out of a source \texttt{string}. \tref{tbl:sprintf} shows the details for the \texttt{sprintf} function, \tref{tbl:sscanf} shows the details for \texttt{sscanf}.

\begin{table}[h]
  \centering
  \begin{tabular}{|c|p{9cm}|}
    \hline
    \multicolumn{2}{|c|}{\textbf{Function Prototype}} \\
    \hline
    \multicolumn{2}{|c|}{} \\
    \multicolumn{2}{|c|}{\texttt{int sprintf(char *destination, const char *format, \ldots )}} \\
    \multicolumn{2}{|c|}{} \\
    \hline
    \multicolumn{2}{|c|}{\textbf{Returns}} \\
    \hline
    \texttt{int} & The number of characters written to the \texttt{destination} by \texttt{sprintf}. \\
    \hline
    \textbf{Parameter} & \textbf{Description} \\
    \hline
    \texttt{ destination } & The string to write the output into. \textbf{Warning:} You are responsible for ensuring there is enough space.\\
    & \\
    \texttt{ format } & The text that is to be written to the Terminal. This text may contain format tags to include other values. This is the same as \texttt{printf}, see Figure \ref{csynt:program-creation-format-string} for the syntax of the format tag. \\
    & \\
    \texttt{\ldots}   & Optional values, must have at least as many values as format tags. \\
    \hline
  \end{tabular}
  \caption{Parameters that must be passed to \texttt{sprintf}}
  \label{tbl:sprintf}
\end{table}


\begin{table}[h]
  \centering
  \begin{tabular}{|c|p{9.5cm}|}
    \hline
    \multicolumn{2}{|c|}{\textbf{Function Prototype}} \\
    \hline
    \multicolumn{2}{|c|}{} \\
    \multicolumn{2}{|c|}{\texttt{int sscanf(const char *source, const char *format, \ldots )}} \\
    \multicolumn{2}{|c|}{} \\
    \hline
    \multicolumn{2}{|c|}{\textbf{Returns}} \\
    \hline
    \texttt{int} & The number of values read by \texttt{sscanf}. \\
    \hline
    \textbf{Parameter} & \textbf{Description} \\
    \hline
    \texttt{ source } & The string from which the input is read.\\
    & \\
    \texttt{ format } & The format specifier describing what is to be read from the Terminal. This is the same as with \texttt{scanf}, see \tref{tbl:format specifiers}. \\
    & \\
    \texttt{\ldots}   & The variables into which the values will be read. There must be at least as many variables as format tags in the format specifier. \\
    \hline
  \end{tabular}
  \caption{Parameters that must be passed to \texttt{sscanf}}
  \label{tbl:sscanf}
\end{table}

\mynote{
\begin{itemize}
  \item These functions are useful for converting data to and from string values.
  \item You can use \texttt{sprintf} to convert numeric values and store them in a string. Though care must be taken to ensure that there is sufficient space for these values in the destination string.
  \item \texttt{sscanf} can be used to read values out of a string. For example, you could read a numeric value out of a string value entered by the user.
\end{itemize}
}



% subsubsection print_and_scanning_in_strings (end)

\clearpage
\subsubsection{Example use of C-string functions} % (fold)
\label{ssub:example_use_of_c_string_functions}

The Statistics Calculator requires some string manipulation to generate the prompt that will be shown to the user. The prompt is created from the text `\texttt{Enter value}', the value of \texttt{i + 1}, and the text `\texttt{: }'. The four function calls needed to achieve this are shown in \fref{fig:cstringops}.

\begin{figure}[htbp]
   \centering
   \includegraphics[width=\textwidth]{./topics/arrays/images/CStringOps} 
   \caption{Example usage of c-string functions from the Statistics Calculator}
   \label{fig:cstringops}
\end{figure}

\mynote{
\begin{itemize}
  \item The steps shown in \fref{fig:cstringops} perform the following actions:
  \begin{enumerate}
      \item The first instruction uses \texttt{strncpy} to copy the characters in `\texttt{Enter value}' into the \texttt{prompt}. The \texttt{null} character must also be copied so that \texttt{strncat} knows where the string currently ends.
      \item Step 2 uses \texttt{sprintf} to print the decimal value of \texttt{(i + 1) \% 100} into the \texttt{buffer}. This uses \texttt{\% 100} so that only two characters\footnote{\texttt{(i + 1) \% 100} is the remainder of the value i incremented by one after dividing by 100. This ensures that it is always between 0 and 99. For example, when \texttt{i} is 106, \texttt{i + 1} is 107, and \texttt{107 \% 100} is 7, the remainder of dividing 107 by 100.} and the terminal are ever written into the \texttt{buffer}. 
      This assumes that the value of \texttt{i} is 0, so \texttt{i + 1} is \texttt{1}.
      \item Next \texttt{strncat} is used to concatenate \texttt{prompt} and \texttt{buffer}. This copies the text in \texttt{buffer} to the end of \texttt{prompt}, and makes sure that there is a \texttt{null} character at the end. So the \texttt{n} in this case indicate the maximum number of actual characters to add to the destination, as the \texttt{null} is effectively moved in this function.
      \item The operation is completed by concatenating the final `\texttt{: }' to the end of the string.
  \end{enumerate}
  \item Notice that there is one additional character left in this \texttt{prompt} at the end, this gives space for two digit values, e.g. `\texttt{Enter value 10:}'.
\end{itemize}
}


% subsubsection example_use_of_c_string_functions (end)

% subsection c_string (end)
\clearpage
\subsection{C Function (with Array Parameters)} % (fold)
\label{sub:c_fn_with_array}

In C you can only use \nameref{sub:pass_by_reference} to pass an array to a Function or Procedure. There are two ways of passing arrays by reference in C: one uses the bracket notation (\texttt{type name[ ]}), the other an asterisks notation (\texttt{type *name}). The asterisks notation is more general pass by reference, and will be covered in a later chapter in more details. The brackets notation accomplishes the same task, and indicates that the passed data will be an array.\footnote{Which is passed by reference, as arrays are always passed by reference in C.}

The optional \textbf{\texttt{const}} operator allows you to indicate that the passed in value will not be changed in the Function or Procedure. This is important with strings, as if you want to pass a string literal to a parameter it must be a \texttt{const char *}, as the literal cannot be changed.

\csyntax{csynt:fn-with-array}{Functions with Array Parameters}{arrays/fn-array-param}

\begin{figure}[p]
  \csection{\ccode{clst:test-array-passing}{Code illustrating array passing in C}{code/c/array/test-array-passing.c}}
\end{figure}

\mynote{
\begin{itemize}
  \item This syntax shows you how to code \nameref{sub:pass_by_reference} into your Functions and Procedures in C.
  \item See \lref{clst:test-array-passing} for examples of the different ways of declaring pass by reference parameters in C.
  \item Notice that in the call you \textbf{do not} need to get the address of arrays, as you would do with other types that are passed by reference. This is because C does this for you in the background. Remember arrays are always passed by reference in C.
  \item When using the \texttt{[ ]} syntax you do not specify the size of the array. This allows arrays of varying size to be passed into the Function or Procedure. The \texttt{size} parameter is then used by \emph{convention} to carry across the size of the array.
\end{itemize}
}

% subsection c_fn_with_array (end)

% section more_data_types_in_c (end)


% ==================
% = Pascal Section =
% ==================
\cleardoublepage
\def\pageLang{pas}
\section{Managing Multiple Values in Pascal} % (fold)
\label{sec:arrays_in_pascal}

\subsection{Implementing Statistics Calculator in Pascal} % (fold)
\label{sub:implementing_statistics_calculator_in_pas}

\sref{sec:arrays_using_these_concepts} of this Chapter introduced the Statistics Calculator. A partial implementation of this program is shown in Listing \ref{lst:pas-stats-calc}, with the logic in the \texttt{max} and \texttt{variance} functions still to be implemented. This program reads a number of values from the user into an array, and then calculates and outputs the \textbf{sum}, \textbf{mean}, \textbf{variance}, and \textbf{maximum} value from this data.

\straightcode{\pascode{lst:pas-stats-calc}{Pascal code for the Statistics Calculator}{code/pascal/array/SimpleStats.pas}}

\mynote{
\begin{itemize}
  \item \texttt{SysUtils} is used to give access to the \texttt{TryStrToFloat} function. This \emph{tries} to convert a string to a double, and returns a boolean to indicate if it succeeded.
  \item The arrays in this program are passed by reference using \textbf{const} (for in only) or \textbf{var} (for in and out).
  \item The \texttt{Low(\ldots)} function gives you the first index of the array, \texttt{High(\ldots)} gives you the last index of the array, and \texttt{Length(\ldots)} tells you the number of elements in the array.
\end{itemize}
}

% subsection implementing_statistics_calculator_in_c (end)

\clearpage
\subsection{Pascal Array Declaration} % (fold)
\label{sub:pas_array_declaration}

Pascal allows you to declare array variables and parameters. 

\passyntax{psynt:type-decl-variable-decl}{Array Variable and Parameter Declarations}{arrays/array-decl}

\passection{\pascode{plst:test-array}{Pascal code demonstrating array declaration}{code/pascal/array/ArrayTest.pas}}

\mynote{
\begin{itemize}
  \item This is the Pascal syntax to declare an \nameref{sub:array}.
  \item You can initialise an array when it is declared using a list of values in parenthesis (\{\ldots\}). This can only be done to initialise arrays, and is not valid elsewhere.
  \item You can use an assignment statement to copy the contents from one array over another of the same size.
\end{itemize}
}

% subsection c_array_declaration (end)
\clearpage
\subsection{Pascal For Loop} % (fold)
\label{sub:pas_for_loop}

The \nameref{sub:for_loop} in Pascal can be used to implement the logic to process each element of an array. With the for loop you specify a control variable, and the range of values it will loop over. When the loop is started the control variable is assigned the initial value, at the end of each loop this value is incremented (for \texttt{to}) or decremented (for \texttt{downto}) until it has processed all values in the indicated range.

\passyntax{psynt:looping-for-loop}{a for loop}{arrays/for-loop}

\passection{\pascode{plst:test-for}{Code illustrating the for loop in Pascal}{code/pascal/array/TestFor.pas}}

\mynote{
\begin{itemize}
  \item This is the Pascal syntax for implementing a \nameref{sub:for_loop}.
  \item The first expression is the \emph{initial value} given to the variable.
  \item If you use \texttt{to} the variable's value is increased by one at the end of each loop, ending the loop when the variable's value is larger than the second expression.
  \item Alternatively \texttt{downto} decreased the variable's value by one at the end of each loop, ending when the variable's value is less than the second expression.
\end{itemize}
}

% subsection c_while_loop (end)
\subsection{Pascal Array Functions} % (fold)
\label{sec:pascal_array_functions}

Pascal includes a number of functions to help you work with arrays. These allow you to determine the length of the array, and its range of indexes. These functions are typically used in conjuncture with the for loop.

\begin{table}[h]
  \centering
  \begin{tabular}{|c|p{9cm}|}
    \hline
    \multicolumn{2}{|c|}{\textbf{Function Prototypes}} \\
    \hline
    \multicolumn{2}{|c|}{} \\
    \multicolumn{2}{|c|}{\texttt{function Length( arg: array of \ldots ): Integer}} \\
    \multicolumn{2}{|c|}{} \\
    \hline
    \multicolumn{2}{|c|}{\textbf{Returns}} \\
    \hline
    \texttt{Integer} & The number of elements in the \texttt{arg} array. \\
    \hline
    \textbf{Parameter} & \textbf{Description} \\
    \hline
    \texttt{ arg } & The array you want to get the length of.\\
    \hline
    \multicolumn{2}{c}{} \\
    \hline
    \multicolumn{2}{|c|}{} \\
    \multicolumn{2}{|c|}{\texttt{function Low( arg: array of \ldots ): Integer}} \\
    \multicolumn{2}{|c|}{} \\
    \hline
    \multicolumn{2}{|c|}{\textbf{Returns}} \\
    \hline
    \texttt{Integer} & The lowest index in the \texttt{arg} array. \\
    \hline
    \textbf{Parameter} & \textbf{Description} \\
    \hline
    \texttt{ arg } & The array you want to get the lowest index of.\\
    \hline
    \multicolumn{2}{c}{} \\
    \hline
    \multicolumn{2}{|c|}{} \\
    \multicolumn{2}{|c|}{\texttt{function High( arg: array of \ldots ): Integer}} \\
    \multicolumn{2}{|c|}{} \\
    \hline
    \multicolumn{2}{|c|}{\textbf{Returns}} \\
    \hline
    \texttt{Integer} & The last index in the \texttt{arg} array. \\
    \hline
    \textbf{Parameter} & \textbf{Description} \\
    \hline
    \texttt{ arg } & The array you want to get the last index of.\\
    \hline
  \end{tabular}
  \caption{Parameters that must be passed to \texttt{Length}, \texttt{High}, and \texttt{Low}}
  \label{tbl:pas-array-fns}
\end{table}

\passection{\pascode{plst:array_fn_test}{}{code/pascal/array/TestArrayFunctions.pas}}

% section pascal_array_functions (end)
\clearpage
\subsection{Pascal Array Parameters} % (fold)
\label{sub:pas_array_parameters}

Pascal allows arrays to be passed as parameters to functions and procedures. Like other parameters these can be passed by value or by reference. 

\passyntax{psynt:array-params}{array parameters}{arrays/fn-array-param}

\begin{figure}[p]
  \passection{\pascode{plst:test-array-passing}{Code illustrating array passing in Pascal}{code/pascal/array/TestArrayPassing.pas}}
\end{figure}

\mynote{
\begin{itemize}
  \item This syntax shows you how to code \nameref{sub:pass_by_reference} into your functions and procedures in Pascal.
  \item See \lref{plst:test-array-passing} for example code.
  \item When declaring array parameters you do not specify the size, this allows the parameter to accept arrays of any size.
\end{itemize}
}

% subsection c_fn_with_array (end)
\clearpage
\subsection{Pascal String} % (fold)
\label{sub:pas_string}

Pascal has built in support for strings. Behind the scenes Pascal uses an array of characters to store the text for each string. The first element of this array stores an integer that indicates the size of the string, and this is followed by the text characters.

\begin{table}[h]
\begin{minipage}{\textwidth}
  \centering
\begin{tabular}{|l|c|c|c|c|c|}
\hline
Characters: & & F & r & e & d  \\
\hline
Bytes Values\footnote{Byte values are shown as decimal.}: & \texttt{4} & \texttt{70} & \texttt{114} & \texttt{101} & \texttt{100} \\
\hline
\end{tabular}
\caption{The characters and byte values for the string containing the text `Fred' in Pascal}
\label{tbl:pas-string-fred}
\end{minipage}
\end{table}

\passection{\pascode{plst:test-strings}{Code illustrating working with strings in Pascal}{code/pascal/array/TestString.pas}}

\mynote{
\begin{itemize}
  \item You can access individual characters in a string using array notation, the first text character is at index 1.
  \item Pascal strings can be concatenated used \texttt{+}, for example `Hello' + ` World'.
\end{itemize}
}

% subsection pas_string (end)

% section more_data_types_in_pascal (end)


% ========================
% = Understanding Arrays =
% ========================

\clearpage
\def\pageLang{none}
\section{Understanding Arrays} % (fold)
\label{sec:understanding_arrays}

\nameref{sub:array}s offer a means of storing and working with a list of values in your code. Each array has a number of elements, each of which has a value, and can be accessed using an index. Together with the \nameref{sub:for_loop}, arrays provide a means of managing multiple values in your code. The following illustrations show how these work in the computer, and should help you better understand how arrays can be used within your code.

\subsection{Understanding Populate Array} % (fold)
\label{sub:understanding_populate_array}

\sref{sub:designing_statistics_calculator} \nameref{sub:designing_statistics_calculator} outlined the pseudocode and flowcharts for a small statistics programs. This included a number of functions and procedures that helped divide the Program's code into smaller units of work. One of these procedures was \texttt{Populate Array}, discussed in \sref{ssub:populating_the_array} \nameref{ssub:populating_the_array}. This procedure is responsible for reading values from the user and using these to populate the program's array, and the flowchart for this logic is shown in \fref{fig:populate-array-flow-understanding}.

\begin{figure}[htbp]
   \centering
   \includegraphics[width=0.5\textwidth]{./topics/arrays/diagrams/PopulateArray} 
   \caption{Flowchart showing the process for \texttt{Populate Array}, from \fref{fig:populate-array-flow}}
   \label{fig:populate-array-flow-understanding}
\end{figure}

The following illustrations will show this code running to populate an array that contains three values. This will show how the array is passed by reference, and how the for loop works together with the array to populate all elements.

\clearpage
\subsubsection{Main starts, and the array is allocated space} % (fold)
\label{ssub:main_starts_and_the_array_is_allocated_space}

All local variables are allocated space on the Stack when the function or procedure they are declared in is called. In this example the \texttt{Main} procedure is executed and space is allocated for its \texttt{my\_data} variable. This variable is an \nameref{sub:array} that is used to store three \texttt{double} values. When \texttt{Main} is loaded onto the Stack there is space allocated for three \texttt{double} values associated with the \texttt{my\_data} variable.

\begin{figure}[htbp]
   \centering
   \includegraphics[width=\textwidth]{./topics/arrays/images/PopulateArray1} 
   \caption{When the program starts \texttt{Main} allocates space for its local variables, including the array}
   \label{fig:populate-array-vis-1}
\end{figure}

\mynote{
\begin{itemize}
  \item In \fref{fig:populate-array-vis-1} the indicated areas show the following:
  \begin{enumerate}
    \item The Program starts and \texttt{Main} is loaded onto the stack, allocating space for its local variables.
    \item The \texttt{my\_data} array is allocated space to store its values.
  \end{enumerate}
  \medskip
  \item Notice that the three values in the array are allocated next to each other.
  \item The indexes can be used to access the array's elements. The index value determines the number of elements that must be skipped to find where the value is stored.
\end{itemize}
}
% subsubsection main_starts_and_the_array_is_allocated_space (end)

\clearpage
\subsubsection{Populate array is called, and a reference to \texttt{my{\textunderscore}data} passed in} % (fold)
\label{ssub:populate_array_is_called_and_a_reference_to_my_data_passed_in}

\texttt{Populate Array} is called as the first step in \texttt{Main}. This is passed the \texttt{my\_data} variable (pass by reference), rather than being passed the values from within that variable. This gives the \texttt{data} parameter access to the memory where \texttt{my\_data} is stored.

\begin{figure}[htbp]
   \centering
   \includegraphics[width=\textwidth]{./topics/arrays/images/PopulateArray2} 
   \caption{Populate array is called, and a reference to the \texttt{my\_data} array is pass to its \texttt{data} parameter}
   \label{fig:populate-array-vis-2}
\end{figure}

\mynote{
\begin{itemize}
  \item In \fref{fig:populate-array-vis-2} the indicated areas show the following:
  \begin{enumerate}
    \item When \texttt{Populate Array} is called it is loaded onto the Stack. Its \texttt{data} parameter receives the address of \texttt{my\_data} from \texttt{Main}. In C the value 3 would also be passed to the \texttt{size} parameter, as C does not keep track of the length of the array for you. At the same time space for \texttt{Populate Array}'s local variables \texttt{i} and \texttt{prompt} are allocated on the Stack.
    \item Notice that in \texttt{Populate Array} the \texttt{data} parameter only stores the address of the array, as it is passed by reference. This saves time and space, and is needed in this case as the procedure wants to store data into the variable passed to this parameter.
    \item The \texttt{prompt} local variable is also an array. It is allocated spaces on the stack as is done for all local variables. 
  \end{enumerate}
  \medskip
  \item Arrays are allocated a contiguous area of memory to store its elements.
  \item A String is an array of characters.
\end{itemize}
}

% subsubsection populate_array_is_called_and_a_reference_to_my\_data_passed_in (end)

\clearpage
\subsubsection{Step 1 of \texttt{Populate Array} is run} % (fold)
\label{ssub:step_1_of_populate_array_is_run}

Step 1 of \texttt{Populate Array} initialises the for loop's control variable (\texttt{i} in this case). This variable keeps track of the times the loop body has executed, and can be used to get the \emph{current} value from the array.

\begin{figure}[htbp]
   \centering
   \includegraphics[width=\textwidth]{./topics/arrays/images/PopulateArray3} 
   \caption{Step 1 of \texttt{Populate Array} is called, and the for loop sets i to the lowest index of the \texttt{data} array}
   \label{fig:populate-array-vis-3}
\end{figure}

\mynote{
\begin{itemize}
  \item In \fref{fig:populate-array-vis-3} the indicated areas show the following:
  \begin{enumerate}
    \item Step 1 of \texttt{Populate Array} is a for loop. The first action of the for loop is to initialise the value of \texttt{i} to the \emph{lowest index value} of the \texttt{data} array. The lowest index is 0, so \texttt{i} is assigned the value 0.
    \item Next the loop checks its condition, it is loop from 0 to 2, and has not passed 2 so the body of the loop will be executed. This will be checked again when the for loop ends.
  \end{enumerate}
  \medskip
  \item The first action of a for loop is to initialise the value of the control variable.
  \item When processing each element of an array the for loop should initialise the control variable to 0, the first index of the array.
\end{itemize}
}

% subsubsection step_1_of_populate_array_is_run (end)

\clearpage
\subsubsection{Step 2 constructs the prompt to be shown to the user} % (fold)
\label{ssub:step_2_constructs_the_prompt_to_be_shown_to_the_user}

The user needs to be told what to enter. The \texttt{prompt} is a string that will contain this message so that it can be passed to \texttt{Read Double}. The value for the prompt will use the loop's control variable (the counter) so that the user known which value they are up to.

\begin{figure}[htbp]
   \centering
   \includegraphics[width=\textwidth]{./topics/arrays/images/PopulateArray4} 
   \caption{Step 2 builds the prompt \texttt{Enter value 1: } which will be shown to the user}
   \label{fig:populate-array-vis-4}
\end{figure}

\mynote{
\begin{itemize}
  \item In \fref{fig:populate-array-vis-4} the indicated areas show the following:
  \begin{enumerate}
    \item Step 2 of \texttt{Populate Array} builds the prompt by concatenating `\texttt{Enter Value}' with \texttt{i + 1} and `\texttt{: }'.
    \item To achieve this \texttt{i + 1} must be converted from an Integer to a String.
    \item The result is stored in the \texttt{prompt} variable.
  \end{enumerate}
  \medskip
  \item This action will be performed each time the loop executes. In this case the value stored in \texttt{prompt} will be `\texttt{Enter value  1: }'.
  \item The small square shown at the end of the \texttt{prompt} represents the overhead. In C this is the \emph{sentinel} value, in Pascal it is the length\footnote{Pascal actually stores the length at the start of the string, but the idea is the same.} of the array.
\end{itemize}
}


% subsubsection step_2_constructs_the_prompt_to_be_shown_to_the_user (end)

\clearpage
\subsubsection{Step 3 reads a value from the user and stores it in the array at index 0} % (fold)
\label{ssub:step_3_reads_a_value_from_the_user_and_stores_it_in_the_array_at_index_0}

The next step calls the \texttt{Read Double} function. This is responsible for reading a value from the user, and returning it to the caller. The value returned is then stored in an element of the array. The \texttt{i} variable is read to determine the position where this value should be stored. This means that you can think of \texttt{i} as referring to the \emph{current} element of the array.

\begin{figure}[htbp]
   \centering
   \includegraphics[width=\textwidth]{./topics/arrays/images/PopulateArray5} 
   \caption{Step 3 reads a \texttt{double} value from the user and stores it in \texttt{data[i]}}
   \label{fig:populate-array-vis-5}
\end{figure}

\mynote{
\begin{itemize}
  \item In \fref{fig:populate-array-vis-5} the indicated areas show the following:
  \begin{enumerate}
    \item Step 3 of \texttt{Populate Array} calls \texttt{Read Double}, passing across the prompt to be shown to the user.
    \item Within \texttt{Read Double} the value is read from the Terminal. This includes some validation to make sure that the value entered is a number.
    \item To determine where the value is stored the computer needs to evaluate the expression used to index the array. In this case that is the value of the \texttt{i} variable.
    \item The \texttt{data} reference is followed, and \texttt{0} elements skipped, to find the location where the data should be stored. 
  \end{enumerate}
  \medskip
  \item Notice that this has stored the value in \texttt{my\_data}, as \texttt{data} is passed by reference.
\end{itemize}
}

% subsubsection step_3_reads_a_value_from_the_user_and_stores_it_in_the_array_at_index_0 (end)

\clearpage
\subsubsection{Control returns to Step 1 as the loop body has ended} % (fold)
\label{ssub:control_returns_to_step_1_as_the_loop_body_has_ended}

At the end of the loop body the for loop performs two actions. It has finished the first pass through the loop, so its control variable (the counter) needs to be incremented to 1. Then it needs to jump back to check its condition. This will determine if the loop's body is executed again or skipped. In this case \texttt{i} is still in the defined range so the loop's body is run again.

\begin{figure}[htbp]
   \centering
   \includegraphics[width=\textwidth]{./topics/arrays/images/PopulateArray6} 
   \caption{At the end of the loop body \emph{i} is incremented and control jumps back to check the loop's condition}
   \label{fig:populate-array-vis-6}
\end{figure}

\mynote{
\begin{itemize}
  \item In \fref{fig:populate-array-vis-6} the indicated areas show the following:
  \begin{enumerate}
    \item The end of the loop body indicates that two things needs to occur. Firstly the value of \texttt{i} needs to be incremented, and then control needs to jump back to check the condition of the loop (step 1).
    \item The loop's condition checks if \texttt{i} is still in range (\texttt{i} is in the range \texttt{0..2}, coded as \texttt{i < size} in C). As it is still in range the loop's body will execute again.
  \end{enumerate}
  \medskip
\end{itemize}
}

% subsubsection control_returns_to_step_1_as_the_loop_body_has_ended (end)

\clearpage
\subsubsection{Second prompt is built asking the user to enter value 2} % (fold)
\label{ssub:second_prompt_is_built_asking_the_user_to_enter_value_2}

Back at step 2 again, the prompt needs to be recreated. This time its message will be `\texttt{Enter value 2: }'. The process to create this is the same, with the value of \texttt{i + 1} being converted to a String, and the three parts concatenated together and stored in \texttt{prompt}. This overrides the details in the existing array, reusing the same memory to store these values.

\begin{figure}[htbp]
   \centering
   \includegraphics[width=\textwidth]{./topics/arrays/images/PopulateArray7} 
   \caption{Step 2 builds the prompt \texttt{Enter value 2: } which will be shown to the user}
   \label{fig:populate-array-vis-7}
\end{figure}

\mynote{
\begin{itemize}
  \item In \fref{fig:populate-array-vis-7} the indicated areas show the following:
  \begin{enumerate}
    \item Step 2 of \texttt{Populate Array} builds the prompt by concatenating `\texttt{Enter Value}' with \texttt{i + 1} and `\texttt{: }'.
    \item To achieve this \texttt{i + 1} must be converted from an Integer to a String.
    \item The result is stored in the \texttt{prompt} variable.
  \end{enumerate}
  \medskip
  \item Notice that this is using the same location to store its data.
  \item The small square shown at the end of the \texttt{prompt} represents the overhead. In C this is the \emph{sentinel} value, in Pascal it is the length\footnote{Pascal actually stores the length at the start of the string, but the idea is the same.} of the array.
\end{itemize}
}

% subsubsection second_prompt_is_built_asking_the_user_to_enter_value_2 (end)

\clearpage
\subsubsection{\texttt{Populate Array} stores the second value read into \texttt{data[1]}} % (fold)
\label{ssub:populate_array_stores_the_second_value_read_into_data[1]}

Step 3 uses \texttt{Read Double} again to get the value to store in the second element of the array. To find where this should be stored the computer calculates the value of the index, reading this from the \texttt{i} variable. The value returned from \texttt{Read Double} is then stored in the array referred to by \texttt{data}, at index \texttt{1}.

\begin{figure}[htbp]
   \centering
   \includegraphics[width=\textwidth]{./topics/arrays/images/PopulateArray8} 
   \caption{The second value read is stored in \texttt{data[1]}}
   \label{fig:populate-array-vis-8}
\end{figure}

\mynote{
\begin{itemize}
  \item In \fref{fig:populate-array-vis-8} the indicated areas show the following:
  \begin{enumerate}
    \item Step 3 of \texttt{Populate Array} calls \texttt{Read Double}, passing across the prompt to be shown to the user.
    \item Within \texttt{Read Double} the value is read from the Terminal. This includes some validation to make sure that the value entered is a number.
    \item The value of the index is now \texttt{1}, as read from variable \texttt{i}.
    \item The \texttt{data} reference is followed, and \texttt{1} element is skipped, to find the location where the data should be stored.
  \end{enumerate}
  \medskip
\end{itemize}
}

% subsubsection populate_array_stores_the_second_value_read_into_data[1] (end)

\clearpage
\subsubsection{\texttt{i} is incremented again, and control returns to Step 1 to determine if loop runs again} % (fold)
\label{ssub:i_is_incremented_again_and_control_returns_to_step_1_to_determine_if_loop_runs_again}

The end of the for loop's body has been reached again, so it performs two actions: it increments its control variable (\texttt{i} in this case) and jumps back to check its condition (step 1). The condition then determines if the loop's body is to be executed again or skipped. In this case \texttt{i} is still in the defined range so the loop's body is run a third time.

\begin{figure}[htbp]
   \centering
   \includegraphics[width=0.98\textwidth]{./topics/arrays/images/PopulateArray9} 
   \caption{At the end of the loop body \emph{i} is incremented and control jumps back to check the loop's condition}
   \label{fig:populate-array-vis-9}
\end{figure}

\mynote{
\begin{itemize}
  \item In \fref{fig:populate-array-vis-9} the indicated areas show the following:
  \begin{enumerate}
    \item The end of the loop body indicates that two things needs to occur: the value of \texttt{i} is incremented, and control jump back to check the condition of the loop.
    \item The loop's condition checks if \texttt{i} is still in range (\texttt{i} is in the range \texttt{0..2}, coded as \texttt{i < size} in C). As it is still in range the loop's body is execute again.
  \end{enumerate}
  \medskip
  \item These same actions always occur at the end of the for loop. It increments its control variable, and jumps back to check its condition.
\end{itemize}
}

\csection{The C for loop can be used for more than just counting. At the end of the loop the instructions from the \nameref{sub:c_for_loop}'s \emph{increment} occur, typically something like \texttt{i++}.}


% subsubsection i_is_incremented_again_and_control_returns_to_step_1_to_determine_if_loop_runs_again (end)

\clearpage
\subsubsection{Third prompt is built asking the user to enter value 3} % (fold)
\label{ssub:third_prompt_is_built_asking_the_user_to_enter_value_3}

Back at step 2 for the third time. This step recreates the prompt, this time with the message `\texttt{Enter value 3: }'. The process to create this is the same as before, with the value of \texttt{i + 1} being converted to a String, and the three parts concatenated together and stored in \texttt{prompt}. Remember that this overrides the data currently in the \texttt{prompt} array.

\begin{figure}[htbp]
   \centering
   \includegraphics[width=\textwidth]{./topics/arrays/images/PopulateArray10} 
   \caption{Step 2 builds the prompt \texttt{Enter value 3: } which will be shown to the user}
   \label{fig:populate-array-vis-10}
\end{figure}

\mynote{
\begin{itemize}
  \item In \fref{fig:populate-array-vis-10} the indicated areas show the following:
  \begin{enumerate}
    \item Step 2 of \texttt{Populate Array} builds the prompt by concatenating `\texttt{Enter Value}' with \texttt{i + 1} and `\texttt{: }'.
    \item To achieve this \texttt{i + 1} must be converted from an Integer to a String.
    \item The result is stored in the \texttt{prompt} variable.
  \end{enumerate}
  \medskip
\end{itemize}
}

% subsubsection third_prompt_is_built_asking_the_user_to_enter_value_3 (end)

\clearpage
\subsubsection{\texttt{Populate Array} stores the third value read into \texttt{data[2]}} % (fold)
\label{ssub:populate_array_stores_the_third_value_read_into_data[2]}

Once again, step 3 uses \texttt{Read Double} get the value to store in the array. In this case \texttt{i} indicates that this should be stored in the element at index 2 (skipping 2 elements, so storing the value in the third).

\begin{figure}[htbp]
   \centering
   \includegraphics[width=\textwidth]{./topics/arrays/images/PopulateArray11} 
   \caption{The second value read is stored in \texttt{data[1]}}
   \label{fig:populate-array-vis-11}
\end{figure}

\mynote{
\begin{itemize}
  \item In \fref{fig:populate-array-vis-11} the indicated areas show the following:
  \begin{enumerate}
    \item Step 3 of \texttt{Populate Array} calls \texttt{Read Double} again.
    \item Within \texttt{Read Double} the value is read from the Terminal.
    \item The value of the index is now \texttt{2}, as read from variable \texttt{i}.
    \item The \texttt{data} reference is followed, and \texttt{2} elements are skipped to find the location where the data should be stored.
  \end{enumerate}
  \medskip
  \item Notice how over these three loops \texttt{i} has been used to determine which element is the \emph{current element}. The loop's processing is the same, but changing \texttt{i} means that this action is applied to each element in the array one at a time.
  \item The loop's body determines the action to apply to the \texttt{$i^{th}$} element of the array.
  \item The for loop then updates \texttt{i} so that the body is applied to \emph{each} element of the array.
\end{itemize}
}

% subsubsection populate_array_stores_the_third_value_read_into_data[2] (end)

\clearpage
\subsubsection{\texttt{i} is incremented again, and control jumps back to check the condition a fourth time} % (fold)
\label{ssub:i_is_incremented_again_and_control_jumps_back_to_check_the_condition_a_fourth_time}

The end of the for loop's body has been reached again, so it performs two actions: it increments its control variable (\texttt{i} in this case) and jumps back to check its condition (step 1). This time the value of \texttt{i} is outside the range of the indexes for \texttt{data} (\texttt{0..2}, it is no longer less than 3). This means that the loop's body should not be run again, and control will jump past the body to the next step. As there are no more steps in \texttt{Populate Array} it will end.

\begin{figure}[htbp]
   \centering
   \includegraphics[width=\textwidth]{./topics/arrays/images/PopulateArray12} 
   \caption{At the end of the loop body \emph{i} is incremented and control jumps back to check the loop's condition}
   \label{fig:populate-array-vis-12}
\end{figure}

\mynote{
\begin{itemize}
  \item In \fref{fig:populate-array-vis-12} the indicated areas show the following:
  \begin{enumerate}
    \item The end of the loop body causes the value of \texttt{i} to be incremented, and control jump back to check the condition of the loop.
    \item The loop's condition checks if \texttt{i} is still in range (\texttt{i} is in the range \texttt{0..2}, coded as \texttt{i < size} in C).
    \item \texttt{i} is \textbf{not} in the range \texttt{0..2}, so the loop body is skipped. As there are no more steps in this procedure it ends.
  \end{enumerate}
  \medskip
  \item The condition in the for loop is responsible for determining if the loop runs again. This time the condition indicates that the loop is not to be run again, so control will jump past the end of the loop.
  \item The for loop works like a while loop, with additional logic to keep a counter.
\end{itemize}
}

% subsubsection i_is_incremented_again_and_control_jumps_back_to_check_the_condition_a_fourth_time (end)

\clearpage
\subsubsection{\texttt{Populate Array} ends, and has populated \texttt{Main}'s \texttt{my{\textunderscore}data} array} % (fold)
\label{ssub:populate_array_ends_and_has_populated_main_s_my_data_array}

When \texttt{Populate Array} ends its space on the stack is released so that it can be used again, and control returns to \texttt{Main}. \texttt{Populate Array} was responsible for reading values from the user and storing these in the array passed to it, and if you look at \fref{fig:populate-array-vis-13} you can see that this has been achieved.

The instructions in \texttt{Populate Array} commanded the computer to read a value from the user and store it in the current element (the $i^{th}$ element) of the array. These actions were then repeated by the \nameref{sub:for_loop} \emph{for each} index of the array. Together the for loop and its body allow you to define actions that must be performed on all elements in an array.

\begin{figure}[htbp]
   \centering
   \includegraphics[width=\textwidth]{./topics/arrays/images/PopulateArray13} 
   \caption{Control returns to \texttt{Main}, and its \texttt{my\_data} array has been populated}
   \label{fig:populate-array-vis-13}
\end{figure}

\mynote{
\begin{itemize}
  \item In \fref{fig:populate-array-vis-13} the indicated areas show the following:
  \begin{enumerate}
    \item The last step in \texttt{Populate Array} has been completed, so it ends and control returns to \texttt{Main}.
    \item The values in \texttt{my\_data} have been updated. Passing this array by reference to \texttt{Populate Array} allowed it to make changes to this array's values.
  \end{enumerate}
  \medskip
  \item Notice that the data associated with the \texttt{Populate Array} procedure has been released, including the data used for the \texttt{prompt} array.
  \item \texttt{Populate Array} is responsible for filling the array passed to it with values from the user, and that is what it has done.
  
\end{itemize}
}

% subsubsection populate_array_ends_and_has_populated_main_s_my_data_array (end)
% subsection understanding_populate_array (end) 

\clearpage
\subsection{Understanding \texttt{Sum}} % (fold)
\label{sub:understanding_sum}

\fref{fig:sum-understanding} shows the flowchart of the \texttt{Sum} function from the Statistics Calculator program. This algorithm was developed in the \sref{ssub:calculating_sum}, and its Pseudocode is shown in Listing \ref{plst:sum}. The \texttt{Sum} function is responsible for calculating the sum of all of the values in the array passed to it. This is achieved by having a \texttt{total} variable that is initialised to 0, and then has the value of \emph{each element} from the array added to it.

\begin{figure}[htbp]
   \centering
   \includegraphics[width=0.5\textwidth]{./topics/arrays/diagrams/SumFlow} 
   \caption{Flowchart showing the process for \texttt{Sum}}
   \label{fig:sum-understanding}
\end{figure}

The following code will show how this function is executed on an array with three values in it. This will continue the execution from \sref{sub:understanding_populate_array} \nameref{sub:understanding_populate_array}, though the same process would occur for any array values.

\mynote{
The most important thing to pay attention to in this illustration is the interactions between the \nameref{sub:for_loop} and the \nameref{sub:array}. Make sure you can see how this combination allows you to specify the actions to be performed on an element (in the loop's body), and then to have this run for each element of the array (controlled by the loop condition).
}

\clearpage
\subsubsection{\texttt{Sum} is called, and passed the \texttt{my data} array} % (fold)
\label{ssub:sum_is_called_and_passed_the_values_in_my_data}

When \texttt{Sum} is called it is passed the array to read its values from. This is passed in the same way as was done in \texttt{Populate Array}. The difference here is that this is passed using a \texttt{const} reference, to indicate that \texttt{Sum} is not allowed to change the data in the array. This means that \texttt{Sum} will not compile if you update values in this array, and provide a guarantee to the caller that their data will not change when given to the \texttt{Sum} function. 

\begin{figure}[htbp]
   \centering
   \includegraphics[width=\textwidth]{./topics/arrays/images/Sum1} 
   \caption{\texttt{Sum} is called, and it is passed the array to get its values from}
   \label{fig:sum-array-vis-1}
\end{figure}

\mynote{
\begin{itemize}
  \item In \fref{fig:sum-array-vis-1} the indicated areas show the following:
  \begin{enumerate}
    \item Step 1 of \texttt{Main} has populated the array with values.
    \item Step 2 calls the \texttt{Sum} function, passing in the \texttt{my\_data} array.
    \item When sum starts it is allocated space on the heap. Its \texttt{data} parameter is passed the address of \texttt{my\_data}. In C the \texttt{size} parameter will be passed the value 3, this isn't needed in Pascal as the language takes care of these details for you. Space is also allocated for the local variables \texttt{i} and \texttt{total}.
  \end{enumerate}
  \medskip
  \item Passing \texttt{my\_data} by reference means that sum gets a reference that \emph{points} to the start of the array.
  \item \texttt{Sum} is responsible for calculating the sum value of the values in the array.
  
\end{itemize}
}
% subsubsection sum_is_called_and_passed_the_values_in_my_data (end)

\clearpage
\subsubsection{\texttt{total} is initialised to 0} % (fold)
\label{ssub:total_is_initialised_to_0}

The first action in \texttt{Sum} is to set the value of \texttt{total} to 0. \texttt{total} will be used to store the running total of the array, and it must start at 0. Remember that the space on the stack was used before, and therefore these variables get seemingly random values initially. It is important to remember to always initialise the variables you are using.

\begin{figure}[htbp]
   \centering
   \includegraphics[width=\textwidth]{./topics/arrays/images/Sum2} 
   \caption{\texttt{Total} is initialised, having its value set to 0}
   \label{fig:sum-array-vis-2}
\end{figure}

\mynote{
\begin{itemize}
  \item In \fref{fig:sum-array-vis-2} the indicated areas show the following:
  \begin{enumerate}
    \item The \texttt{total} variable is assigned the value 0.
  \end{enumerate}
  \medskip
  \item Local variables do not get a default value when the Function or Procedure starts. You need to make sure you initialise these to appropriate values before you use them.
  \item \texttt{total} will keep a running total of the elements in the array, it must start at the value 0.
\end{itemize}
}

% subsubsection total_is_initialised_to_0 (end)

\clearpage
\subsubsection{For loop initialises \texttt{i}} % (fold)
\label{ssub:for_loop_initialises_i}

Step 2 of \texttt{Sum} starts the for loop that will iterate over the elements of the \texttt{data} array. The for loop's control variable, \texttt{i}, is set to the first index of the array and the condition checks if the loop's body should run. As \texttt{i} is in the range \texttt{0..2}, it is less than 3, control will jump into the loop's body making step 3 the next instruction.

\begin{figure}[htbp]
   \centering
   \includegraphics[width=\textwidth]{./topics/arrays/images/Sum3} 
   \caption{\texttt{i} is initialised by the for loop, and control jumps to the loop's body}
   \label{fig:sum-array-vis-3}
\end{figure}

\mynote{
\begin{itemize}
  \item In \fref{fig:sum-array-vis-3} the indicated areas show the following:
  \begin{enumerate}
    \item The for loop initialises its control variable \texttt{i}, assigning it the value 0.
    \item The condition of the loop is checked to see if the body should execute. As \texttt{i} is still in the range \texttt{0..2}, it is less than 3, the loop's body will execute.
  \end{enumerate}
  \medskip
  \item The first action of a for loop is always to initialise its control variable. After this it checks if the body of the loop should run.
\end{itemize}
}

% subsubsection for_loop_initialises_i (end)

\clearpage
\subsubsection{\texttt{total} is increased by the value in \texttt{data[0]}} % (fold)
\label{ssub:total_is_incremented_by_the_value_in_data[0]}

The body of the for loop reads the \emph{current} value from the \texttt{data} array, \texttt{data[i]}. As \texttt{i} is currently 0 this reads the first element of the array. This reads the value 10.0, which is added to the current \texttt{total}, 0. The resulting value is then stored back into the \texttt{total} variable, giving it the value 10.0, calculated from the expression \texttt{total + data[0]}, which is \texttt{0 + 10.0}.

\begin{figure}[htbp]
   \centering
   \includegraphics[width=\textwidth]{./topics/arrays/images/Sum4} 
   \caption{Total is increased by the value in \texttt{data[0]}}
   \label{fig:sum-array-vis-4}
\end{figure}

\mynote{
\begin{itemize}
  \item In \fref{fig:sum-array-vis-4} the indicated areas show the following:
  \begin{enumerate}
    \item The value of \texttt{data[i]} must be read in this expression. At this point \texttt{i} is 0, so \texttt{data[0]} must be read.
    \item \texttt{data[0]} is found in the array referred to by \texttt{data}, after skipping \texttt{0} elements. This reads the value of the first array element.
    \item The expression evaluates \texttt{total\footnote{Remember that at this point \texttt{total} has not been changed, so it has the value \texttt{0.0} as shown in \fref{fig:sum-array-vis-3}.} + data[i]}, giving \texttt{total + data[0]}, which is \texttt{0 + 10.0}, with the final value being \texttt{10.0}.
    \item The value 10.0, calculated above, is then stored in \texttt{total}.
  \end{enumerate}
  \medskip
  \item The body of the for loop provides the instructions that are run for each element of the array.
\end{itemize}
}

% subsubsection total_is_incremented_by_the_value_in_data[0] (end)

\clearpage
\subsubsection{For loop increases the value of i and runs the loop body a second time} % (fold)
\label{ssub:for_loop_increases_the_value_of_i_and_runs_the_loop_body_a_second_time}

At the end of the for loop it increments the value of its control variable, assigning \texttt{i} the value 1, and then jumps back to check its condition. As \texttt{i} is still in the range 0..2 the loop body will be run again, making step 3 the next action. 

\begin{figure}[htbp]
   \centering
   \includegraphics[width=0.95\textwidth]{./topics/arrays/images/Sum5} 
   \caption{At the end of the loop body \texttt{i} is incremented, and control jumps back to check the condition}
   \label{fig:sum-array-vis-5}
\end{figure}

\mynote{
\begin{itemize}
  \item In \fref{fig:sum-array-vis-5} the indicated areas show the following:
  \begin{enumerate}
    \item The end of the loop has been reached. The for loop increments its control variable, and jumps back to check its condition.
    \item In this case \texttt{i} is the control variable, so its value is increased to 1.
    \item The condition of the loop is checked to see if the body should execute. As \texttt{i} is still in the range \texttt{0..2} (it is less than 3) the loop's body will execute, making step 3 the next action.
  \end{enumerate}
  \medskip
  \item The end of each for loop always performs these steps. Increasing the value of its control variable, and jumping back to check its condition.
\end{itemize}
}

\csection{Remember C can do more than just counting in the for loop, though this is its main use.}


% subsubsection for_loop_increases_the_value_of_i_and_runs_the_loop_body_a_second_time (end)

\clearpage
\subsubsection{The value of \texttt{total} is increased by the value in \texttt{data[1]}} % (fold)
\label{ssub:the_value_of_total_is_increased_by_the_value_in_data[1]}

The body of the for loop reads the \emph{current} value from the \texttt{data} array, \texttt{data[i]}. Now \texttt{i} has the value 1 it reads the second element of the array. This reads the value -5, which is added to the current \texttt{total}, 10.0. The resulting value is then stored back into the \texttt{total} variable, giving it the value 5.0, calculated from the expression \texttt{total + data[1]}, which is \texttt{10.0 + -5.0}.

\begin{figure}[htbp]
   \centering
   \includegraphics[width=\textwidth]{./topics/arrays/images/Sum6} 
   \caption{Total is increased by the value in \texttt{data[1]}}
   \label{fig:sum-array-vis-6}
\end{figure}

\mynote{
\begin{itemize}
  \item In \fref{fig:sum-array-vis-6} the indicated areas show the following:
  \begin{enumerate}
    \item The value of \texttt{data[i]} must be read in this expression. At this point \texttt{i} is 1, so \texttt{data[1]} must be read.
    \item \texttt{data[1]} is found in the array referred to by \texttt{data}, after skipping \texttt{1} element. This reads the value of the second array element.
    \item The expression evaluates \texttt{total\footnote{As before, \texttt{total} has not been changed at this point, so it has the value \texttt{10.0} as shown in \fref{fig:sum-array-vis-5}.} + data[i]}, giving \texttt{total + data[1]}, which is \texttt{10.0 + -5.0}, with the final value being \texttt{5.0}.
    \item The value 5.0, calculated above, is then stored in \texttt{total}.
  \end{enumerate}
  \medskip
  \item Notice that this is performing the same task, just using the next element from the array.
\end{itemize}
}

% subsubsection the_value_of_total_is_increased_by_the_value_in_data[1] (end)

\clearpage
\subsubsection{For loop increases the value of i and runs the loop body a third time} % (fold)
\label{ssub:for_loop_increases_the_value_of_i_and_runs_the_loop_body_a_third_time}

The end of the for loop has been reached again, so it increments the value of its control variable, assigning \texttt{i} the value 2, and then jumps back to check its condition. As \texttt{i} is still in the range 0..2 the loop body will run a third time, making step 3 the next action. 

\begin{figure}[htbp]
   \centering
   \includegraphics[width=0.95\textwidth]{./topics/arrays/images/Sum7} 
   \caption{At the end of the loop body \texttt{i} is incremented, and control jumps back to check the condition}
   \label{fig:sum-array-vis-7}
\end{figure}

\mynote{
\begin{itemize}
  \item In \fref{fig:sum-array-vis-7} the indicated areas show the following:
  \begin{enumerate}
    \item The end of the loop has been reached. The for loop increments its control variable, and jumps back to check its condition.
    \item In this case \texttt{i} is the control variable, so its value is increased to 2.
    \item The condition of the loop is checked to see if the body should execute. As \texttt{i} is still in the range \texttt{0..2} (it is less than 3) the loop's body will execute, making step 3 the next action.
  \end{enumerate}
  \medskip
  \item The end of each for loop always performs these steps. Increasing the value of its control variable, and jumping back to check its condition.
\end{itemize}
}

% subsubsection for_loop_increases_the_value_of_i_and_runs_the_loop_body_a_third_time (end)

\clearpage
\subsubsection{The value of \texttt{total} is increased by the value in \texttt{data[2]}} % (fold)
\label{ssub:the_value_of_total_is_increased_by_the_value_in_data[2]}

The body of the for loop reads the \emph{current} value from the \texttt{data} array, \texttt{data[i]}. Now \texttt{i} has the value 2 it reads the second element of the array. This reads the value 17.21, which is added to the current \texttt{total}, 5.0. The resulting value is then stored back into the \texttt{total} variable, giving it the value 22.21, calculated from the expression \texttt{total + data[2]}, which is \texttt{5.0 + 17.21}.

\begin{figure}[htbp]
   \centering
   \includegraphics[width=\textwidth]{./topics/arrays/images/Sum8} 
   \caption{Total is increased by the value in \texttt{data[1]}}
   \label{fig:sum-array-vis-8}
\end{figure}

\mynote{
\begin{itemize}
  \item In \fref{fig:sum-array-vis-8} the indicated areas show the following:
  \begin{enumerate}
    \item The value of \texttt{data[i]} must be read in this expression. At this point \texttt{i} is 2, so \texttt{data[2]} must be read.
    \item \texttt{data[2]} is found in the array referred to by \texttt{data}, after skipping \texttt{2} elements. This reads the value of the third array element.
    \item The expression evaluates \texttt{total\footnote{As before, \texttt{total} has not been changed at this point, so it has the value \texttt{5.0} as shown in \fref{fig:sum-array-vis-7}.} + data[i]}, giving \texttt{total + data[2]}, which is \texttt{5.0 + 17.21}, with the final value being \texttt{22.21}.
    \item The value 5.0, calculated above, is then stored in \texttt{total}.
  \end{enumerate}
  \medskip
  \item Notice that this is performing the same task, just using the next element from the array.
\end{itemize}
}

% subsubsection the_value_of_total_is_increased_by_the_value_in_data[2] (end)

\clearpage
\subsubsection{For loop increases the value of \texttt{i}, and this time the loop finishes} % (fold)
\label{ssub:for_loop_increases_the_value_of_i_and_this_time_the_loop_finishes}

The end of the for loop has been reached again, so it increments the value of its control variable, assigning \texttt{i} the value 3, and then jumps back to check its condition. This time \texttt{i} is no longer in the range 0..2 (it is not less than 3), so the loop body will now be skipped, making step 4 the next action. 

\begin{figure}[htbp]
   \centering
   \includegraphics[width=0.95\textwidth]{./topics/arrays/images/Sum9} 
   \caption{At the end of the loop body \texttt{i} is incremented, and control jumps back to check the condition}
   \label{fig:sum-array-vis-9}
\end{figure}

\mynote{
\begin{itemize}
  \item In \fref{fig:sum-array-vis-9} the indicated areas show the following:
  \begin{enumerate}
    \item The end of the loop has been reached. The for loop increments its control variable, and jumps back to check its condition.
    \item In this case \texttt{i} is the control variable, so its value is increased to 3.
    \item The condition of the loop is checked to see if the body should execute. \texttt{i} is no longer in the range \texttt{0..2} (it is not less than 3), so the loop's body will be skipped, making step 4 the next action.
  \end{enumerate}
  \medskip
  \item The for works just like a while loop, checking its condition each loop and skipping the body when the condition is not met (when it is false).
\end{itemize}
}

% subsubsection for_loop_increases_the_value_of_i_and_this_time_the_loop_finishes (end)

\clearpage
\subsubsection{\texttt{Sum} function returns the \texttt{total} to the expression in \texttt{Main}} % (fold)
\label{ssub:sum_function_returns_the_total_to_the_expression_in_main}

The end of the for loop has been reached again, so it increments the value of its control variable, assigning \texttt{i} the value 3, and then jumps back to check its condition. This time \texttt{i} is no longer in the range 0..2 (it is not less than 3), so the loop body will now be skipped, making step 4 the next action. 

\begin{figure}[htbp]
   \centering
   \includegraphics[width=0.95\textwidth]{./topics/arrays/images/Sum10} 
   \caption{Step 4 indicates that the value in \texttt{total} is returned to \texttt{Main}}
   \label{fig:sum-array-vis-10}
\end{figure}

\mynote{
\begin{itemize}
  \item In \fref{fig:sum-array-vis-10} the indicated areas show the following:
  \begin{enumerate}
    \item Step 4 indicates that the value in the \texttt{total} variable should be returned to the caller.
    \item The result returned by the \texttt{Sum} function is used in the expression in \texttt{Main}.
    \item As this is the end of the function its space on the stack is released, allowing control to return to \texttt{Main}.
  \end{enumerate}
  \medskip
  \item The for works just like a while loop, checking its condition each loop and skipping the body when the condition is not met (when it is false).
\end{itemize}
}

% subsubsection sum_function_returns_the_total_to_the_expression_in_main (end)

\clearpage
\subsubsection{\texttt{Main} outputs the sum to the Terminal} % (fold)
\label{ssub:main_outputs_the_sum_to_the_terminal}

\texttt{Sum} returns its value to be used in step 2 of \texttt{Main}. This step outputs the value returned to the Terminal. So by the end of Step 2 in \texttt{Main} the sum has been calculated, and written to the Terminal.

\begin{figure}[htbp]
   \centering
   \includegraphics[width=0.95\textwidth]{./topics/arrays/images/Sum11} 
   \caption{Step 4 indicates that the value in \texttt{total} is returned to \texttt{Main}}
   \label{fig:sum-array-vis-11}
\end{figure}

\mynote{
\begin{itemize}
  \item In \fref{fig:sum-array-vis-11} the indicated areas show the following:
  \begin{enumerate}
    \item Back in \texttt{Main}, the sum is output to the Terminal.
    \item The value that is output is the value returned from the \texttt{Sum} function.
  \end{enumerate}
\end{itemize}
}

% subsubsection main_outputs_the_sum_to_the_terminal (end)

% subsection understanding_sum (end)


% section understanding_arrays (end)

% ============
% = Examples =
% ============
\clearpage
\section{Array Examples} % (fold)
\label{sec:array_examples}

\subsection{Bubble Game (start)} % (fold)
\label{sub:bubble_game_start_}

The bubble game has ten floating bubbles for the user to `pop'. Each bubble is a sprite, which gives it a location on the screen and a movement vector. When the program run the bubbles move in random directions, reappearing at a new random location if they go off the screen.

\begin{figure}[h]
   \centering
   \includegraphics[width=0.8\textwidth]{./topics/type-decl/examples/Bubbles.png} 
   \caption{Example execution of the Bubbles program}
   \label{fig:bubbles-img}
\end{figure}

\clearpage

\cppsection{\ccode{cpplst:bubble-game}{Starting code for a Bubble Game in C++, continues in \lref{cpplst:bubble-game1} }{topics/arrays/examples/bubble-game.c}}

\cppsection{\ccode{cpplst:bubble-game1}{Starting code for a Bubble Game in C++ (continued)}{topics/arrays/examples/bubble-game1.c}}

\passection{\pascode{plst:bubble-game}{Starting code for a Bubble Game in Pascal, continues in \lref{plst:bubble-game1} }{topics/arrays/examples/BubbleGame.pas}}

\passection{\pascode{plst:bubble-game1}{Starting code for a Bubble Game in Pascal (continued)}{topics/arrays/examples/BubbleGame1.pas}}


% subsection bubble_game_start_ (end)

% section array_examples (end)


% =============
% = Exercises =
% =============
\clearpage
\section{Array Exercises} % (fold)
\label{sec:array_exercises}

Read over the concepts in this chapter and answer the following questions:
\begin{enumerate}
  \item How is an array different to a standard variable?
  \item How do arrays make it easier to work with multiple values?
  \item Why is 0 the index of the first element in an array?
  \item How many bytes in memory would an array of 10 integers require?
  \item Draw a picture to show how the 10 integer values are stored in memory. Indicate how these values relate to the array.
  \item How can you access an element from an array?
  \item How can you copy the contents of one array into another array?
  \item Why should you pass an array to a parameter by reference, rather than by value?
  \item Can you perform an action on all elements of an array?
  \item How should you \emph{rethink} actions that needs to be performed on all elements in an array?
  \item How does the for loop work together with an array to perform an action on each element in an array?
  \item Why is a string an array? What values are stored in the elements of a string?
  \item Strings in C and Pascal have a single byte overhead. What is this overhead for? Why is it needed?
  \item Can you read/write past the end of an array (i.e. reading/writing to the 11th element when the array only contains 10 elements)? What can happen if you do this?
\end{enumerate}

\bigskip

\csection
{
\begin{enumerate}
  \item In C you need to pass an additional size parameter for any arrays passed in a function/procedure call. Explain why this is needed.
\end{enumerate}
}

\clearpage

Use what you have learnt to read and understand the following code samples.
\begin{enumerate}
  \item Read the C code in \lref{clst:all-below}, or the Pascal code in \lref{plst:all-below}, and answer the following questions:
  \begin{enumerate}
    \item What does the code do?
    \item What would be a good name for this function?
    \item Why is the array passed in using the const modifier?
    \item Will arrays be passed in by value or by reference when this function is called?
  \end{enumerate}
  \begin{figure}[h]
    \csection{\ccode{clst:all-below}{What does this C function do?}{topics/arrays/exercises/all-below.c}}
  \end{figure}
  \begin{figure}[h]
    \passection{\pascode{plst:all-below}{What does this Pascal function do?}{topics/arrays/exercises/AllBelow.pas}}
  \end{figure}
  
  \clearpage
  \item Read the C code in \lref{clst:contains}, or the Pascal code in \lref{plst:contains}, and answer the following questions:
  \begin{enumerate}
    \item What does the code do?
    \item What would be a good name for this function?
    \item What is returned by the function if it is passed ... 
    
    \begin{table}[h]
      \centering
      \begin{tabular}{|c|c|c|}
      \hline
       \textbf{data[]} & \textbf{sz} (C only)  & \textbf{val}  \\
       \hline 
       \{ 1, 2, 3, 4 \} & 4 & 3 \\
       \hline
       \{ 7, 12, 20, 51, -6 \} & 5 & 10 \\
       \hline
       \{ 2, 7, 1 \} & 3 & 1 \\
       \hline
       \{ -1, -2, -3, -4 \} & 4 & -3 \\
       \hline
       \{ 6, 12, 18, 24, 30 \} & 5 & 22 \\
       \hline
      \end{tabular}
    \end{table}
    
      \end{enumerate}
  \begin{figure}[h]
    \csection{\ccode{clst:contains}{What does this C function do?}{topics/arrays/exercises/contains.c}}
  \end{figure}
  \begin{figure}[h]
    \passection{\pascode{plst:contains}{What does this Pascal function do?}{topics/arrays/exercises/Contains.pas}}
  \end{figure}
  
  \clearpage
  \item The following code is designed to find the median (middle value) of a array of numbers, but does it work? Read the C code in \lref{clst:median}, or the Pascal code in \lref{plst:median}, and answer the following questions:
  \begin{enumerate}
    \item Assume that data := [1,2,3,4,5,6,7,8], execute this code by hand and show the steps involved. Explain any shortcuts you take.
    \item What value is returned when the function ends?
    \item You should have noticed that there is a bug in this program. How can it be fixed?
    \item What does this program assume about the data in the array?
    \item Can you think of a simpler solution? Explain your solution.
  \end{enumerate}
  \begin{figure}[h]
    \csection{\ccode{clst:median}{A median function written using C}{topics/arrays/exercises/median.c}}
  \end{figure}
  \begin{figure}[h]
    \passection{\pascode{plst:median}{A median function written using Pascal}{topics/arrays/exercises/Median.pas}}
  \end{figure}

  \clearpage
  \item Read the C code in \lref{clst:insert_at}, or the Pascal code in \lref{plst:insert_at}, and answer the following questions:
  \begin{enumerate}
    \item This procedure alters the array passed to it. Hand execute the procedure for the values shown in the following table. Record your workings as well as the final answer.
    \item What does the code do?
    \item What would be a good name for this procedure?
    \item What would be good names for \texttt{param3} and \texttt{param4}?
    
    \begin{table}[h]
      \centering
      \begin{tabular}{|c|c|c|c|}
      \hline
       \textbf{data[]} & \textbf{sz} (C only)  & \textbf{param3} & \textbf{param4}  \\
       \hline 
       \{ 1, 2, 3, 4 \} & 4 & 2 & 10 \\
       \hline
       \{ 8, 10, 11 \} & 3 & 1 & 9 \\
       \hline
       \{ -1, -2, -3, -4 \} & 4 & 0 & 0 \\
       \hline
       \{ 42, 42 \} & 2 & 1 & 73 \\
       \hline
      \end{tabular}
    \end{table}
    
      \end{enumerate}
  \begin{figure}[h]
    \csection{\ccode{clst:insert_at}{What does this C function do?}{topics/arrays/exercises/insert-at.c}}
  \end{figure}
  \begin{figure}[h]
    \passection{\pascode{plst:insert_at}{What does this Pascal function do?}{topics/arrays/exercises/InsertAt.pas}}
  \end{figure}

\end{enumerate}

\clearpage

Apply what you have learnt to the following tasks:
\begin{enumerate}
  \item Complete the implementation of the Statistics Calculator. Then add the following additional functionality:
  \begin{enumerate}
      \item Add a \textbf{Print All} procedure to print all of the values stored in the array to the Terminal.
      \begin{figure}[h]
        \csection{\csnipet{void print_all(const int data[], int sz)}\ldots}
        \passection{\passnipet{procedure PrintAll(const data: array of Integer);}\ldots}
      \end{figure}

      \item Add a \textbf{Frequency} function that calculates the frequency of a value in the array.
\begin{figure}[h]
  \csection{\csnipet{int frequency(const int data[], int sz, int n)}\ldots}
  \passection{\passnipet{function Frequency(const data: array of Integer; n: Integer): Integer;}\ldots}
\end{figure}

      \item Add a \textbf{Standard Deviation} function.
      \begin{figure}[h]
        \csection{\csnipet{int stddev(const int data[], int sz)}\ldots}
        \passection{\passnipet{function Stddev(const data: array of Integer): Integer;}\ldots}
      \end{figure}

      \item Add a \textbf{Minimum} function.
      \begin{figure}[h]
        \csection{\csnipet{int min(const int data[], int sz)}\ldots}
        \passection{\passnipet{function Min(const data: array of Integer): Integer;}\ldots}
      \end{figure}
  \end{enumerate}
  \clearpage
  \item Implement the Bubble Game from \sref{sub:bubble_game_start_} \nameref{sub:bubble_game_start_}. You will need to do the following:
  \begin{enumerate}
    \item Download the appropriate SwinGame template.
    \item Extract the template into a folder called `Bubbles'
    \item Find a picture of a bubble that is about 30 pixels wide and high.
    \item Place the bubble image in the \texttt{Resources/images} folder in your SwinGame project.
    \item Implement the code from \sref{sub:bubble_game_start_}.
    \item Run the game and you should something similar to \fref{fig:bubble_game}.
  \end{enumerate}
  
  \begin{figure}[h]
     \centering
     \includegraphics[width=0.6\textwidth]{./topics/arrays/exercises/BubblesGame.png} 
     \caption{The start of a bubble game}
     \label{fig:bubble_game}
  \end{figure}
  
  \item Try extending the game in one or more of the following ways:
  
 \begin{enumerate}
  \item Play a sound effect when the bubbles appear.
  \item Add a background image.
  \item Check if the user has clicked the bubble. If they have, play a pop sound effect and place the bubble.
  \item Add a score, and give points for the number of bubbles popped.
  \item Try changing the game dynamics... add gravity and have the bubbles pop when they hit the ground, ending the game. Start new bubbles at the top of the screen.
  \item \ldots
 \end{enumerate}
  
\end{enumerate}

\bigskip

If you want to further your knowledge in this area you can try to answer the following questions. The answers to these questions will require you to think harder, and possibly look at other sources of information.
\begin{enumerate}
  \item Write the code to sort the values in your statistic calculator.
  \item Write a function that calculates the \textbf{Mode}, the most frequent value.
  \item Get your bubbles to bounce off each other. Hint: have a look at SwinGame physics \textbf{Collide Circles} procedure.
  
  
  
  
\end{enumerate}

% section array_exercises (end)


% ===========
% = Project =
% ===========
% \clearpage
% \section{Arrays in the Project} % (fold)
% \label{sec:arrays_in_the_project}

% section arrays_in_the_project (end)
% chapter more_data_types (end)