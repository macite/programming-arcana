\clearpage
\subsection{Pascal Array Parameters} % (fold)
\label{sub:pas_array_parameters}

Pascal allows arrays to be passed as parameters to functions and procedures. Like other parameters these can be passed by value or by reference. 

\passyntax{psynt:array-params}{array parameters}{arrays/fn-array-param}

\begin{figure}[p]
  \passection{\pascode{plst:test-array-passing}{Code illustrating array passing in Pascal}{code/pascal/array/TestArrayPassing.pas}}
\end{figure}

\mynote{
\begin{itemize}
  \item This syntax shows you how to code \nameref{sub:pass_by_reference} into your functions and procedures in Pascal.
  \item See \lref{plst:test-array-passing} for example code.
  \item When declaring array parameters you do not specify the size, this allows the parameter to accept arrays of any size.
\end{itemize}
}

% subsection c_fn_with_array (end)