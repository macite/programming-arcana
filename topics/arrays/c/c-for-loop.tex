\clearpage
\subsection{C For Loop} % (fold)
\label{sub:c_for_loop}

The \nameref{sub:for_loop} in C can do much more than just counting, but that is its primary purpose. You can use this to implement the logic to process each element of an array.

The for loop itself is controlled by three aspects. The first is an initialiser, it sets the first value for the control variable (usually \texttt{i} if you are  using it to index an array). The second part is the condition, the body will run \emph{while} this is true just like a \nameref{sub:c_while_loop}. The third part is a post loop increment, you use this to move the index to the next value. 

The standard for loop is: \csnipet{for(i = 0; i < size; i++)\{...\}}. This can be read as `for \texttt{i} starts at 0, while \texttt{i} is less than \texttt{size}, do the following then increment \texttt{i}'. If \texttt{size} is three then this counts 0, 1, 2. Repeating the body of the loop three times.

\csyntax{csynt:looping-for-loop}{a For Loop}{looping/for-loop}

\csection{\ccode{clst:test-for}{Code illustrating the for loop in C}{code/c/array/test-for.c}}

\mynote{
\begin{itemize}
  \item This is the C syntax for implementing a \nameref{sub:for_loop}.
  \item The \emph{initialiser} of the for loop is used to set the initial values before the loop body is first executed.
  \item The \emph{control} is a \emph{condition}, the loop executes \emph{while}\footnote{In the same way a while loop does, checking the condition to determine if the body should execute or be skipped when the condition is checked.} this is true.
  \item The \emph{increment} is a post loop action, used to move to the next index.
\end{itemize}
}

% subsection c_while_loop (end)